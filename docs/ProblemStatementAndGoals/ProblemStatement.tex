\documentclass{article}

\usepackage{tabularx}
\usepackage{booktabs}

\title{Problem Statement and Goals\\\progname}

\author{\authname}

\date{}

\input{../Comments}
\input{../Common}

\begin{document}

\maketitle

\begin{table}[hp]
  \caption{Revision History} \label{TblRevisionHistory}
  \begin{tabularx}{\textwidth}{llX}
    \toprule
    \textbf{Date} & \textbf{Developer(s)} & \textbf{Change}\\
    \midrule
    2024/09/17 & Whole team & Initial review and formatting of problem statement\\
    2024/09/24 & Whole team & Initial problem statement and goals. Document completed\\
    2025/03/19 & Mitchell Hynes & Update markdown to \LaTeX \\
    \bottomrule
  \end{tabularx}
\end{table}

\section{Problem Statement}

\subsection{Problem}

The City of Hamilton, Water Division requires an application to assist in the management and security of their
water and wastewater stations. Stations are visited by internal staff and external contractors regularly, but no
electronic log of their visit to site exists to confirm work that was performed. This makes it difficult to verify
work completion and accurate invoicing. Each station has many documents associated with it (such as entry
protocols, hazard assessments, etc) which are frequently updated and need to be effortlessly redistributed to
relevant parties as required. This is currently completed manually and is very time consuming for the
stakeholder and prone to human error. Many stations have site specific information, which is difficult to
capture in a single document. Instead, a dynamic application which displays only information relevant to that
station as it is needed would be advantageous. Information needs to be easily accessible to authorized site
visitors.\\
\\
Many documents require signing, and currently it is a manual process to distribute and collect routine
signatures. This functionality currently requires the stakeholder to use multiple applications. The stakeholder
also currently has many different computer applications for documentation storage. Each has a different
standard for storing and managing that information. This leads to duplication and outdated documents in
many locations, rather than a single source of truth. The stakeholder requries contract management tools
including syncing of contract files to the application and automatic alerts when documents, training, or
signatures are set to expire.\\
\\
Directly related to station access is a management system for contractors. This includes the ability to collect
and distribute contract documentation, contact information, training, and other records. A key control
management subsystem would be beneficial to view key distribution in real time, as this is currently managed
in a separate application. A system to authenticate users at stations prior to access would improve visibility
and protection. A digital key system should control access and entry to the station approved from a work
order generated in the work order system.\\

\subsection{Inputs and Outputs}

\textbf{Inputs:}
\begin{itemize}
\item User login information for staff, internal contractors, and visitors
\item Uploading of documents
\item Signing of documents
\item Completion of training
\item Verification of arrivals and departures from the plant for contractors
\item Adding of new staff and contractors
\end{itemize}
\textbf{Outputs:}
\begin{itemize}
\item Station documentation
\item Station maps and access protocol information
\item Station forms
\item Site contact information
\end{itemize}

\subsection{Stakeholders}
The stakeholder for this project is the City of Hamilton, Water Division. Primary stakeholders with the City are
the Facilities team, SCADA (Supervisory Control and Data Acquisition) team, and Corporate Security.
Depending on what is decided during the requirements gathering process, other stakeholders from the City
may need to be included in the project, such as City IT.
\begin{itemize}
\item City of Hamilton, Water Division: Primary stakeholder and client for the project
\item Facilities Team: Subdivision of the primary stakeholder
\item SCADA (Supervisory Control and Data Acquisition): Subdivision of the primary stakeholder
\item Corporate Security: A stakeholder with an interest regarding Hamilton Water station security
\end{itemize}
\subsection{Environment}
\begin{itemize}
\item Software: Windows 10 operating system, android, iOS
\item Hardware: Laptop computers, tablets, smartphones
\end{itemize}

\section{Goals}

\section{Stretch Goals}

\section{Challenge Level and Extras}

\wss{State your expected challenge level (advanced, general or basic).  The
  challenge can come through the required domain knowledge, the
  implementation or
  something else.  Usually the greater the novelty of a project the greater its
  challenge level.  You should include your rationale for the selected level.
  Approval of the level will be part of the discussion with the instructor for
  approving the project.  The challenge level, with the approval (or request) of
the instructor, can be modified over the course of the term.}

\wss{Teams may wish to include extras as either potential bonus grades, or to
  make up for a less advanced challenge level.  Potential extras
  include usability
  testing, code walkthroughs, user documentation, formal proof, GenderMag
  personas, Design Thinking, etc.  Normally the maximum number of extras will be
  two.  Approval of the extras will be part of the discussion with
  the instructor
  for approving the project.  The extras, with the approval (or request) of the
instructor, can be modified over the course of the term.}

\newpage{}

\section*{Appendix --- Reflection}

\wss{Not required for CAS 741}

\input{../Reflection.tex}

\begin{enumerate}
  \item What went well while writing this deliverable?
  \item What pain points did you experience during this deliverable, and how
    did you resolve them?
  \item How did you and your team adjust the scope of your goals to ensure
    they are suitable for a Capstone project (not overly ambitious but also of
    appropriate complexity for a senior design project)?
\end{enumerate}

\end{document}
