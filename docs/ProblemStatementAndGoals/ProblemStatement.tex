\documentclass{article}

\usepackage{tabularx}
\usepackage{booktabs}
\usepackage{longtable}

\title{Problem Statement and Goals\\\progname}

\author{\authname}

\date{}

%% Comments

\usepackage{color}

\newif\ifcomments\commentstrue %displays comments
%\newif\ifcomments\commentsfalse %so that comments do not display

\ifcomments
\newcommand{\authornote}[3]{\textcolor{#1}{[#3 ---#2]}}
\newcommand{\todo}[1]{\textcolor{red}{[TODO: #1]}}
\else
\newcommand{\authornote}[3]{}
\newcommand{\todo}[1]{}
\fi

\newcommand{\wss}[1]{\authornote{blue}{SS}{#1}} 
\newcommand{\plt}[1]{\authornote{magenta}{TPLT}{#1}} %For explanation of the template
\newcommand{\an}[1]{\authornote{cyan}{Author}{#1}}

%% Common Parts

\newcommand{\progname}{ProgName} % PUT YOUR PROGRAM NAME HERE
\newcommand{\authname}{Team \#, Team Name
\\ Student 1 name
\\ Student 2 name
\\ Student 3 name
\\ Student 4 name} % AUTHOR NAMES                  

\usepackage{hyperref}
    \hypersetup{colorlinks=true, linkcolor=blue, citecolor=blue, filecolor=blue,
                urlcolor=blue, unicode=false}
    \urlstyle{same}
                                


\begin{document}

\maketitle

\begin{table}[hp]
  \caption{Revision History} \label{TblRevisionHistory}
  \begin{tabularx}{\textwidth}{llX}
    \toprule
    \textbf{Date} & \textbf{Developer(s)} & \textbf{Change}\\
    \midrule
    2024/09/17 & Whole team & Initial review and formatting of problem statement\\
    2024/09/24 & Whole team & Initial problem statement and goals. Document completed\\
    2025/03/19 & Mitchell Hynes & Update markdown to \LaTeX \\
    \bottomrule
  \end{tabularx}
\end{table}

\section{Problem Statement}

\subsection{Problem}

The City of Hamilton, Water Division requires an application to assist in the management and security of their
water and wastewater stations. Stations are visited by internal staff and external contractors regularly, but no
electronic log of their visit to site exists to confirm work that was performed. This makes it difficult to verify
work completion and accurate invoicing. Each station has many documents associated with it (such as entry
protocols, hazard assessments, etc) which are frequently updated and need to be effortlessly redistributed to
relevant parties as required. This is currently completed manually and is very time consuming for the
stakeholder and prone to human error. Many stations have site specific information, which is difficult to
capture in a single document. Instead, a dynamic application which displays only information relevant to that
station as it is needed would be advantageous. Information needs to be easily accessible to authorized site
visitors.\\
\\
Many documents require signing, and currently it is a manual process to distribute and collect routine
signatures. This functionality currently requires the stakeholder to use multiple applications. The stakeholder
also currently has many different computer applications for documentation storage. Each has a different
standard for storing and managing that information. This leads to duplication and outdated documents in
many locations, rather than a single source of truth. The stakeholder requries contract management tools
including syncing of contract files to the application and automatic alerts when documents, training, or
signatures are set to expire.\\
\\
Directly related to station access is a management system for contractors. This includes the ability to collect
and distribute contract documentation, contact information, training, and other records. A key control
management subsystem would be beneficial to view key distribution in real time, as this is currently managed
in a separate application. A system to authenticate users at stations prior to access would improve visibility
and protection. A digital key system should control access and entry to the station approved from a work
order generated in the work order system.\\

\subsection{Inputs and Outputs}

\textbf{Inputs:}
\begin{itemize}
\item User login information for staff, internal contractors, and visitors
\item Uploading of documents
\item Signing of documents
\item Completion of training
\item Verification of arrivals and departures from the plant for contractors
\item Adding of new staff and contractors
\end{itemize}
\textbf{Outputs:}
\begin{itemize}
\item Station documentation
\item Station maps and access protocol information
\item Station forms
\item Site contact information
\end{itemize}

\subsection{Stakeholders}
The stakeholder for this project is the City of Hamilton, Water Division. Primary stakeholders with the City are
the Facilities team, SCADA (Supervisory Control and Data Acquisition) team, and Corporate Security.
Depending on what is decided during the requirements gathering process, other stakeholders from the City
may need to be included in the project, such as City IT.
\begin{itemize}
\item City of Hamilton, Water Division: Primary stakeholder and client for the project
\item Facilities Team: Subdivision of the primary stakeholder
\item SCADA (Supervisory Control and Data Acquisition): Subdivision of the primary stakeholder
\item Corporate Security: A stakeholder with an interest regarding Hamilton Water station security
\end{itemize}
\subsection{Environment}
\begin{itemize}
\item Software: Windows 10 operating system, android, iOS
\item Hardware: Laptop computers, tablets, smartphones
\end{itemize}

\section{Goals}
\begin{longtable}{|m{5cm}|m{8cm}|}
\hline
The system enables the synchronization of files across distributed consumers & 
This is a basic goal that must be achieved for the proposed system to be
useful. This involves both file synchronization and conflict resolution\\
\hline
Intuitive GUI with high learnability & The interface of the system should be easy to understand for first-time users
as many of them will be contractors logging in for the first time\\
\hline
Accurate verification of users at stations & The application should be able to accurately verify that a user is at a specific
station. This will provide visibility into who is at what station\\
\hline
Should integrate with current business practices of client & The project aims to assist the stakeholders and 
provide value as a more efficient and secure method of completing existing tasks.
It should not interfere with current business processes or create additional workload\\
\hline
\end{longtable}
\section{Stretch Goals}

\begin{longtable}{|m{5cm}|m{8cm}|}
\hline
Demonstrate the advantage of a
single centralized platform instead
of multiple disconnected platforms & Hamilton Water has applications which do not communicate
between each other. Loss of information and working in “silos” is
common as a result. Demonstrating the benefits of a platform
which integrates the features of separate applications into one
would be advantageous.\\
\hline
Expand the platform to be a contract management system,
capable of having contract management tools accessible to authenticated project managers. &
This would enable a greater integration of station and contractor
documentation directly into the projects that utilize this information.\\
\hline
\end{longtable}

\section{Challenge Level and Extras}

The challenge level for this project is a general challenge level. This designation was decided because we do
not believe there will be a research element required. The extras for this project will be conducting user
testing and developing a user manual. These extras are a good choice for this project as it is being developed
for a real client, with the objective of creating a usable tool. Extensive user testing and documentation will be
critical for its long term use and maintenance by the client.

\newpage{}

\section*{Appendix --- Reflection}

\begin{enumerate}
  \item What went well while writing this deliverable?
  
\begin{itemize}
\item Sections were divided efficiently, and everyone was aligned on what was to be written and who would
write each section
\item We had great and open communication with the City during this deliverable which greatly aided us in
developing the problem statement
\item Our team effectively documented the actions we took and helped each other learn how to use the git
workflow for developing our project
\item Ample time was given to complete the deliverable
\item Multiple meetings were set to review work and hold accountability for work done
\item Everyone adhered to the self-imposed deadlines set for each section of the deliverable
\end{itemize}

  \item What pain points did you experience during this deliverable, and how
    did you resolve them?
  \begin{itemize}
  \item Some members had a better idea of the problem space, while others were more familiar with the
  development of software solutions. We overcame this by dividing members between this deliverable
  and the Development Plan
  \item Version control does not allow live collaboration on written documents in the same way that Google
  Docs or Microsoft Word would. It was decided that the text for the different sections would be drafted
  on Google Docs before being committed to GitHub  
  \end{itemize}

  \item How did you and your team adjust the scope of your goals to ensure
    they are suitable for a Capstone project (not overly ambitious but also of
    appropriate complexity for a senior design project)?\\
    \\
To adjust the scope of the goals, we first discussed the project with the stakeholders to figure out what
their main objectives were with the project. We then discussed as a team to figure out how to narrow
these down to the most important and achievable goals and brought them back to the stakeholders to
make sure they were acceptable. The goals we decided on are appropriate in complexity for a senior
design project, are measurable, and focus on aspects of the system that are most important to the
stakeholder.
\end{enumerate}

\end{document}
