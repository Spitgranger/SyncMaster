\documentclass[12pt, titlepage]{article}

\usepackage{booktabs}
\usepackage{tabularx}
\usepackage{hyperref}
\hypersetup{
  colorlinks,
  citecolor=black,
  filecolor=black,
  linkcolor=red,
  urlcolor=blue
}
\usepackage{longtable}
\usepackage[round]{natbib}

%% Comments

\usepackage{color}

\newif\ifcomments\commentstrue %displays comments
%\newif\ifcomments\commentsfalse %so that comments do not display

\ifcomments
\newcommand{\authornote}[3]{\textcolor{#1}{[#3 ---#2]}}
\newcommand{\todo}[1]{\textcolor{red}{[TODO: #1]}}
\else
\newcommand{\authornote}[3]{}
\newcommand{\todo}[1]{}
\fi

\newcommand{\wss}[1]{\authornote{blue}{SS}{#1}} 
\newcommand{\plt}[1]{\authornote{magenta}{TPLT}{#1}} %For explanation of the template
\newcommand{\an}[1]{\authornote{cyan}{Author}{#1}}

%% Common Parts

\newcommand{\progname}{ProgName} % PUT YOUR PROGRAM NAME HERE
\newcommand{\authname}{Team \#, Team Name
\\ Student 1 name
\\ Student 2 name
\\ Student 3 name
\\ Student 4 name} % AUTHOR NAMES                  

\usepackage{hyperref}
    \hypersetup{colorlinks=true, linkcolor=blue, citecolor=blue, filecolor=blue,
                urlcolor=blue, unicode=false}
    \urlstyle{same}
                                


\begin{document}

\title{Verification and Validation Report: \progname}
\author{\authname}
\date{\today}

\maketitle

\pagenumbering{roman}

\section{Revision History}

\begin{tabularx}{\textwidth}{p{3cm}p{2cm}X}
  \toprule {\bf Date} & {\bf Version} & {\bf Notes}\\
  \midrule
  Date 1 & 1.0 & Notes\\
  Date 2 & 1.1 & Notes\\
  \bottomrule
\end{tabularx}

~\newpage

\section{Symbols, Abbreviations and Acronyms}

\renewcommand{\arraystretch}{1.2}
\begin{tabular}{l l}
  \toprule
  \textbf{symbol} & \textbf{description}\\
  \midrule
  T & Test\\
  \bottomrule
\end{tabular}\\

\wss{symbols, abbreviations or acronyms -- you can reference the SRS
tables if needed}

\newpage

\tableofcontents

\listoftables %if appropriate

\listoffigures %if appropriate

\newpage

\pagenumbering{arabic}

This document ...

\section{Functional Requirements Evaluation}

\section{Nonfunctional Requirements Evaluation}

\subsection{Usability}

\subsection{Performance}

\subsection{Maintainability and Support}

\begin{longtable}{|m{2cm}|m{1cm}|m{6cm}|m{3cm}|}
  \hline
  \textbf{Test. ID} & \textbf{Input} & \textbf{Expected Output} &
  \textbf{Result} \\
  \hline
  TC-MS-4 & - & There is 95\% line coverage and 90\% branch coverage
  on our code & Pass, this is being checked in GitHub actions\\ \hline
  TC-MS-5 & - & All functional requirements have a corresponding unit
  test & Pass\\ \hline
  TC-MS-6 & - & Contribution guidelines and maintainer documentation
  of system approved by the City of Hamilton & Fail, not yet approved\\ \hline
  TC-MS-7 & - & User manual exists and has been approved by the City
  of Hamilton & Fail, not yet created\\ \hline
  TC-MS-8 & - & OAS3 compliant documentation has been provided for
  all API's & Fail, not yet created\\ \hline
  TC-MS-9 & - & Internal abstractions (classes and functions) in the
  system have documentation associated with them & Pass, linter
  checks this\\ \hline
  TC-MS-10 & - & Documentation for deployment of the system exists &
  Fail, not yet created\\ \hline
  \caption{Maintainability and Support Test Cases}
\end{longtable}

\subsection{etc.}

\section{Comparison to Existing Implementation}

This section will not be appropriate for every project.

\section{Unit Testing}

Note: Unit tests all use \href{https://pypi.org/project/moto/}{moto}
to mock AWS services.

\subsection{Database Interaction Module Unit Tests}

\begin{longtable}{|m{2cm}|m{10cm}|m{1.4cm}|}
  \hline
  \textbf{Test. ID} & \textbf{Description} & \textbf{Result} \\ \hline
  UT-DB1 & Use the DBTable.put method to create an item, and check
  that it exists in the database afterwards & Pass\\ \hline
  UT-DB2 & Use the DBTable.put method with a failing precondition,
  and check that a ConditionCheckFailed is raised, and the item is
  not created & Pass\\ \hline
  UT-DB3 & Use the DBTable.put method with a permission error from
  AWS, and check that a PermissionException is raised, and the item
  is not created & Pass\\ \hline
  UT-DB4 & Use the DBTable.put method with an arbitrary error from
  AWS, and check that a ExternalServiceException is raised, and the
  item is not created & Pass\\ \hline
  UT-DB5 & Use the DBTable.get method to get a pre-existing item &
  Pass\\ \hline
  UT-DB6 & Use the DBTable.get method to get an item which does not
  exist, and check that a ResourceNotFound is raised & Pass\\ \hline
  UT-DB7 & Use the DBTable.get method with an arbitrary error from
  AWS, and check that a ExternalServiceException is raised & Pass\\ \hline
  UT-DB8 & Use the DBTable.delete method on a pre-existing item, and
  check that the item is no longer in the database & Pass\\ \hline
  UT-DB9 & Use the DBTable.delete method with a failing
  precondition, and check that a ConditionCheckFailed is raised, and
  the item is not deleted & Pass\\ \hline
  UT-DB10 & Use the DBTable.delete method with a permission error
  from AWS, and check that a PermissionException is raised, and the
  item is not deleted & Pass\\ \hline
  UT-DB11 & Use the DBTable.delete method with an arbitrary error
  from AWS, and check that a ExternalServiceException is raised, and
  the item is not deleted & Pass\\ \hline
  UT-DB12 & Use the DBTable.update method to update an item adding
  new attributes, and modifying existing ones, and check that these
  changes are reflected in the database & Pass\\ \hline
  UT-DB13 & Use the DBTable.update method to update an item removing
  some attributes and check that these changes are reflected in the
  database & Pass\\ \hline
  UT-DB14 & Use the DBTable.update method with a failing
  precondition, and check that a ConditionCheckFailed is raised, and
  the item is not updated & Pass\\ \hline
  UT-DB15 & Use the DBTable.update method with a permission error
  from AWS, and check that a PermissionException is raised, and the
  item is not updated & Pass\\ \hline
  UT-DB16 & Use the DBTable.update method with an arbitrary error
  from AWS, and check that a ExternalServiceException is raised, and
  the item is not updated & Pass\\ \hline
  UT-DB17 & Use the DBTable.query method, and ensure the returned
  items all match the query criterion & Pass\\ \hline
  UT-DB18 & Use the DBTable.query method, under a different GSI, and
  ensure the returned items all match the query criterion & Pass\\ \hline
  UT-DB19 & Use the DBTable.query method, with a reversed query
  direction, and ensure the returned items all match the query
  criterion & Pass\\ \hline
  UT-DB20 & Use the DBTable.query method, with using the start\_key
  of an existing db item, to determine where to start the query from,
  and ensure the returned items all match the query criterion & Pass\\ \hline
  UT-DB21 & Use the DBTable.query method, using a filter expression
  alongside a key condition, and ensure the returned items all match
  the query criterion & Pass\\ \hline
  UT-DB22 & Use the DBTable.query method, with an invalid key
  condition, and ensure that a ConditionValidationError is raised &
  Pass\\ \hline
  UT-DB23 & Use the DBTable.query method with an arbitrary error
  from AWS, and check that a ExternalServiceException is raised & Pass\\ \hline
  \caption{Unit Test Cases for Database Interaction Module}
\end{longtable}

\subsection{File Storage Interaction Module Unit Tests}

\begin{longtable}{|m{2cm}|m{10cm}|m{1.4cm}|}
  \hline
  \textbf{Test. ID} & \textbf{Description} & \textbf{Result} \\ \hline
  UT-FS1 & Use the S3Bucket.create\_upload\_url method to create an
  upload url, upload some content to the url, and check that the
  content exists in the S3 bucket & Pass\\ \hline
  UT-FS2 & Use the S3Bucket.create\_upload\_url method with a
  S3Bucket object that does not have write permissions, and check
  that a PermissionException gets raised & Pass\\ \hline
  UT-FS3 & Use the S3Bucket.create\_get\_url method for a
  pre-existing file in S3, and check that the file content can be
  accessed through the url, and check that a PermissionException gets
  raised & Pass\\ \hline
  UT-FS4 & Use the S3Bucket.delete method on a pre-existing file in
  S3, and check that the file no longer exists & Pass\\ \hline
  UT-FS5 & Use the S3Bucket.delete method on a pre-existing file in
  S3, with an ETag mismatch, and check that a ResourceNotFound is
  raised & Pass\\ \hline
  UT-FS6 & Use the S3Bucket.delete method with a permission error,
  and check that a PermissionException is raised & Pass\\ \hline
  UT-FS7 & Use the S3Bucket.delete method with an arbitrary error
  from AWS, and check that a ExternalServiceException is raised & Pass\\ \hline
  \caption{Unit Test Cases for File Storage Interaction Module}
\end{longtable}

\subsection{Location Verification Module Unit Tests}

\begin{longtable}{|m{2cm}|m{10cm}|m{1.4cm}|}
  \hline
  \textbf{Test. ID} & \textbf{Description} & \textbf{Result} \\ \hline
  UT-LV1 & Use the verify\_location method to verify a coordinate
  within a defined radius, and ensure the return value is True & Pass\\ \hline
  UT-LV2 & Use the verify\_location method to verify a coordinate
  outside a defined radius, and ensure the return value is False & Pass\\ \hline
  UT-LV3 & Use the verify\_location method to verify a coordinate
  just on the boundary of a defined radius, and ensure the return
  value is True & Pass\\ \hline
  \caption{Unit Test Cases for Location Verification Module}
\end{longtable}

\subsection{User Authentication Module Unit Tests}

\begin{longtable}{|m{2cm}|m{10cm}|m{1.4cm}|}
  \hline
  \textbf{Test. ID} & \textbf{Description} & \textbf{Result} \\ \hline
  UT-UA1 & Attempt to authenticate with a pre-existing user, and
  ensure that a token is successfully generated & Pass\\ \hline
  UT-UA2 & Attempt to authenticate with an invalid location, and
  ensure that an UnauthorizedException is raised & Pass\\ \hline
  UT-UA3 & Attempt to authenticate with an initial one-time
  password, and ensure that an ForceChangePasswordException is raised
  & Pass\\ \hline
  UT-UA4 & Attempt to authenticate with the wrong password, and
  ensure that an UnauthorizedException is raised & Pass\\ \hline
  UT-UA5 & Attempt to authenticate with a user that does not exist,
  and ensure that a ResourceNotFound is raised & Pass\\ \hline
  UT-UA6 & Attempt to authenticate with an arbitrary error from AWS,
  and ensure that a ExternalServiceException is raised & Pass\\ \hline
  UT-UA7 & Attempt to signout a user, with a valid access token, and
  ensure that the token is successfully invalidated & Pass\\ \hline
  UT-UA8 & Attempt to signout a user, with an invalid access token,
  and ensure that a BadRequestException is raised & Pass\\ \hline
  UT-UA9 & Attempt to signout a user, with an arbitrary error from
  AWS, and ensure that a ExternalServiceException is raised & Pass\\ \hline
  \caption{Unit Test Cases for User Authentication Module}
\end{longtable}

\subsection{User Management Module Unit Tests}

\begin{longtable}{|m{2cm}|m{10cm}|m{1.4cm}|}
  \hline
  \textbf{Test. ID} & \textbf{Description} & \textbf{Result} \\ \hline
  UT-UM1 & Create an admin, employee, and contractor account, using
  admin credentials & Pass\\ \hline
  UT-UM2 & Create a user, using admin credentials, where there is
  already another user using the same email, and check that a
  ConflictException is raised & Pass\\ \hline
  UT-UM3 & Create a user, using employee and contractor credentials,
  and ensure an UnauthorizedException is raised & Pass\\ \hline
  UT-UM4 & Create a user, with an arbitrary error occuring from AWS,
  and check that an ExternalServiceException is raised & Pass\\ \hline
  UT-UM5 & Get the details of an existing user, using admin and
  employee credentials & Pass\\ \hline
  UT-UM6 & Get the details of a user that does not exist, using
  admin and employee credentials, and ensure a ResourceNotFound is
  raised & Pass\\ \hline
  UT-UM7 & Use contractor credentials to get the details of their
  own user & Pass\\ \hline
  UT-UM8 & Use contractor credentials to get the details of another
  user, and ensure an UnauthorizedException is raised & Pass\\ \hline
  UT-UM9 & Get the details of a user, with an arbitrary error
  occuring from AWS, and check that an ExternalServiceException is
  raised & Pass\\ \hline
  UT-UM10 & Update a user, using admin credentials & Pass\\ \hline
  UT-UM11 & Update a user, using admin credentials, where the user
  to update does not exist, and check that a ResourceNotFound is
  raised & Pass\\ \hline
  UT-UM12 & Update a user, using employee and contractor
  credentials, and ensure an UnauthorizedException is raised & Pass\\ \hline
  UT-UM13 & Update a user, with an arbitrary error occuring from
  AWS, and check that an ExternalServiceException is raised & Pass\\ \hline
  UT-UM14 & Delete a user, using admin credentials & Pass\\ \hline
  UT-UM15 & Delete a user, using admin credentials, where the user
  to delete does not exist, and check that a ResourceNotFound is
  raised & Pass\\ \hline
  UT-UM16 & Delete a user, using employee and contractor
  credentials, and ensure an UnauthorizedException is raised & Pass\\ \hline
  UT-UM17 & Delete a user, with an arbitrary error occuring from
  AWS, and check that an ExternalServiceException is raised & Pass\\ \hline
  \caption{Unit Test Cases for User Management Module}
\end{longtable}

\subsection{Logging Module Unit Tests}

\begin{longtable}{|m{2cm}|m{10cm}|m{1.4cm}|}
  \hline
  \textbf{Test. ID} & \textbf{Description} & \textbf{Result} \\ \hline
  UT-LG1 & Create a site visit log, and ensure it gets added to the
  database & Pass\\ \hline
  UT-LG2 & Create a site visit log, with the same site id, user id,
  and entry time as an existing log, and ensure a ResourceConflict is
  raised & Pass\\ \hline
  UT-LG3 & Add an exit time to an existing log with no logged exit
  time, and ensure the change is reflected in the database & Pass\\ \hline
  UT-LG4 & Add an exit time to an existing log with no logged exit
  time, but an entry time prior to the exit time we are attempting to
  add, and ensure a TimeConsistencyException is raised & Pass\\ \hline
  UT-LG5 & Add an exit time to an existing log with a logged exit
  time, and ensure an ExitTimeConflict is raised & Pass\\ \hline
  UT-LG6 & Add an exit time, where there are no existing logs for
  the user at the given site, and ensure an ResourceNotFound is
  raised & Pass\\ \hline
  UT-LG7 & Get a list of all site visit logs, using admin and
  employee credentials & Pass\\ \hline
  UT-LG8 & Get a list of all site visit logs, using contractor
  credentials, and ensure that an UnauthorizedException is raised &
  Pass\\ \hline
  UT-LG9 & Get a list of all site visit logs using filters to only
  include logs modified between two given dates, and ensure that logs
  are correctly filtered & Pass\\ \hline
  UT-LG10 & Get a list of all site visit logs using a database
  start\_key to start the listing from a certain log, and ensure that
  logs start from the correct location & Pass\\ \hline
  \caption{Unit Test Cases for Logging Module}
\end{longtable}

\subsection{Document Management Module Unit Tests}

\begin{longtable}{|m{2cm}|m{10cm}|m{1.4cm}|}
  \hline
  \textbf{Test. ID} & \textbf{Description} & \textbf{Result} \\ \hline
  UT-DM1 & Create a document, and ensure it gets added to the
  database with the correct information & Pass\\ \hline
  UT-DM2 & Create a document with the name parent folder and document names as
  an existing document and check that a ResourceConflict is raised &
  Pass\\ \hline
  UT-DM3 & Get a single document for a particular site id and parent
  folder & Pass\\ \hline
  UT-DM4 & Get a list of documents including both files and folders for a
  particular site id and parent folder & Pass\\ \hline
  UT-DM5 & Get a list of documents for a site id and folder that has no
  documents and ensure that an empty list is returned & Pass\\ \hline
  UT-DM6 & Generate a presigned url that is used to upload a binary file to
  Amazon S3 give a particular S3 key & Pass\\ \hline
  UT-DM7 & Delete an existing file for a given site id and parent folder given
  that the file exists & Pass\\ \hline
  UT-DM8 & Delete a document of type folder that has with documents
  inside for a given site id
  and parent folder and ensure that both the folder and its contents are
  recursively deleted & Pass\\ \hline
  \caption{Unit Test Cases for Logging Module}
\end{longtable}

\section{Changes Due to Testing}

\wss{This section should highlight how feedback from the users and from
  the supervisor (when one exists) shaped the final product.  In particular
  the feedback from the Rev 0 demo to the supervisor (or to potential users)
should be highlighted.}

\section{Automated Testing}
N/A, the only automated testing we have is the unit tests.

\section{Trace to Requirements}

\begin{longtable}{|l|l|}
  \hline
  \textbf{Req. ID} & \textbf{Test ID's} \\
  \hline
  FR1 & TC-FR1, TC-FR2, UT-DM1\\ \hline
  FR3 & TC-FR3, UT-UM3\\ \hline
  FR4 & TC-FR4, UT-LG7\\ \hline
  FR5 & TC-FR5, UT-UA2\\ \hline
  FR6 & TC-FR6, UT-DM1\\ \hline
  FR7 & TC-FR8, UT-DM4, UT-DM6\\ \hline
  FR8 & TC-FR7\\ \hline
  FR9 & TC-FR9\\ \hline
  LF-AP1 & TC-LF-1 \\ \hline
  LF-ST1 & TC-LF-2 \\ \hline
  UH-EU1 & TC-EU1\\ \hline
  UH-EU2 & TC-EU2\\ \hline
  UH-LR1 & TC-LR1\\ \hline
  UH-LR2 & TC-LR2\\ \hline
  UH-UP1 & TC-UP1\\ \hline
  UH-UP2 & TC-UP2\\ \hline
  UH-AS1 & TC-AS1\\ \hline
  PR-SL1 & TC-PR-1\\ \hline
  PR-SL3 & TC-PR-2\\ \hline
  PR-SC1 & TC-PR-3\\ \hline
  PR-SC2 & TC-PR-4\\ \hline
  PR-PA1 & TC-PR-5\\ \hline
  PR-RFT1 & TC-PR-7\\ \hline
  PR-CR2 & TC-PR-10\\ \hline
  PR-SE1 & TC-PR-11\\ \hline
  OE-PE1 & TC-OE-1 \\ \hline
  OE-WE1 & TC-OE-2 \\ \hline
  OE-WE2 & TC-OE-2 \\ \hline
  OE-REL1 & TC-OE-4 \\ \hline
  OE-REL2 & TC-OE-4 \\ \hline
  OE-REL3 & TC-OE-4 \\ \hline
  OE-REL4 & TC-OE-4 \\ \hline
  MS-MTN4 & TC-MS-4 \\ \hline
  MS-MTN5 & TC-MS-5 \\ \hline
  MS-MTN6 & TC-MS-6 \\ \hline
  MS-SUP1 & TC-MS-7 \\ \hline
  MS-SUP2 & TC-MS-8 \\ \hline
  MS-SUP3 & TC-MS-9 \\ \hline
  MS-SUP4 & TC-MS-10 \\ \hline
  MS-ADP1 & TC-LF-2 \\ \hline
  MS-ADP2 & TC-LF-2 \\ \hline
  MS-ADP3 & TC-LF-2 \\ \hline
  SR-AR1 & TC-SS-1 \\ \hline
  SR-AR2 & TC-SS-1 \\ \hline
  SR-AR3 & TC-SS-1 \\ \hline
  SR-AR4 & TC-SS-2 \\ \hline
  SR-IR1 & TC-SS-2 \\ \hline
  SR-IR3 & TC-SS-4 \\ \hline
  SR-PR1 & TC-SS-5\\ \hline
  SR-AU1 & TC-SS-6 \\ \hline
  SR-IMR1 & TC-SS-7 \\ \hline
  SR-S1 & TC-SS-8 \\ \hline
  CR-CR1 & TC-CR-1 \\ \hline
  \caption{Requirements to Test Case Traceability Matrix}
\end{longtable}

\section{Trace to Modules}

Note: * indicates that any test prefixed with the test case ID,
covers the given module

\begin{longtable}{|l|l|}
  \hline
  \textbf{Module} & \textbf{Test ID's} \\
  \hline
  Database Interaction & UT-DB*\\ \hline
  Logging & UT-LG*\\ \hline
  File Storage & UT-FS*\\ \hline
  Location Verification & UT-LV*\\ \hline
  User Management & UT-UM*\\ \hline
  User Authentication & UT-AU*\\ \hline
  Document Management & UT-DM*\\ \hline
  \caption{Module to Test Case Traceability Matrix}
\end{longtable}

\section{Code Coverage Metrics}

The unit testing achieves 95\% line coverage and 90\% branch
coverage. This is checked by our GitHub Actions on every pull request.

\bibliographystyle{plainnat}
\bibliography{../../refs/References}

\newpage{}
\section*{Appendix --- Reflection}

The information in this section will be used to evaluate the team members on the
graduate attribute of Reflection.

The purpose of reflection questions is to give you a chance to assess your own
learning and that of your group as a whole, and to find ways to improve in the
future. Reflection is an important part of the learning process.  Reflection is
also an essential component of a successful software development process.  

Reflections are most interesting and useful when they're honest, even if the
stories they tell are imperfect. You will be marked based on your depth of
thought and analysis, and not based on the content of the reflections
themselves. Thus, for full marks we encourage you to answer openly and honestly
and to avoid simply writing ``what you think the evaluator wants to hear.''

Please answer the following questions.  Some questions can be answered on the
team level, but where appropriate, each team member should write their own
response:


\begin{enumerate}
  \item What went well while writing this deliverable?
  \item What pain points did you experience during this deliverable, and how
    did you resolve them?
  \item Which parts of this document stemmed from speaking to your client(s) or
    a proxy (e.g. your peers)? Which ones were not, and why?
  \item In what ways was the Verification and Validation (VnV) Plan different
    from the activities that were actually conducted for VnV?  If there were
    differences, what changes required the modification in the plan?  Why did
    these changes occur?  Would you be able to anticipate these
    changes in future
    projects?  If there weren't any differences, how was your team
    able to clearly
    predict a feasible amount of effort and the right tasks needed to build the
    evidence that demonstrates the required quality?  (It is expected that most
    teams will have had to deviate from their original VnV Plan.)
\end{enumerate}

\end{document}
