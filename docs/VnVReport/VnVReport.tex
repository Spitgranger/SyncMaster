\documentclass[12pt, titlepage]{article}

\usepackage{booktabs}
\usepackage{tabularx}
\usepackage{hyperref}
\hypersetup{
  colorlinks,
  citecolor=black,
  filecolor=black,
  linkcolor=red,
  urlcolor=blue
}
\usepackage{longtable}
\usepackage[round]{natbib}

%% Comments

\usepackage{color}

\newif\ifcomments\commentstrue %displays comments
%\newif\ifcomments\commentsfalse %so that comments do not display

\ifcomments
\newcommand{\authornote}[3]{\textcolor{#1}{[#3 ---#2]}}
\newcommand{\todo}[1]{\textcolor{red}{[TODO: #1]}}
\else
\newcommand{\authornote}[3]{}
\newcommand{\todo}[1]{}
\fi

\newcommand{\wss}[1]{\authornote{blue}{SS}{#1}} 
\newcommand{\plt}[1]{\authornote{magenta}{TPLT}{#1}} %For explanation of the template
\newcommand{\an}[1]{\authornote{cyan}{Author}{#1}}

%% Common Parts

\newcommand{\progname}{ProgName} % PUT YOUR PROGRAM NAME HERE
\newcommand{\authname}{Team \#, Team Name
\\ Student 1 name
\\ Student 2 name
\\ Student 3 name
\\ Student 4 name} % AUTHOR NAMES                  

\usepackage{hyperref}
    \hypersetup{colorlinks=true, linkcolor=blue, citecolor=blue, filecolor=blue,
                urlcolor=blue, unicode=false}
    \urlstyle{same}
                                


\begin{document}

\title{Verification and Validation Report: \progname}
\author{\authname}
\date{\today}

\maketitle

\pagenumbering{roman}

\section{Revision History}

\begin{tabularx}{\textwidth}{p{3cm}p{2cm}X}
  \toprule {\bf Date} & {\bf Version} & {\bf Notes}\\
  \midrule
  Date 1 & 1.0 & Notes\\
  Date 2 & 1.1 & Notes\\
  \bottomrule
\end{tabularx}

~\newpage

\section{Symbols, Abbreviations and Acronyms}

\renewcommand{\arraystretch}{1.2}
\begin{tabular}{l l}
  \toprule
  \textbf{symbol} & \textbf{description}\\
  \midrule
  T & Test\\
  \bottomrule
\end{tabular}\\

\wss{symbols, abbreviations or acronyms -- you can reference the SRS
tables if needed}

\newpage

\tableofcontents

\listoftables %if appropriate

\listoffigures %if appropriate

\newpage

\pagenumbering{arabic}

This document ...

\section{Functional Requirements Evaluation}

\section{Nonfunctional Requirements Evaluation}

\subsection{Usability}

\subsection{Performance}

\subsection{Maintainability and Support}

\begin{longtable}{|m{2cm}|m{1cm}|m{6cm}|m{3cm}|}
  \hline
  \textbf{Test. ID} & \textbf{Input} & \textbf{Expected Output} &
  \textbf{Result} \\
  \hline
  TC-MS-4 & - & There is 95\% line coverage and 90\% branch coverage
  on our code & Pass, this is being checked in GitHub actions\\ \hline
  TC-MS-5 & - & All functional requirements have a corresponding unit
  test & Pass\\ \hline
  TC-MS-6 & - & Contribution guidelines and maintainer documentation
  of system approved by the City of Hamilton & Fail, not yet approved\\ \hline
  TC-MS-7 & - & User manual exists and has been approved by the City
  of Hamilton & Fail, not yet created\\ \hline
  TC-MS-8 & - & OAS3 compliant documentation has been provided for
  all API's & Fail, not yet created\\ \hline
  TC-MS-9 & - & Internal abstractions (classes and functions) in the
  system have documentation associated with them & Pass, linter
  checks this\\ \hline
  TC-MS-10 & - & Documentation for deployment of the system exists &
  Fail, not yet created\\ \hline
  \caption{Maintainability and Support Test Cases}
\end{longtable}

\subsection{etc.}

\section{Comparison to Existing Implementation}

This section will not be appropriate for every project.

\section{Unit Testing}

\section{Changes Due to Testing}

\wss{This section should highlight how feedback from the users and from
  the supervisor (when one exists) shaped the final product.  In particular
  the feedback from the Rev 0 demo to the supervisor (or to potential users)
should be highlighted.}

\section{Automated Testing}
N/A, the only automated testing we have is the unit tests.

\section{Trace to Requirements}

\begin{longtable}{|l|l|}
  \hline
  \textbf{Req. ID} & \textbf{Test ID's} \\
  \hline
  FR1 & TC-FR1, TC-FR2\\ \hline
  FR3 & TC-FR3\\ \hline
  FR4 & TC-FR4\\ \hline
  FR5 & TC-FR5\\ \hline
  FR6 & TC-FR6\\ \hline
  FR7 & TC-FR8\\ \hline
  FR8 & TC-FR7\\ \hline
  FR9 & TC-FR9\\ \hline
  LF-AP1 & TC-LF-1 \\ \hline
  LF-ST1 & TC-LF-2 \\ \hline
  UH-EU1 & TC-EU1\\ \hline
  UH-EU2 & TC-EU2\\ \hline
  UH-LR1 & TC-LR1\\ \hline
  UH-LR2 & TC-LR2\\ \hline
  UH-UP1 & TC-UP1\\ \hline
  UH-UP2 & TC-UP2\\ \hline
  UH-AS1 & TC-AS1\\ \hline
  PR-SL1 & TC-PR-1\\ \hline
  PR-SL3 & TC-PR-2\\ \hline
  PR-SC1 & TC-PR-3\\ \hline
  PR-SC2 & TC-PR-4\\ \hline
  PR-PA1 & TC-PR-5\\ \hline
  PR-RFT1 & TC-PR-7\\ \hline
  PR-CR2 & TC-PR-10\\ \hline
  PR-SE1 & TC-PR-11\\ \hline
  OE-PE1 & TC-OE-1 \\ \hline
  OE-WE1 & TC-OE-2 \\ \hline
  OE-WE2 & TC-OE-2 \\ \hline
  OE-REL1 & TC-OE-4 \\ \hline
  OE-REL2 & TC-OE-4 \\ \hline
  OE-REL3 & TC-OE-4 \\ \hline
  OE-REL4 & TC-OE-4 \\ \hline
  MS-MTN4 & TC-MS-4 \\ \hline
  MS-MTN5 & TC-MS-5 \\ \hline
  MS-MTN6 & TC-MS-6 \\ \hline
  MS-SUP1 & TC-MS-7 \\ \hline
  MS-SUP2 & TC-MS-8 \\ \hline
  MS-SUP3 & TC-MS-9 \\ \hline
  MS-SUP4 & TC-MS-10 \\ \hline
  MS-ADP1 & TC-LF-2 \\ \hline
  MS-ADP2 & TC-LF-2 \\ \hline
  MS-ADP3 & TC-LF-2 \\ \hline
  SR-AR1 & TC-SS-1 \\ \hline
  SR-AR2 & TC-SS-1 \\ \hline
  SR-AR3 & TC-SS-1 \\ \hline
  SR-AR4 & TC-SS-2 \\ \hline
  SR-IR1 & TC-SS-2 \\ \hline
  SR-IR3 & TC-SS-4 \\ \hline
  SR-PR1 & TC-SS-5\\ \hline
  SR-AU1 & TC-SS-6 \\ \hline
  SR-IMR1 & TC-SS-7 \\ \hline
  SR-S1 & TC-SS-8 \\ \hline
  CR-CR1 & TC-CR-1 \\ \hline
  \caption{Requirements to Test Case Traceability Matrix}
\end{longtable}

\section{Trace to Modules}

\section{Code Coverage Metrics}

The unit testing achieves 95\% line coverage and 90\% branch
coverage. This is checked by our GitHub Actions on every pull request.

\bibliographystyle{plainnat}
\bibliography{../../refs/References}

\newpage{}
\section*{Appendix --- Reflection}

The information in this section will be used to evaluate the team members on the
graduate attribute of Reflection.

The purpose of reflection questions is to give you a chance to assess your own
learning and that of your group as a whole, and to find ways to improve in the
future. Reflection is an important part of the learning process.  Reflection is
also an essential component of a successful software development process.  

Reflections are most interesting and useful when they're honest, even if the
stories they tell are imperfect. You will be marked based on your depth of
thought and analysis, and not based on the content of the reflections
themselves. Thus, for full marks we encourage you to answer openly and honestly
and to avoid simply writing ``what you think the evaluator wants to hear.''

Please answer the following questions.  Some questions can be answered on the
team level, but where appropriate, each team member should write their own
response:


\begin{enumerate}
  \item What went well while writing this deliverable?
  \item What pain points did you experience during this deliverable, and how
    did you resolve them?
  \item Which parts of this document stemmed from speaking to your client(s) or
    a proxy (e.g. your peers)? Which ones were not, and why?
  \item In what ways was the Verification and Validation (VnV) Plan different
    from the activities that were actually conducted for VnV?  If there were
    differences, what changes required the modification in the plan?  Why did
    these changes occur?  Would you be able to anticipate these
    changes in future
    projects?  If there weren't any differences, how was your team
    able to clearly
    predict a feasible amount of effort and the right tasks needed to build the
    evidence that demonstrates the required quality?  (It is expected that most
    teams will have had to deviate from their original VnV Plan.)
\end{enumerate}

\end{document}
