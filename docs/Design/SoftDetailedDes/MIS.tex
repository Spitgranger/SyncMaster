\documentclass[12pt, titlepage]{article}

\usepackage{amsmath, mathtools}

\usepackage[round]{natbib}
\usepackage{amsfonts}
\usepackage{amssymb}
\usepackage{graphicx}
\usepackage{colortbl}
\usepackage{xr}
\usepackage{hyperref}
\usepackage{longtable}
\usepackage{xfrac}
\usepackage{tabularx}
\usepackage{float}
\usepackage{siunitx}
\usepackage{booktabs}
\usepackage{multirow}
\usepackage[section]{placeins}
\usepackage{caption}
\usepackage{fullpage}
\usepackage[normalem]{ulem}

\hypersetup{
  bookmarks=true,     % show bookmarks bar?
  colorlinks=true,       % false: boxed links; true: colored links
  linkcolor=red,          % color of internal links (change box color
  % with linkbordercolor)
  citecolor=blue,      % color of links to bibliography
  filecolor=magenta,  % color of file links
  urlcolor=cyan          % color of external links
}

\usepackage{array}

\externaldocument{../../SRS/SRS}

%% Comments

\usepackage{color}

\newif\ifcomments\commentstrue %displays comments
%\newif\ifcomments\commentsfalse %so that comments do not display

\ifcomments
\newcommand{\authornote}[3]{\textcolor{#1}{[#3 ---#2]}}
\newcommand{\todo}[1]{\textcolor{red}{[TODO: #1]}}
\else
\newcommand{\authornote}[3]{}
\newcommand{\todo}[1]{}
\fi

\newcommand{\wss}[1]{\authornote{blue}{SS}{#1}} 
\newcommand{\plt}[1]{\authornote{magenta}{TPLT}{#1}} %For explanation of the template
\newcommand{\an}[1]{\authornote{cyan}{Author}{#1}}

%% Common Parts

\newcommand{\progname}{ProgName} % PUT YOUR PROGRAM NAME HERE
\newcommand{\authname}{Team \#, Team Name
\\ Student 1 name
\\ Student 2 name
\\ Student 3 name
\\ Student 4 name} % AUTHOR NAMES                  

\usepackage{hyperref}
    \hypersetup{colorlinks=true, linkcolor=blue, citecolor=blue, filecolor=blue,
                urlcolor=blue, unicode=false}
    \urlstyle{same}
                                


\begin{document}

\title{Module Interface Specification for \progname{}}

\author{\authname}

\date{\today}

\maketitle

\pagenumbering{roman}

\section{Revision History}

\begin{tabularx}{\textwidth}{p{3cm}p{2cm}X}
  \toprule {\bf Date} & {\bf Version} & {\bf Notes}\\
  \midrule
  1/17/2025 & 1.0 & Initial Draft of MIS for Rev0\\
  \bottomrule
\end{tabularx}

~\newpage

\section{Symbols, Abbreviations and Acronyms}

See SRS Documentation at
\url{https://github.com/Spitgranger/SyncMaster/blob/main/docs/SRS-Volere/SRS.pdf
}.

\newpage

\tableofcontents

\newpage

\pagenumbering{arabic}

\section{Introduction}

The following document details the Module Interface Specifications for
SyncMaster, a facilities management application for the Technical
Services team at the Water Division of the City of Hamilton.
This application enables the Technical Services team to effectively distribute
water station documentation to stakeholders as a single source of
truth. It also acts as an authentication tool for external
contractors to verify their presence at stations and confirm work
being performed.

Complementary documents include the System Requirement Specifications
and Module Guide.  The full documentation and implementation can be
found at \url{https://github.com/Spitgranger/SyncMaster}.

\section{Notation}

The structure of the MIS for modules comes from \citet{HoffmanAndStrooper1995},
with the addition that template modules have been adapted from
\cite{GhezziEtAl2003}.  The mathematical notation comes from Chapter 3 of
\citet{HoffmanAndStrooper1995}.  For instance, the symbol := is used for a
multiple assignment statement and conditional rules follow the form $(c_1
\Rightarrow r_1 | c_2 \Rightarrow r_2 | ... | c_n \Rightarrow r_n )$.

The following table summarizes the primitive data types used by \progname.

\begin{center}
  \renewcommand{\arraystretch}{1.2}
  \noindent
  \begin{tabular}{l l p{7.5cm}}
    \toprule
    \textbf{Data Type} & \textbf{Notation} & \textbf{Description}\\
    \midrule
    character & char & a single symbol or digit\\
    integer & $\mathbb{Z}$ & a number without a fractional component
    in (-$\infty$, $\infty$) \\
    natural number & $\mathbb{N}$ & a number without a fractional
    component in [1, $\infty$) \\
    real & $\mathbb{R}$ & any number in (-$\infty$, $\infty$)\\
    boolean & boolean & either true or false\\
    decimal & decimal & type that performs floating point arithmetic in exactly
    the same way as mathematical arithmetic\\
    any & any & any data type\\
    \bottomrule
  \end{tabular}
\end{center}

\noindent
The specification of \progname \ uses some derived data types:
sequences, strings, tuples, map, enum, KeyCondition,
AttributeCondition, S3File, HTTPRequest, HTTPResponse, and Optional. Sequences
are lists filled with elements of the same data type. Strings
are sequences of characters. Tuples contain a list of values, potentially of
different types. In addition, \progname \ uses functions, which
are defined by the data types of their inputs and outputs. Local functions are
described by giving their type signature followed by their specification. Maps
are a collection of key-value pairs, where there does not necessarily need to
be a restriction on the data types of the keys and values, but one can be
placed. An enum is a set of values of which the data can be.
AttributeConditions are conditionals placed on an attribute of a
database entry. KeyConditions are a subset of AttributeConditions,
and are condtionals placed on the key attributes of a database entry.
An S3File, is a unique file that exists in AWS S3. HTTPRequest and HTTPResponse
are a subset of maps which conform to the HTTP standard. An Optional is not a
datatype by itself, but specifies that there may be the absence of a value of a
specific type. LogEntry is an object consisting of a UserID, SiteID, an ISO
formatted date and time string, and a type of log (entry or exit). A
Document is an object
consisting of the following fields [ siteId: string, expiryDate: string, name:
  string, createdDatetime: string, lastEditedDateTime: string, s3Link: string,
  parentDocumentId: map[string $\rightarrow$ any], requireAck: boolean, userId:
string ]

\section{Module Decomposition}

The following table is taken directly from the Module Guide document
for this project.

\begin{table}[h!]
  \centering
  \begin{tabular}{p{0.3\textwidth} p{0.6\textwidth}}
    \toprule
    \textbf{Level 1}                                       &
    \textbf{Level 2}
    \\
    \midrule

    {Hardware-Hiding Module}                               & N/A
    \\
    \midrule

    \multirow{7}{0.3\textwidth}{Software Decision Modules} &
    \sout{Audit and Compliance Module}\\ &
    \textcolor{red}{Site Management Module}
    \\
    & User Authentication Module      \\
    & Location Verification Module    \\
    & Logging Module \textcolor{red}{(Site Visits)}                  \\
    & Analytics and Reporting Module  \\
    & User Management Module          \\
    & Document Management Module      \\
    &\sout{Job Management Module}           \\
    \midrule

    \multirow{3}{0.3\textwidth}{Behaviour-Hiding Modules}  & {API
    Integration Module}
    \\
    & Database Interaction Module     \\
    & Blob Storage Interaction Module \\
    & Request Routing Module          \\
    & Function Compute Module         \\
    \bottomrule
  \end{tabular}
  \caption{Module Hierarchy}
  \label{TblMH}
\end{table}
\section{MIS of Site Management Module}
\label{sec:SM}

\subsection{Module}
Site Management Module

\subsection{Uses}
\hyperref[sec:DI]{Database Interaction Module}

\subsection{Syntax}

\subsubsection{Exported Constants}
N/A
\subsubsection{Exported Access Programs}
\begin{center}
  \resizebox{\textwidth}{!}{
    \begin{tabular}{|p{4.5cm}|p{4cm}|p{4cm}|p{5cm}|}
      \hline
      \textbf{Name} & \textbf{In} & \textbf{Out} & \textbf{Exceptions} \\
      \hline
      createSite & \textbf{siteId:} string \newline
      \textbf{siteLatitude:} decimal \newline
      \textbf{siteLongitude:} decimal \newline
      \textbf{acceptableRange:} int &
      map[string $\rightarrow$ any] &
      \textbf{ExternalServiceFailure:} An internal error from AWS \newline
      \textbf{ResourceConflict:} A site already exists with the same
      siteId \newline
      \textbf{PermissionException:} There does not exist write
      permissions for the database \\
      \hline
      updateSite & \textbf{siteId:} string \newline
      \textbf{siteLatitude:} decimal \newline
      \textbf{siteLongitude:} decimal \newline
      \textbf{acceptableRange:} int &
      map[string $\rightarrow$ any] &
      \textbf{ExternalServiceFailure:} An internal error from AWS \newline
      \textbf{TimeConsistency:} An update has been made to the site
      since the update was requested. \newline
      \textbf{PermissionException:} There does not exist write
      permissions for the database \\
      \hline
      deleteSite & \textbf{siteId:} string & None &
      \textbf{ExternalServiceFailure:} An internal error from AWS \newline
      \textbf{BadRequest:} Documents belonging to this site exist \newline
      \textbf{TimeConsistency:} An update has been made to the site
      since the delete was requested. \newline
      \textbf{PermissionException:} There does not exist write
      permissions for the database \\
      \hline
      getSite & \textbf{siteId:} string & map[string $\rightarrow$
      any] & \textbf{ExternalServiceFailure:} An internal error from
      AWS \newline \textbf{ResourceNotFound:} No site exists for the given id \\
      \hline
      listSites & - & sequence[map[string $\rightarrow$ any]] &
      \textbf{ExternalServiceFailure:} An internal error from AWS \\
      \hline
    \end{tabular}
  }
\end{center}
\subsection{Semantics}

\subsubsection{State Variables}

N/A
\subsubsection{Environment Variables}
\begin{center}
  \begin{tabular}{p{6cm} p{10cm}}
    \hline
    \textbf{Name} & \textbf{Description} \\
    \hline
    LAMBDA\_EXECUTION\_ROLE & When an AWS Lambda Function (The chosen
    AWS compute service for this project), it has an AWS IAM role
    attached to it, that it uses when running. This role needs
    permission for database access for this module to work \\
    \hline
  \end{tabular}
\end{center}

\subsubsection{Assumptions}

LAMBDA\_EXECUTION\_ROLE has the required permissions in AWS to access
the database.

\subsubsection{Access Routine Semantics}
\noindent createSite(siteId: string, siteLatitude: decimal, siteLongitude:
decimal, acceptableRange: int):
\begin{itemize}
  \item output: map[string $\rightarrow$ any]
  \item exception: ExternalServiceFailure, ResourceConflict, PermissionException
\end{itemize}

\noindent updateSite(siteId: string, siteLatitude, siteLongitude,
acceptableRange):
\begin{itemize}
  \item output: map[string $\rightarrow$ any]
  \item exception: ExternalServiceFailure, TimeConsistency, PermissionException
\end{itemize}

\noindent deleteSite(siteId: string):
\begin{itemize}
  \item output: N/A
  \item exception: ExternalServiceFailure, BadRequest, TimeConsistency,
    PermissionException
\end{itemize}

\noindent getSite(siteId: string):
\begin{itemize}
  \item output: map[string $\rightarrow$ any]
  \item exception: ExternalServiceFailure
\end{itemize}

\noindent listSites(FromDatetime: string, ToDatetime: string):
\begin{itemize}
  \item output: sequence[map[string $\rightarrow$ any]]
  \item exception: ExternalServiceFailure
\end{itemize}

\subsubsection{Local Functions}

N/A

\section{MIS of Database Interaction Module}
\label{sec:DI}

\subsection{Module}

Database Interaction Module

\subsection{Uses}

boto3 (AWS SDK for Python), AWS DynamoDB (AWS Cloud Service for NoSQL Databases)

\subsection{Syntax}

\subsubsection{Exported Constants}

\begin{center}
  \begin{tabular}{p{4cm} p{12cm}}
    \hline
    \textbf{Name} & \textbf{Description} \\
    \hline
    DBTable.Name & The name of the underlying DynamoDB table resource. \\
    \hline
    DBTable.Access & The level of access the DBTable object has on
    the DynamoDB table. Either ``read'' or ``write''. \\
    \hline
  \end{tabular}
\end{center}

\subsubsection{Exported Access Programs}

\begin{center}
  \begin{tabular}{>{\raggedright}p{3cm} >{\raggedright}p{5cm}
    >{\raggedright}p{4cm} p{4cm}}
    \hline
    \textbf{Name} & \textbf{In} & \textbf{Out} & \textbf{Exceptions} \\
    \hline
    DBTable & \textbf{TableName:} string \newline \textbf{Access:}
    enum[``read'', ``write''] & DBTable & - \\
    \hline
    DBTable.get & \textbf{Key:} map[string $\rightarrow$ any] &
    map[string $\rightarrow$ any] & \textbf{ItemNotFound:} Item with
    requested key does not exist in the database \newline
    \textbf{ExternalServiceFailure:} An internal error from AWS \\
    \hline
    DBTable.put & \textbf{Item:} map[string $\rightarrow$ any]
    \newline \textbf{Condition:} AttributeCondition & map[string
    $\rightarrow$ any] &
    \textbf{ConditionCheckFailed:} The given condition is not met
    \newline \textbf{ExternalServiceFailure:} An internal error from
    AWS \newline \textbf{PermissionException:} If the current access
    level is read-only \\
    \hline
    DBTable.delete & \textbf{Key:} map[string $\rightarrow$ any]
    \newline \textbf{Condition:} AttributeCondition & map[string
    $\rightarrow$ any] & \textbf{ExternalServiceFailure:} An internal
    error from AWS \newline \textbf{ConditionCheckFailed:} The given
    condition is not met \newline \textbf{PermissionException:} If
    the current access level is read-only \\
    \hline
    DBTable.query & \textbf{KeyConditions:} sequence[KeyCondition]
    \newline \textbf{AttributeConditions:}
    sequence[AttributeCondition] & sequence[map[string $\rightarrow$
    any]] & \textbf{ExternalServiceFailure:} An internal error from AWS \\
    \hline
  \end{tabular}
\end{center}

\subsection{Semantics}

\subsubsection{State Variables}

\begin{center}
  \begin{tabular}{p{4cm} p{12cm}}
    \hline
    \textbf{Name} & \textbf{Description} \\
    \hline
    Database & The underlying AWS DynamoDB table, can be represented
    as a set of items: $\{i_0, i_1, ..., i_n\}$, where $i_k:
    map[string \rightarrow  any], k\in[0,n]$ \\
    \hline
  \end{tabular}
\end{center}

\subsubsection{Environment Variables}

\begin{center}
  \begin{tabular}{p{6cm} p{10cm}}
    \hline
    \textbf{Name} & \textbf{Description} \\
    \hline
    LAMBDA\_EXECUTION\_ROLE & When an AWS Lambda Function (The chosen
    AWS compute service for this project), it has an AWS IAM role
    attached to it, that it uses when running. This role needs
    permission for database access for this module to work \\
    \hline
  \end{tabular}
\end{center}

\subsubsection{Assumptions}

LAMBDA\_EXECUTION\_ROLE has the required permissions in AWS to access
the database.

\subsubsection{Access Routine Semantics}

\noindent DBTable.get(Key: k):
\begin{itemize}
  \item output: $i_k$, where $i_k \in Database \land dbKey(i_k) == k$
  \item exception: ExternalServiceFailure
\end{itemize}

\noindent DBTable.put(Item: $i_{new}$, Condition: c):
\begin{itemize}
  \item transition: $Database \rightarrow Database \cup \{i_{new}\}$, if
    $c == true \land DBTable.Access == ``write''$
  \item output: $i_{new}$
  \item exception: ExternalServiceFailure, ConditionCheckFailed,
    PermissionException
\end{itemize}

\noindent DBTable.delete(Key: k, Condition: c):
\begin{itemize}
  \item transition: $Database \rightarrow Database - \{i_{old}\}$, where
    $k==dbKey(i_{old}, DBTable.Name)$, if $c == true \land
    DBTable.Access == ``write''$
  \item output: $i_{old}$
  \item exception: ExternalServiceFailure, ConditionCheckFailed,
    PermissionException
\end{itemize}

\noindent DBTable.query(KeyConditions: kc, AttributeConditions: ac):
\begin{itemize}
  \item output: [$i_k | i_k \in Database \land \forall c \in kc (c ==
    true) \land \forall c \in ac (c == true)$]
  \item exception: ExternalServiceFailure
\end{itemize}

\subsubsection{Local Functions}

\begin{center}
  \begin{tabular}{>{\raggedright}p{2cm} >{\raggedright}p{5cm}
    >{\raggedright}p{3.5cm} p{4.5cm}}
    \hline
    \textbf{Name} & \textbf{In} & \textbf{Out} & \textbf{Description} \\
    \hline
    dbKey & \textbf{Item:} map[string $\rightarrow$ any] \newline
    \textbf{TableName:} string & map[string $\rightarrow$ any] &
    Returns the keys of the given db item, assuming it is from the
    given table \\
    \hline
  \end{tabular}
\end{center}

\section{MIS of Logging Module \textcolor{red}{(Site Visits)}}
\label{sec:LM}

\subsection{Module}

Logging Module

\subsection{Uses}

\hyperref[sec:DI]{Database Interaction Module}

\subsection{Syntax}

\subsubsection{Exported Constants}

N/A

\subsubsection{Exported Access Programs}

\begin{center}
  \begin{tabular}{>{\raggedright}p{3cm} >{\raggedright}p{5cm}
    >{\raggedright}p{4cm} p{4cm}}
    \hline
    \textbf{Name} & \textbf{In} & \textbf{Out} & \textbf{Exceptions} \\
    \hline
    AddLog & \textbf{UserID:} string \newline \textbf{SiteID:} string
    \newline \textbf{Datetime:} string \newline \textbf{Type:}
    enum[``entry'', ``exit''] & LogEntry &
    \textbf{ExternalServiceFailure:} An internal error from AWS \\
    \hline
    ListLogs & \textbf{UserID:} string \newline \textbf{SiteID:}
    string \newline \textbf{FromDatetime:} string \newline
    \textbf{ToDatetime:} string &
    sequence[LogEntry] & \textbf{ExternalServiceFailure:} An internal
    error from AWS \\
    \hline
    PurgeLogs & \textbf{UserID:} string & - &
    \textbf{ExternalServiceFailure:} An internal error from AWS \\
    \hline
  \end{tabular}
\end{center}

\subsection{Semantics}

\subsubsection{State Variables}

\begin{center}
  \begin{tabular}{p{4cm} p{12cm}}
    \hline
    \textbf{Name} & \textbf{Description} \\
    \hline
    LogEntryDatabase & The underlying AWS DynamoDB table, can be
    represented as a set of items: $\{l_0, l_1, ..., l_n\}$, where
    $l_k: LogEntry, k\in[0,n]$ \\
    \hline
  \end{tabular}
\end{center}

\subsubsection{Environment Variables}

\begin{center}
  \begin{tabular}{p{6cm} p{10cm}}
    \hline
    \textbf{Name} & \textbf{Description} \\
    \hline
    LAMBDA\_EXECUTION\_ROLE & When an AWS Lambda Function (The chosen
    AWS compute service for this project), it has an AWS IAM role
    attached to it, that it uses when running. This role needs
    permission for logging database access for this module to work \\
    \hline
  \end{tabular}
\end{center}

\subsubsection{Assumptions}

LAMBDA\_EXECUTION\_ROLE has the required permissions in AWS to access
the database.

\subsubsection{Access Routine Semantics}

\noindent AddLog(UserID: u, SiteID: s, Datetime: d, Type: t):
\begin{itemize}
  \item transition: $LogEntryDatabase \rightarrow LogEntryDatabase
    \cup \{l_{new}\}$, where $l_{new}.UserID = u$, $l_{new}.SiteID =
    s$, $l_{new}.Datetime = d$, $l_{new}.Type = t$
  \item output: $l_{new}$
  \item exception: ExternalServiceFailure
\end{itemize}

\noindent ListLogs(UserID: u, SiteID: s, FromDatetime: fd, ToDatetime: td):
\begin{itemize}
  \item output: $[l_k | l_k \in LogEntryDatabase \land l_k.UserID ==
    u \land l_k.SiteID == s \land fd \le l_k.Datetime \le td]$
  \item exception: ExternalServiceFailure
\end{itemize}

\noindent PurgeLogs(UserID: u):
\begin{itemize}
  \item transition: $LogEntryDatabase \rightarrow LogEntryDatabase -
    \{l_k | l_k \in LogEntryDatabase \land l_k.UserID == u \}$
  \item exception: ExternalServiceFailure
\end{itemize}

\subsubsection{Local Functions}

N/A

\section{MIS of Analytics and Reporting Module}
\label{sec:AR}

\subsection{Module}
Analytics and Reporting Module

\subsection{Uses}
\hyperref[sec:DI]{Database Interaction Module

  \subsection{Syntax}

  \subsubsection{Exported Constants}
  N/A
  \subsubsection{Exported Access Programs}
  \begin{center}
    \begin{tabular}{p{4.5cm} p{4cm} p{4cm} p{4cm}}
      \hline
      \textbf{Name} & \textbf{In} & \textbf{Out} & \textbf{Exceptions} \\
      \hline

      GetErrorsReport & \textbf{FromDatetime:} string \newline
      \textbf{ToDatetime:} string&
      sequence[map[string $\rightarrow$ any]] &
      \textbf{ExternalServiceFailure:} An internal error from AWS
      \\
      \hline
      GetResourceUsageReport & \textbf{FromDatetime:} string \newline
      \textbf{ToDatetime:} string \newline
      \textbf{Resources:} \newline sequence[resource]&
      sequence[map[string $\rightarrow$ any]] &
      \textbf{ExternalServiceFailure:} An internal error from AWS
      \\
      \hline
      GetUserActivityReport & \textbf{FromDatetime:} string \newline
      \textbf{ToDatetime:} string \newline
      \textbf{UserID:} string&
      sequence[map[string $\rightarrow$ any]] &
      \textbf{ExternalServiceFailure:} An internal error from AWS
      \\
      \hline
      GetSystemHealthReport & \textbf{FromDatetime:} string \newline
      \textbf{ToDatetime:} string &
      sequence[map[string $\rightarrow$ any]] &
      \textbf{ExternalServiceFailure:} An internal error from AWS
      \\
      \hline
      GetLoginAttemptsReport & \textbf{FromDatetime:} string \newline
      \textbf{ToDatetime:} string &
      sequence[map[string $\rightarrow$ any]] &
      \textbf{ExternalServiceFailure:} An internal error from AWS
      \\
      \hline
    \end{tabular}
  \end{center}

  \subsection{Semantics}

  \subsubsection{State Variables}

  N/A
  \subsubsection{Environment Variables}
  \begin{center}
    \begin{tabular}{p{6cm} p{10cm}}
      \hline
      \textbf{Name} & \textbf{Description} \\
      \hline
      LAMBDA\_EXECUTION\_ROLE & When an AWS Lambda Function (The chosen
      AWS compute service for this project), it has an AWS IAM role
      attached to it, that it uses when running. This role needs
      permission for database access for this module to work \\
      \hline
    \end{tabular}
  \end{center}

  \subsubsection{Assumptions}

  LAMBDA\_EXECUTION\_ROLE has the required permissions in AWS to access
  the database.

  \subsubsection{Access Routine Semantics}
  \noindent GetErrorsReport(FromDatetime: string, ToDatetime: string):
  \begin{itemize}
    \item output: sequence[map[string $\rightarrow$ any]]
    \item exception: ExternalServiceFailure
  \end{itemize}

  \noindent GetResourceUsageReport(FromDatetime: string, ToDatetime: string):
  \begin{itemize}
    \item output: sequence[map[string $\rightarrow$ any]]
    \item exception: ExternalServiceFailure
  \end{itemize}

  \noindent GetUserActivityReport(FromDatetime: string, ToDatetime: string):
  \begin{itemize}
    \item output: sequence[map[string $\rightarrow$ any]]
    \item exception: ExternalServiceFailure
  \end{itemize}

  \noindent GetSystemHealthReport(FromDatetime: string, ToDatetime: string):
  \begin{itemize}
    \item output: sequence[map[string $\rightarrow$ any]]
    \item exception: ExternalServiceFailure
  \end{itemize}

  \noindent GetLoginAttemptsReport(FromDatetime: string, ToDatetime: string):
  \begin{itemize}
    \item output: sequence[map[string $\rightarrow$ any]]
    \item exception: ExternalServiceFailure
  \end{itemize}

  \subsubsection{Local Functions}

  N/A

  \section{MIS of File Storage Interaction Module}
  \label{sec:FS}

  \subsection{Module}

  File Storage Interaction Module

  \subsection{Uses}

  boto3 (AWS SDK for Python), AWS S3 (AWS Cloud Service for storing files)

  \subsection{Syntax}

  \subsubsection{Exported Constants}

  \begin{center}
    \begin{tabular}{p{4cm} p{12cm}}
      \hline
      \textbf{Name} & \textbf{Description} \\
      \hline
      S3Bucket.Name & The name of the underlying S3 Bucket resource. \\
      \hline
      S3Bucket.Access & The level of access the S3Bucket object has on
      the S3 Bucket resource in AWS. Either ``read'' or ``write''. \\
      \hline
    \end{tabular}
  \end{center}

  \subsubsection{Exported Access Programs}

  \begin{center}
    \begin{tabular}{>{\raggedright}p{5cm} >{\raggedright}p{4cm}
      >{\raggedright}p{2cm} p{4cm}}
      \hline
      \textbf{Name} & \textbf{In} & \textbf{Out} & \textbf{Exceptions} \\
      \hline
      S3Bucket & \textbf{BucketName:} string \newline \textbf{Access:}
      enum[``read'', ``write''] & S3Bucket & - \\
      \hline
      S3Bucket.createPresignedUrl & \textbf{Key:} string \newline
      \textbf{VersionID:} string \newline \textbf{ETag:} string
      \newline \textbf{Method:} enum[``get'', ``upload''] \newline
      \textbf{ExpiresIn:} integer & string &
      \textbf{PermissionException:} If attempting to get an upload url,
      while only having read permissions \newline
      \textbf{ExternalServiceFailure:} An internal error from AWS \\
      \hline
      S3Bucket.delete & \textbf{Key:} string \newline
      \textbf{VersionID:} string \newline \textbf{ETag:} string & - &
      \textbf{FileNotFound:} The given key and version do not match any
      file in the bucket \newline \textbf{ETagMismatch:} The given ETag
      does not match the ETag of the file with the given key and
      version \newline \textbf{ExternalServiceFailure:} An internal
      error from AWS \newline \textbf{PermissionException:} If the
      current access level is read-only \\
      \hline
    \end{tabular}
  \end{center}

  \subsection{Semantics}

  \subsubsection{State Variables}

  \begin{center}
    \begin{tabular}{p{4cm} p{12cm}}
      \hline
      \textbf{Name} & \textbf{Description} \\
      \hline
      Bucket & The underlying AWS S3 Bucket, can be represented
      as a set of files: $\{f_0, f_1, ..., f_n\}$, where $f_k: S3File
      \land k\in[0,n]$ \\
      \hline
    \end{tabular}
  \end{center}

  \subsubsection{Environment Variables}

  \begin{center}
    \begin{tabular}{p{6cm} p{10cm}}
      \hline
      \textbf{Name} & \textbf{Description} \\
      \hline
      LAMBDA\_EXECUTION\_ROLE & When an AWS Lambda Function (The chosen
      AWS compute service for this project), it has an AWS IAM role
      attached to it, that it uses when running. This role needs
      permission for S3 bucket access for this module to work \\
      \hline
    \end{tabular}
  \end{center}

  \subsubsection{Assumptions}

  LAMBDA\_EXECUTION\_ROLE has the required permissions in AWS to access
  the S3 bucket.

  \subsubsection{Access Routine Semantics}

  \noindent S3Bucket.createPresignedUrl(Key: k, VersionID: v, ETag: e,
  Method: m, ExpiresIn: x):
  \begin{itemize}
    \item output: string
    \item exception: ExternalServiceFailure, PermissionException
  \end{itemize}

  \noindent S3Bucket.delete(Key: k, VersionID: v, ETag: e):
  \begin{itemize}
    \item transition: $Bucket \rightarrow Bucket - \{f_{old}\}$, where
      $map\{Key: k, VersionID: v, ETag:
      e\}==S3Bucket.metadata(f_{old})$, if $S3Bucket.Access == ``write''$
    \item exception: ExternalServiceFailure, ConditionCheckFailed,
      PermissionException
  \end{itemize}

  \subsubsection{Local Functions}

  \begin{center}
    \begin{tabular}{>{\raggedright}p{4cm} >{\raggedright}p{3cm}
      >{\raggedright}p{3.5cm} p{5.5cm}}
      \hline
      \textbf{Name} & \textbf{In} & \textbf{Out} & \textbf{Description} \\
      \hline
      S3Bucket.metadata & \textbf{File:} S3File & map[string
      $\rightarrow$ any] & Returns S3 key, versionID, and Etag of the
      given file, assuming it is from the given S3Bucket \\
      \hline
    \end{tabular}
  \end{center}

  \section{MIS of Function Compute Module}
  \label{sec:FC}

  \subsection{Module}

  Function Compute Module

  \subsection{Uses}

  AWS Lambda (AWS Service that executes code in response to events and
  manages the compute resources needed to run the code)

  All other modules run on function compute. Therefore, this module
  uses all the others, except for the Routing Module, and API
  Integration Module.

  \subsection{Syntax}

  \subsubsection{Exported Constants}

  N/A

  \subsubsection{Exported Access Programs}

  \begin{center}
    \begin{tabular}{>{\raggedright}p{3cm} >{\raggedright}p{5cm}
      >{\raggedright}p{4cm} p{4cm}}
      \hline
      \textbf{Name} & \textbf{In} & \textbf{Out} & \textbf{Exceptions} \\
      \hline
      Invoke & \textbf{FunctionName:} string \newline \textbf{Event:}
      map[any $\rightarrow$ any] & map[any $\rightarrow$ any] &
      \textbf{ExternalServiceFailure:} An internal error from AWS
      \newline \textbf{ExecutionError:} Any error that occurs while the
      function is running \\
      \hline
    \end{tabular}
  \end{center}

  \subsection{Semantics}

  \subsubsection{State Variables}

  N/A

  \subsubsection{Environment Variables}

  \begin{center}
    \begin{tabular}{p{6cm} p{10cm}}
      \hline
      \textbf{Name} & \textbf{Description} \\
      \hline
      LAMBDA\_EXECUTION\_ROLE & When an AWS Lambda Function runs, it
      has an AWS IAM role attached to it, which it uses when running.
      This role gives the function the necessary permissions to execute
      without issue. \\
      \hline
    \end{tabular}
  \end{center}

  \subsubsection{Assumptions}

  LAMBDA\_EXECUTION\_ROLE has the required permissions in AWS to
  execute the lambda's required tasks.

  \subsubsection{Access Routine Semantics}

  N/A

  \subsubsection{Local Functions}

  N/A

  \section{MIS of Routing Module}
  \label{sec:RM}

  \subsection{Module}

  Routing Module

  \subsection{Uses}

  AWS APIGateway (AWS Cloud Service for handling routing of API to an
  underlying serverless function), \hyperref[sec:UA]{User Authentication Module}

  \subsection{Syntax}

  \subsubsection{Exported Constants}

  \begin{center}
    \begin{tabular}{p{4cm} p{12cm}}
      \hline
      \textbf{Name} & \textbf{Description} \\
      \hline
      BaseUrl & The base url of the REST API \\
      \hline
    \end{tabular}
  \end{center}

  \subsubsection{Exported Access Programs}

  \begin{center}
    \begin{tabular}{>{\raggedright}p{3cm} >{\raggedright}p{5cm}
      >{\raggedright}p{4cm} p{4cm}}
      \hline
      \textbf{Name} & \textbf{In} & \textbf{Out} & \textbf{Exceptions} \\
      \hline
      SubmitRequest & \textbf{Request:} map[any $\rightarrow$ any]
      \newline \textbf{Path:} String & map[any $\rightarrow$ any] &
      \textbf{ExternalServiceFailure:} An internal error from
      AWS \newline \textbf{ExecutionError:} If the underlying compute
      resource that the request gets routed to encounters an error
      during execution \\
      \hline
    \end{tabular}
  \end{center}

  \subsection{Semantics}

  \subsubsection{State Variables}

  N/A

  \subsubsection{Environment Variables}

  N/A

  \subsubsection{Assumptions}

  N/A

  \subsubsection{Access Routine Semantics}

  N/A

  \subsubsection{Local Functions}

  N/A

  \section{MIS of Location Verification Module}
  \label{sec:LV}
  \subsection{Module}

  Location Verification Module

  \subsection{Uses}

  Browser location/GPS API, \hyperref[sec:DI]{Database Interaction Module}

  \subsection{Syntax}

  \subsubsection{Exported Constants}

  N/A

  \subsubsection{Exported Access Programs}

  \begin{center}
    \begin{tabular}{>{\raggedright}p{2.75cm} >{\raggedright}p{3.25cm}
      >{\raggedright}p{4.5cm} p{5cm}}
      \hline
      \textbf{Name} & \textbf{In} & \textbf{Out} & \textbf{Exceptions} \\
      \hline
      verifyLocation & \textbf{siteID:} string \newline
      \textbf{latitude:} float \newline
      \textbf{longitude:} float \newline \textbf{accuracy:} float &
      \textbf{locationState:} boolean &
      \textbf{Invalidlocation:} co-ordinates are invalid \\
      \hline
    \end{tabular}
  \end{center}

  \subsection{Semantics}

  \subsubsection{State Variables}

  N/A

  \subsubsection{Environment Variables}

  \begin{center}
    \begin{tabular}{p{6cm} p{10cm}}
      \hline
      \textbf{Name} & \textbf{Description} \\
      \hline
      LAMBDA\_EXECUTION\_ROLE & When an AWS Lambda Function (The chosen
      AWS compute service for this project), it has an AWS IAM role
      attached to it, that it uses when running. This role needs
      permission for site database access for this module to work \\
      \hline
    \end{tabular}
  \end{center}

  \subsubsection{Assumptions}

  LAMBDA\_EXECUTION\_ROLE has the required permissions in AWS to
  execute the lambda's required tasks.

  \subsubsection{Access Routine Semantics}

  \noindent verifyLocation(siteID: siteID, latitude: lat, longitude:
  long, accuracy: acc):
  \begin{itemize}
    \item output: true if the distance calculated by calculateDistance
      is within the range of the intended site, adjusted for accuracy;
      false otherwise.
    \item exception: Invalidlocation
  \end{itemize}

  \subsubsection{Local Functions}

  \begin{center}
    \begin{tabular}{>{\raggedright}p{3cm} >{\raggedright}p{4cm}
      >{\raggedright}p{4cm} p{4.5cm}}
      \hline
      \textbf{Name} & \textbf{In} & \textbf{Out} & \textbf{Description} \\
      \hline
      calculateDistance & \textbf{siteID:} string \newline
      \textbf{latitude:} float \newline
      \textbf{longitude:} float & \textbf{distance:} float &
      Uses siteID to get a second set of coordinates from Database.
      Using two sets of coordinates the haversine distance between the
      two points is returned. \\
      \hline
    \end{tabular}
  \end{center}

  \section{MIS of User Management Module}
  \label{sec:UM}

  \subsection{Module}

  User Management Module

  \subsection{Uses}

  boto3 (AWS SDK for Python), AWS Cognito (AWS Cloud Service for user
  authentication), \hyperref[sec:DI]{Database Interaction Module}

  \subsection{Syntax}

  \subsubsection{Exported Constants}

  N/A

  \subsubsection{Exported Access Programs}

  \begin{center}
    \begin{tabular}{>{\raggedright}p{2.5cm} >{\raggedright}p{3.5cm}
      >{\raggedright}p{3.5cm} p{5cm}}
      \hline
      \textbf{Name} & \textbf{In} & \textbf{Out} & \textbf{Exceptions} \\
      \hline
      createUser & \textbf{email:} string \newline
      \textbf{details:} map[string $\rightarrow$ any] &
      \textbf{userID:} string &
      \textbf{DuplicateUser:} Existing email address used \\
      \hline
      editUser & \textbf{userID:} string \newline
      \textbf{edit:} map[string $\rightarrow$ any] & boolean &
      \textbf{UserNotFound:}
      userID not in database \\
      \hline
      deleteUser & \textbf{userID:} string & boolean & \textbf{UserNotFound:}
      userID not in database \\
      \hline
      getUser & \textbf{userID:} string & map[string $\rightarrow$ any]
      & \textbf{UserNotFound:}
      userID not in database \\
      \hline
      createRequest & \textbf{email:} string \newline \textbf{details:}
      map[string $\rightarrow$ any] & None
      & \textbf{DuplicateUser:}
      user or user request already exists with same email \\
      \hline
      actionRequest & \textbf{email:} string & \textbf{userID:} string
      & \textbf{UserRequestNotFound}
      user request not found in database\\
      \hline
      listRequests & - & sequence[map[string $\rightarrow$ any]] &
      \textbf{ExternalServiceFailure:} An internal error from AWS\\
      \hline
    \end{tabular}
  \end{center}

  \subsection{Semantics}

  \subsubsection{State Variables}
  \begin{center}
    \begin{tabular}{p{4cm} p{12cm}}
      \hline
      \textbf{Name} & \textbf{Description} \\
      \hline
      Database & Set of registered users and requests, can be
      represented as set of
      items $\{i_0, i_1, ..., i_n\}$, where $i_k:
      map[string \rightarrow  any], k\in[0,n]$ \\
      \hline
    \end{tabular}
  \end{center}

  \subsubsection{Environment Variables}

  \begin{center}
    \begin{tabular}{p{6cm} p{10cm}}
      \hline
      \textbf{Name} & \textbf{Description} \\
      \hline
      LAMBDA\_EXECUTION\_ROLE & When an AWS Lambda Function runs, it
      has an AWS IAM role attached to it, which it uses when running.
      This role gives the function the necessary permissions to execute
      without issue. \\
      \hline
    \end{tabular}
  \end{center}

  \subsubsection{Assumptions}

  LAMBDA\_EXECUTION\_ROLE has the required permissions in AWS to
  execute the lambda's required tasks.

  \subsubsection{Access Routine Semantics}

  \noindent createUser(email: email, details: details):
  \begin{itemize}
    \item transition: $Database \rightarrow Database \cup
      \{User_{new}\}$, if $email \notin Database$, where $User_{new} =
      \{userID, email, password, \ldots\}$.
    \item output: $\{userID, password\}$, where $userID$ is uniquely
      generated for the email and $password$ is a temporary password
      generated during account creation
    \item exception: DuplicateUser
  \end{itemize}

  \noindent editUser(userID: userID, edit: changes):
  \begin{itemize}
    \item transition: $Database \rightarrow Database \cup
      \{User_{edited}\}$, if $(userID \in Database) \wedge
      (changes.email \notin Database)$, where $User_{edited} =
      \{userID, changes\}$
    \item output: true if transition successful, false otherwise.
    \item exception: UserNotFound
  \end{itemize}

  \noindent deleteUser(userID: userID):
  \begin{itemize}
    \item transition: $Database \rightarrow Database - \{User\}$, if
      $\exists(\{User\} \in Database \wedge User.userID == userID)$.
    \item output: true if transition successful, false otherwise.
    \item exception: UserNotFound
  \end{itemize}

  \noindent getUser(userID: userID):
  \begin{itemize}
    \item output: $\{User\}$, if $\exists(\{User\} \in Database \wedge
      User.userID == userID)$.
    \item exception: UserNotFound
  \end{itemize}

  \noindent createRequest(email: string, details: map[string
  $\rightarrow$ any]):
  \begin{itemize}
    \item transition: $Database \rightarrow Database \cup
      \{UserRequest_{new}\}$, if $email \notin Database$, where
      $UserRequest_{new} =
      \{email, company, \ldots\}$.
    \item exception: DuplicateUser
  \end{itemize}

  \noindent actionRequest(email: string):
  \begin{itemize}
    \item transition: $Database \rightarrow Database - \{UserRequest\}$, if
      $\exists(\{UserRequest\} \in Database \wedge UserRequest.email == email)$.
    \item output: $\{userID, password\}$, where $userID$ is uniquely
      generated for the email and $password$ is a temporary password
      generated during account creation
    \item exception: UserRequestNotFound
  \end{itemize}

  \noindent listRequests():
  \begin{itemize}
    \item output: sequence[map[string $\rightarrow$ any]]
    \item exception: ExternalServiceException
  \end{itemize}

  \subsubsection{Local Functions}

  \begin{center}
    \begin{tabular}{>{\raggedright}p{3cm} >{\raggedright}p{4cm}
      >{\raggedright}p{4cm} p{4.5cm}}
      \hline
      \textbf{Name} & \textbf{In} & \textbf{Out} & \textbf{Description} \\
      \hline
      generateUserID & \textbf{email:} string & \textbf{authToken:} string &
      Generates unique userID for each email. \\
      \hline
      generatePassword & \textbf{password:} string &
      \textbf{authToken:} string &
      Creates one-time password for new users.\\
      \hline
    \end{tabular}
  \end{center}

  \section{MIS of User Authentication Module} \label{Module}
  \label{sec:UA}
  \subsection{Module}

  User Authentication Module

  \subsection{Uses}

  boto3 (AWS SDK for Python), AWS Cognito (AWS Cloud Service for user
  authentication), \hyperref[sec:LV]{Location Verification Module},
  \hyperref[sec:DI]{Database Interaction Module}

  \subsection{Syntax}

  \subsubsection{Exported Constants}

  N/A

  \subsubsection{Exported Access Programs}

  \begin{center}
    \begin{tabular}{>{\raggedright}p{4cm} >{\raggedright}p{3cm}
      >{\raggedright}p{4cm} p{4cm}}
      \hline
      \textbf{Name} & \textbf{In} & \textbf{Out} & \textbf{Exceptions} \\
      \hline
      authenticateUser & \textbf{email:} string \newline
      \textbf{password:} string & \textbf{authToken:} string &
      \textbf{InvalidCredentials:} Credentials not in database \\
      \hline
      authenticateContractor & \textbf{email:} string \newline
      \textbf{name:} string \newline \textbf{siteID:} String \newline
      \textbf{userLocation:} \{float, float\} &
      \textbf{authToken:} string & \textbf{InvalidCredentials:}
      Credentials not in database \newline
      \textbf{InvalidSiteID:}
      siteID not in database \newline
      \textbf{LocationVerificationFailed:} Verification of users
      location failed \\
      \hline
    \end{tabular}
  \end{center}

  \subsection{Semantics}

  \subsubsection{State Variables}

  N/A

  \subsubsection{Environment Variables}

  \begin{center}
    \begin{tabular}{p{6cm} p{10cm}}
      \hline
      \textbf{Name} & \textbf{Description} \\
      \hline
      LAMBDA\_EXECUTION\_ROLE & When an AWS Lambda Function runs, it
      has an AWS IAM role attached to it, which it uses when running.
      This role gives the function the necessary permissions to execute
      without issue. \\
      \hline
    \end{tabular}
  \end{center}

  \subsubsection{Assumptions}

  LAMBDA\_EXECUTION\_ROLE has the required permissions in AWS to
  execute the lambda's required tasks.

  \subsubsection{Access Routine Semantics}

  \noindent authenticateUser(email: email, password: password):
  \begin{itemize}
    \item output: $authToken$, where $authToken$ is a unique token
      generated for the user, tracked, and validated by AWS Cognito.
    \item exception: InvalidCredentials
  \end{itemize}

  \noindent authenticateContractor(email: email, name: name, siteID:
  siteID, userLocation: \{latitude, longitude\}):
  \begin{itemize}
    \item output: $authToken$, where $authToken$ is a unique token
      generated for the user, tracked, and validated by AWS Cognito.
    \item exception: InvalidCredentials, InvalidSiteID,
      LocationVerificationFailed
  \end{itemize}

  \subsubsection{Local Functions}

  N/A

  \section{MIS of API Integration Module}
  \label{sec:AI}

  \subsection{Module}
  API Integration Module
  \subsection{Uses}
  \hyperref[sec:RM]{Routing Module}
  \subsection{Syntax}

  \subsubsection{Exported Constants}
  N/A

  \subsubsection{Exported Access Programs}

  \begin{center}
    \begin{tabular}{>{\raggedright}p{3cm} >{\raggedright}p{5cm}
      >{\raggedright}p{4cm} p{4cm}}
      \hline
      \textbf{Name} & \textbf{In} & \textbf{Out} & \textbf{Exceptions} \\
      \hline
      submitApiRequest & \textbf{request:} HTTPRequest
      \newline \textbf{url:} String \newline \textbf{apiToken
      (optional):} String & HTTPResponse &
      \textbf{NetworkException:} If a valid network connection is not detected.
      \newline \textbf{TimeoutException:} If a response is not received within
      $\hyperlink{timeout}{TIMEOUT}$ seconds.\\
      \hline
    \end{tabular}
  \end{center}

  \subsection{Semantics}

  \subsubsection{State Variables}
  N/A

  \subsubsection{Environment Variables}

  \subsubsection{Assumptions}
  \begin{itemize}
    \item API endpoints are up and functional
    \item The system has a internet network connection
  \end{itemize}
  \subsubsection{Access Routine Semantics}

  \noindent submitApiRequest(url: String, request: HTTPRequest, apiToken:
  Optional\textless String\textgreater):
  \begin{itemize}
    \item output: HTTPResponse
    \item exception: NetworkException, TimeoutException
  \end{itemize}

  \subsubsection{Local Functions}

  \begin{center}
    \begin{tabular}{>{\raggedright}p{3cm} >{\raggedright}p{5cm}
      >{\raggedright}p{4cm} p{4cm}}
      \hline
      \textbf{Name} & \textbf{In} & \textbf{Out} & \textbf{Description} \\
      \hline
      timeElapsed & \textbf{since:} $\mathbb{Z}$ & $\mathbb{Z}$ & Returns the
      number of seconds elapsed since the provided time given in seconds since
      Janurary 1, 1970 \\
      \hline
    \end{tabular}
  \end{center}

  \newpage

  \section{MIS of Document Management Module}
  \label{sec:DM}

  \subsection{Module}

  Document Management Module

  \subsection{Uses}

  \hyperref[sec:DI]{Database Interaction Module}

  \subsection{Syntax}

  \subsubsection{Exported Constants}
  N/A

  \subsubsection{Exported Access Programs}

  \begin{center}
    \begin{tabular}{>{\raggedright}p{3cm} >{\raggedright}p{5cm}
      >{\raggedright}p{4cm} p{4cm}}
      \hline
      \textbf{Name} & \textbf{In} & \textbf{Out} & \textbf{Exceptions} \\
      \hline
      RetrieveDocs & \textbf{siteID:} string & sequence[Document] &
      \textbf{ExternalServiceFailure:} An internal error from AWS \\
      \hline
      CreateDoc & \textbf{s3Link:} string \newline \textbf{userID:}
      string \newline \textbf{siteID:} string \newline
      \textbf{parentDocumentID:} Optional \textless map[string
      $\rightarrow$ any]\textgreater \newline \textbf{expiryDate:}
      Optional \textless string\textgreater \newline
      \textbf{requiresAck:} boolean & map[string $\rightarrow$ any] &
      \textbf{ExternalServiceFailure:} An internal error from AWS \newline
      \textbf{ValidationError:} Non-existent IDs provided \\
      \hline
      EditDoc & \textbf{documentID:} map[string $\rightarrow$ any
        \newline \textbf{userID:} string \newline
        \textbf{s3Link:}
        string & - &
        \textbf{ExternalServiceFailure:} An internal error from AWS \newline
        \textbf{ValidationError:} Non-existent IDs provided \\
        \hline
        DeleteDoc & \textbf{documentID:} map[string $\rightarrow$ any] & - &
        \textbf{ExternalServiceFailure:} An internal error from AWS \newline
        \textbf{ValidationError:} Non-existent IDs provided \\
        \hline
      \end{tabular}
    \end{center}

    \subsection{Semantics}

    \subsubsection{State Variables}

    \begin{center}
      \begin{tabular}{p{4cm} p{12cm}}
        \hline
        \textbf{Name} & \textbf{Description} \\
        \hline
        DocumentDatabase & The underlying AWS DynamoDB table, can be
        represented as a set of items: $\{D_0, D_1, ..., D_n\}$, where
        $D_k \in Documents, k\in[0,n]$ \\
        \hline
      \end{tabular}
    \end{center}

    \subsubsection{Environment Variables}

    \begin{center}
      \begin{tabular}{p{6cm} p{10cm}}
        \hline
        \textbf{Name} & \textbf{Description} \\
        \hline
        LAMBDA\_EXECUTION\_ROLE & When an AWS Lambda Function (The chosen
        AWS compute service for this project), it has an AWS IAM role
        attached to it, that it uses when running. This role needs
        permission for document database access for this module to work \\
        \hline
      \end{tabular}
    \end{center}

    \subsubsection{Assumptions}

    LAMBDA\_EXECUTION\_ROLE has the required permissions in AWS to
    execute the lambda's required tasks.

    \subsubsection{Access Routine Semantics}

    \noindent RetrieveDocs(siteId: sID):
    \begin{itemize}
      \item output: $[D_{k} \in DocumentDatabase \land D_{k}$.siteId ==
        sID], $\forall
        k \in \mathbb{Z}$
      \item exception: ExternalServiceFailure
    \end{itemize}

    \noindent CreateDoc(s3Link: sL, userID: uID, siteID: sID,
      parentDocumentID: pID,
    expiryDate: eD, requiresAck: rA):
    \begin{itemize}
      \item transition: $DocumentDatabase \rightarrow DocumentDatabase
        \cup \{D_{new}\}$, where $D_{new}$.userId = uID $\land$
        $D_{new}$.siteId =
        sId $\land$ $D_{new}$.createdDateTime = $getDateTime()$ $\land$
        $D_{new}$.expiryDate = eD $\land$
        $D_{new}$.requiresAck = rA $\land$ $D_{new}$.s3Link = sL
      \item output: map[string $\rightarrow$ any]: The documentId of the created
        document
      \item exception: ExternalServiceFailure, ValidationError
    \end{itemize}

    \noindent EditDoc(documentId: dID, userId: uID, s3Link: sL):
    \begin{itemize}
      \item transition: $DocumentDatabase \rightarrow
        DocumentDatabase - \{D_{old}
        | D_{old} \in DocumentDatabase \land D_{old} == dID\}$\\
        $DocumentDatabase \rightarrow DocumentDatabase \cup
        \{D_{new}\}$ where $D_{new} = copy(D_{old})$ $\land$
        $D_{new}$.s3Link = sL
        $\land$ $D_{new}$.userId = uID $\land$
        $D_{old}$.lastEditedDateTime = $getDateTime()$ and $copy(D)$ is
        a predicate indicating the
        deep copy of document $D$.
      \item exception: ExternalServiceFailure, ValidationError
    \end{itemize}

    \noindent DeleteDoc(documentId: dID):
    \begin{itemize}
      \item transition: $DocumentDatabase \rightarrow DocumentDatabase -
        \{D_{old} | D_{old} \in DocumentDatabase$ $\land$
        $D_{old}$.documentId == dID \}
      \item exception: ExternalServiceFailure, ValidationError
    \end{itemize}

    \subsubsection{Local Functions}

    \begin{center}
      \begin{tabular}{>{\raggedright}p{4cm} >{\raggedright}p{3cm}
        >{\raggedright}p{3.5cm} p{5.5cm}}
        \hline
        \textbf{Name} & \textbf{In} & \textbf{Out} & \textbf{Description} \\
        \hline
        getDateTime & - & string & Returns the current date and time in ISO8601
        string format\\
        \hline
      \end{tabular}
    \end{center}

    \newpage

    \bibliographystyle {plainnat}
    \bibliography {../../../refs/References}

    \newpage

    \section{Appendix} \label{Appendix}

    \subsection{Symbolic Parameters}
    $\hypertarget{timeout}{TIMEOUT}$ = 5\\

    \subsection{AWS Documentation}
    \href{https://boto3.amazonaws.com/v1/documentation/api/latest/index.html}{Boto3
    Documentation}\\
    \href{https://docs.aws.amazon.com/whitepapers/latest/aws-overview/amazon-web-services-cloud-platform.html}{AWS
    Services Documentation}

    \newpage{}

    \section*{Appendix --- Reflection}

    The information in this section will be used to evaluate the team
    members on the
    graduate attribute of Problem Analysis and Design.

    The purpose of reflection questions is to give you a chance to assess your own
learning and that of your group as a whole, and to find ways to improve in the
future. Reflection is an important part of the learning process.  Reflection is
also an essential component of a successful software development process.  

Reflections are most interesting and useful when they're honest, even if the
stories they tell are imperfect. You will be marked based on your depth of
thought and analysis, and not based on the content of the reflections
themselves. Thus, for full marks we encourage you to answer openly and honestly
and to avoid simply writing ``what you think the evaluator wants to hear.''

Please answer the following questions.  Some questions can be answered on the
team level, but where appropriate, each team member should write their own
response:


    The information in this section will be used to evaluate the team
    members on the
    graduate attribute of Problem Analysis and Design.
    \begin{enumerate}
      \item What went well while writing this deliverable?
        \\
        \\
        In this deliverable, our team was able to delegate tasks and
        manage our time more effectively.
        The stakeholder was engaged to show progress on the prototype and
        obtain feedback that
        influenced the design.

      \item What pain points did you experience during this deliverable, and how
        did you resolve them?
        \\
        \\
        One pain point the team experienced was determining the specific
        technologies which would be the most appropriate
        for the clients problem. One example was deciding the AWS modules
        to use, such as choosing an EC2 instance or using AWS lambda.
        This was resolved through discussion of what the advantages of one
        tool would be over another. For this particular problem,
        it was decided to use AWS lambda because it makes it easy to scale
        to 0 instances when not in use, and the application is only expected to
        be used sporadically during working hours, so costs can be saved by
        being able to scale to zero when not in use.

      \item Which of your design decisions stemmed from speaking to
        your client(s)
        or a proxy (e.g. your peers, stakeholders, potential users)?
        For those that
        were not, why, and where did they come from?
        \\
        \\
        Through meeting with the stakeholders, the design decisions about
        the location verification module, creating accounts, file system module,
        and the function compute module were decided. These arose by
        discussing what these modules would be capable of doing, and how
        they would satisfy specific
        requirements identified in the SRS.

      \item While creating the design doc, what parts of your other
        documents (e.g.
        requirements, hazard analysis, etc), it any, needed to be
        changed, and why?
        \\
        \\
        One part of the SRS document which was required to be changed was
        the original SharePoint integration requirements. This was deemed
        to be too difficult to do and would open up vulnerabilities for
        the stakeholder, and as such was removed from the initial
        requirements identified in the SRS.

      \item What are the limitations of your solution?  Put another way, given
        unlimited resources, what could you do to make the project
        better? (LO\_ProbSolutions)
        \\
        \\
        The main limitations of the current solution is that it doesn't
        stream the location of users in real time.
        It currently takes two points in time, (entering and exiting
        time) into account. This limitation provides reduced visibility
        on the actions of the user at each site,
        which limits its sophistication but increases its simplicity.
        Given more time, the project could be further improved to
        increase the robustness of the verification
        functionality to automatically collect more information from the
        user and consider more complicated edge cases that can arise.

      \item Give a brief overview of other design solutions you
        considered.  What
        are the benefits and tradeoffs of those other designs compared
        with the chosen
        design?  From all the potential options, why did you select the
        documented design?
        (LO\_Explores)
        \\
        \\
        As discussed above, one design solution that was considered was
        an Amazon EC2 instance due to its widespread use and support
        which would be very maintainable
        after the completion date of the project. However, the benefit of
        an AWS Lambda instance with scaling to 0 was determined to be the
        best choice due to the cost
        saving which it is able to provide,

        Another design decision which was explored was the ability to use
        presigned URLs that would permit larger uploads than uploading
        through API Gateway.
    \end{enumerate}

    \end{document}
