\documentclass{article}

\usepackage{float}
\restylefloat{table}

\usepackage{booktabs}

\title{Team Contributions: POC\\\progname}

\author{\authname}

\date{}

\input{../Comments}
\input{../Common}

\begin{document}

\maketitle

This document summarizes the contributions of each team member up to the POC
Demo.  The time period of interest is the time between the beginning of the term
and the POC demo.

\section{Demo Plans}

For information on our POC demonstration plans see the \textit{Proof
of Concept Demonstration Plan} section of the
\href{https://github.com/Spitgranger/SyncMaster/blob/main/docs/DevelopmentPlan/DevelopmentPlan.pdf}{Development
Plan}.

\section{Team Meeting Attendance}

\begin{table}[H]
  \centering
  \begin{tabular}{ll}
    \toprule
    \textbf{Student} & \textbf{Meetings}\\
    \midrule
    Total & 16\\
    Kyle D'Souza & 16\\
    Mitchell Hynes & 16\\
    Richard Fan & 16\\
    Rafeed Iqbal & 12\\
    Akshit Gulia & 14\\
    \bottomrule
  \end{tabular}
\end{table}

There is no exact number, but many of our meetings end up being
unplanned or planned on very short notice, so it is understandable
that there will be a decent number of these where some group members
do not attend simply because we didn't give much heads up about the meeting.

\section{Supervisor/Stakeholder Meeting Attendance}

\begin{table}[H]
  \centering
  \begin{tabular}{ll}
    \toprule
    \textbf{Student} & \textbf{Meetings}\\
    \midrule
    Total & 2\\
    Kyle D'Souza & 0\\
    Mitchell Hynes & 2\\
    Richard Fan & 2\\
    Rafeed Iqbal & 2\\
    Akshit Gulia & 1\\
    \bottomrule
  \end{tabular}
\end{table}

Kyle missed one of the stakeholder meetings due to a previously
discussed scheduling conflict.

\section{Lecture Attendance}

\begin{table}[H]
  \centering
  \begin{tabular}{ll}
    \toprule
    \textbf{Student} & \textbf{Lectures}\\
    \midrule
    Total & 7\\
    Kyle D'Souza & 5\\
    Mitchell Hynes & 6\\
    Richard Fan & 5\\
    Rafeed Iqbal & 0\\
    Akshit Gulia & 5\\
    \bottomrule
  \end{tabular}
\end{table}

We only started recording lecture attendance after September 13th, as
our team was not formed before then, therefore 7 lectures are
recorded for attendance. We had all attended all lecture before the
13th, but we have no record of it, so it is not included in the metric.

\section{TA Document Discussion Attendance}

\begin{table}[H]
  \centering
  \begin{tabular}{ll}
    \toprule
    \textbf{Student} & \textbf{Lectures}\\
    \midrule
    Total & 3\\
    Kyle D'Souza & 3\\
    Mitchell Hynes & 3\\
    Richard Fan & 3\\
    Rafeed Iqbal & 3\\
    Akshit Gulia & 3\\
    \bottomrule
  \end{tabular}
\end{table}

\section{Commits}

\begin{table}[H]
  \centering
  \begin{tabular}{lll}
    \toprule
    \textbf{Student} & \textbf{Commits} & \textbf{Percent}\\
    \midrule
    Total & 326 & 100\% \\
    Kyle D'Souza & 79 & 24.2\% \\
    Mitchell Hynes & 128 & 39.3\% \\
    Richard Fan & 52 & 16.0\% \\
    Rafeed Iqbal & 45 & 13.8\% \\
    Akshit Gulia & 22 & 6.7\% \\
    \bottomrule
  \end{tabular}
\end{table}

The numbers found here come from the contributor insights on the
GitHub repository.

\section{Issue Tracker}

\begin{table}[H]
  \centering
  \begin{tabular}{lll}
    \toprule
    \textbf{Student} & \textbf{Authored (O+C)} & \textbf{Assigned (C only)}\\
    \midrule
    Kyle D'Souza & 35 & 25 \\
    Mitchell Hynes & 77 & 29 \\
    Richard Fan & 6 & 20 \\
    Rafeed Iqbal & 1 & 21 \\
    Akshit Gulia & 3 & 17 \\
    \bottomrule
  \end{tabular}
\end{table}

In terms of issues authored the reason it is so imbalanced is because
generally when assigning work for a deliverable we will all have a
call and Mitchell will create issues and assign them to people (but
  we are all agreeing on whats being created and who its being assigned
to). Recently Kyle did this as well which is why his issues created
is also very high. To our knowledge issues can't be co-authored, so
it leads to that disproportionate issue creation statistic.

\section{CICD}

We will use Github Actions to run unit tests and linters on the
project on every pull request, to ensure consistency of code
formatting and that code added in a pull request does not break the
functionality of our application. There will also be a GitHub Action
to run automated end-to-end tests in the development environment for
our project, this is also to ensure correctness of the project while
testing the whole system rather than individual units of the system.
This action will be run only after deployments into the development
environment. There will also be a build and deploy action for
deploying our application into its development and production environments.

Currently we also have a GitHub action for checking the formatting of
our LaTeX documents, which helps ensure consistent formatting.

\section{Additional Metrics}

\subsection{Pull Request (PR) Merges}

\begin{table}[H]
  \centering
  \begin{tabular}{ll}
    \toprule
    \textbf{Student} & \textbf{PR's Merged}\\
    \midrule
    Total & 104\\
    Kyle D'Souza & 31\\
    Mitchell Hynes & 24\\
    Richard Fan & 20\\
    Rafeed Iqbal & 18\\
    Akshit Gulia & 11\\
    \bottomrule
  \end{tabular}
\end{table}

This metric takes into account the number of pull requests merged
that a student authored. It can be put off a bit if someone took over
a pull request after it was authored, but our team has not done this
very much, so the statistic should be accurate.

\subsection{Pull Request (PR) Reviews}

\begin{table}[H]
  \centering
  \begin{tabular}{ll}
    \toprule
    \textbf{Student} & \textbf{PR's Reviewed}\\
    \midrule
    Kyle D'Souza & 69\\
    Mitchell Hynes & 8\\
    Richard Fan & 49\\
    Rafeed Iqbal & 5\\
    Akshit Gulia & 1\\
    \bottomrule
  \end{tabular}
\end{table}

This metric takes into account the number of pull requests reviewed
by a student which they did not author. The reason for disclusion of
pull requests the student authored is that often reviews by a student
on their own pull request is just responding to a comment, or putting
a small annotation, which is helpful, but is not what we are trying
to gather from this metric.

\end{document}
