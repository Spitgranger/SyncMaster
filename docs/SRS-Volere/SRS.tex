% THIS DOCUMENT IS FOLLOWS THE VOLERE TEMPLATE BY Suzanne Robertson and James Robertson
% ONLY THE SECTION HEADINGS ARE PROVIDED
%
% Initial draft from https://github.com/Dieblich/volere
%
% Risks are removed because they are covered by the Hazard Analysis
\documentclass[12pt]{article}

\usepackage{booktabs}
\usepackage{tabularx}
\usepackage{hyperref}
\usepackage{enumerate}
\hypersetup{
    bookmarks=true,         % show bookmarks bar?
      colorlinks=true,      % false: boxed links; true: colored links
    linkcolor=red,          % color of internal links (change box color with linkbordercolor)
    citecolor=green,        % color of links to bibliography
    filecolor=magenta,      % color of file links
    urlcolor=cyan           % color of external links
}

\newcommand{\lips}{\textit{Insert your content here.}}

%% Comments

\usepackage{color}

\newif\ifcomments\commentstrue %displays comments
%\newif\ifcomments\commentsfalse %so that comments do not display

\ifcomments
\newcommand{\authornote}[3]{\textcolor{#1}{[#3 ---#2]}}
\newcommand{\todo}[1]{\textcolor{red}{[TODO: #1]}}
\else
\newcommand{\authornote}[3]{}
\newcommand{\todo}[1]{}
\fi

\newcommand{\wss}[1]{\authornote{blue}{SS}{#1}} 
\newcommand{\plt}[1]{\authornote{magenta}{TPLT}{#1}} %For explanation of the template
\newcommand{\an}[1]{\authornote{cyan}{Author}{#1}}

%% Common Parts

\newcommand{\progname}{ProgName} % PUT YOUR PROGRAM NAME HERE
\newcommand{\authname}{Team \#, Team Name
\\ Student 1 name
\\ Student 2 name
\\ Student 3 name
\\ Student 4 name} % AUTHOR NAMES                  

\usepackage{hyperref}
    \hypersetup{colorlinks=true, linkcolor=blue, citecolor=blue, filecolor=blue,
                urlcolor=blue, unicode=false}
    \urlstyle{same}
                                


\begin{document}

\title{Software Requirements Specification for \progname: Document Management System} 
\author{\authname}
\date{\today}
	
\maketitle

~\newpage

\pagenumbering{roman}

\tableofcontents

~\newpage

\section*{Revision History}

\begin{tabularx}{\textwidth}{p{3cm}p{2cm}X}
\toprule {\textbf{Date}} & {\textbf{Version}} & {\textbf{Notes}}\\
\midrule
Date 1 & 1.0 & Notes\\
Date 2 & 1.1 & Notes\\
\bottomrule
\end{tabularx}

~\\

~\newpage
\section{Purpose of the Project}
\subsection{User Business}
\lips
\subsection{Goals of the Project}
\lips
\section{Stakeholders}
\subsection{Client}

\begin{itemize}
\item Matt Yakymyshyn (P.Eng)
    \begin{itemize}
        \item[-] Role: Senior Project Manager of Technical Services
        \item[-] Interest: Team leader for the facilities team within
        Technical Services who are the primary group responsible for 
        maintaining facility infrastructure at pumping stations.
        Primary contact at the City for the capstone team.
    \end{itemize}
\end{itemize}

\subsection{Customer}

\begin{itemize}
    \item Technical Services team
    \begin{itemize}
        \item[-] Role: Primary user within Hamilton Water
        \item[-] Interest: As the primary stakeholder, the requirements 
        of this project are aimed at their specific needs. Their main concern
        is that station documentation is managed as a single source of truth,
        and is easily distributed/retrieved with relevant parties.
    \end{itemize} 
\end{itemize}
\subsection{Other Stakeholders}

\begin{itemize}
    \item Supervisory Control and Data Acquisition (SCADA) team
    \begin{itemize}
        \item[-] Role: Responsible for the control system controlling
        the treatment process.
        \item[-] Interest: The SCADA team also manages contractors and is 
        interested in accessing and updating documentation maintained 
        in this system. 
        \item[-] They will have valuable input regarding existing City 
        platforms and technologies, as well as how technology can be 
        integrated into pumping stations.
    \end{itemize}
    \item Corporate Security

    \begin{itemize}
        \item[-] Role: Pumping Station Security
        \item[-] Interest: Provide feedback on any system involving station 
        access.
    \end{itemize}
\end{itemize}

\subsection{Hands-On Users of the Project}
\begin{itemize}
    \item Facilities project managers
    \begin{itemize}
        \item[-] The facilities project managers will be using the application
        daily to perform a variety of tasks. This includes verifying work was
        performed by contractors, accessing station documentation to share with
        contractors and internal staff, receiving signed forms and storing in a
        manner which is associated to the signee, and other tasks.
        They will have to be able to achieve all their needed requirements 
        through the applications user interface, and will be a critical
        stakeholder to receive feedback from through user testing.
    \end{itemize}
    \item Facilities co-op students
    \begin{itemize}
        \item[-] Facilities co-op students will be using the application to
        physically authenticate on site when performing inspections, 
        scoping work, and verifying completion of work.
    \end{itemize}
    \item Facilities contractors
    \begin{itemize}
        \item[-] Many contractors access pumping stations daily. They will
        interact with the application each time they perform facilities related
        work at the station. Interactions would include uploading and 
        receiving documentation, and authenticating their presence on site 
        to verify completion of work.
        Contractors come from a wide range of backgrounds depending on their
        trade, but are typically skilled workers in a particular field and are
        of working age (18 - 65 years old).
    \end{itemize}
    \item Facilities sub-contractors
    \begin{itemize}
        \item[-] Many contractors employ sub-contractors as part of their work.
        While these subcontractors work for the contractor who hired them, they
        will still require access to the application to authenticate on site
        and access site information. Their roles and interactions
        will be similar to contractors, except they will be much less familiar
        with the application and may only use it for a short duration as they 
        are not regular workers.
    \end{itemize}
\end{itemize}
\subsection{Personas}

\begin{itemize}
    \item Greg: A facilities manager
    \begin{itemize}
        \item[-] Greg has a few tasks to complete this morning at work.
        He sees that another department has requested compliance documentation 
        on crane inspections for the ministry, so he signs onto SyncMaster to 
        retrieve the current version of these documents for each of these sites.
        While on the application, he receives a notification that a contractor 
        he manages authenticated at site to complete a work order, but the 
        application identified they had not completed their health and safety 
        training. Greg prompts the system to issue this training to them, and 
        runs a report on this contractor to see if any other employees of 
        theirs require their training soon.
    \end{itemize}
    \item Nancy: A newly onboarded contractor
    \begin{itemize}
        \item[-] Nancy is a plumber who is servicing her first work order after 
        being hired by ``Worldly Plumbing''.
        Her company has notified her of the application she must use to 
        authenticate for the work. When she authenticates, she is notified that 
        the device she is servicing is located within a confined space
        and that she must sign and return a hazard assessment form and proof 
        of confined space training.
    \end{itemize}
\end{itemize}
\subsection{Priorities Assigned to Users}

\begin{itemize}
    \item Key users
    \begin{itemize}
        \item[-] Facilities managers. They are the primary stakeholder of the 
        application and it is intended to be a tool to greatly improve the 
        efficiency of their work.
    \end{itemize}
    \item Secondary users
    \begin{itemize}
        \item[-] Contractors. They will frequently use the system, but their 
        needs are secondary to the key users.
    \end{itemize}
\end{itemize}
\subsection{User Participation}

\begin{itemize}
    \item Facilities managers will be frequently consulted throughout the 
    duration of this project to demonstrate prototypes, receive feedback, 
    and conduct user testing.
    \item Co-op students can also be consulted for user testing.
    They tend to have more availability for longer durations and will be able
    to provide insight into how a less experienced user interacts with the 
    system.

\end{itemize}
\subsection{Maintenance Users and Service Technicians}
The facilities managers will be responsible for ensuring that day-to-day data 
is added to the application appropriately. The team does not have a dedicated
software position, so maintainability of the technical aspects of the 
application would need to be brought to the IT division of the City
when the project reaches that stage.

\section{Mandated Constraints}
\subsection{Solution Constraints}
\lips
\subsection{Implementation Environment of the Current System}
\lips
\subsection{Partner or Collaborative Applications}
\lips
\subsection{Off-the-Shelf Software}
\lips
\subsection{Anticipated Workplace Environment}
\lips
\subsection{Schedule Constraints}
\lips
\subsection{Budget Constraints}
\lips
\subsection{Enterprise Constraints}
\lips

\section{Naming Conventions and Terminology}
\subsection{Glossary of All Terms, Including Acronyms, Used by Stakeholders
involved in the Project}
\lips

\section{Relevant Facts And Assumptions}
\subsection{Relevant Facts}
\lips
\subsection{Business Rules}
\lips
\subsection{Assumptions}
\lips

\section{The Scope of the Work}
\subsection{The Current Situation}
\lips
\subsection{The Context of the Work}
\lips
\subsection{Work Partitioning}
\lips
\subsection{Specifying a Business Use Case (BUC)}
\lips

\section{Business Data Model and Data Dictionary}
\subsection{Business Data Model}
\lips
\subsection{Data Dictionary}
\lips

\section{The Scope of the Product}
\subsection{Product Boundary}
\lips
\subsection{Product Use Case Table}
\lips
\subsection{Individual Product Use Cases (PUC's)}
\lips

\section{Functional Requirements}
\subsection{Functional Requirements}
\lips

\section{Look and Feel Requirements}
\subsection{Appearance Requirements}
\lips
\subsection{Style Requirements}
\lips

\section{Usability and Humanity Requirements}
\subsection{Ease of Use Requirements}
\lips
\subsection{Personalization and Internationalization Requirements}
\lips
\subsection{Learning Requirements}
\lips
\subsection{Understandability and Politeness Requirements}
\lips
\subsection{Accessibility Requirements}
\lips

\section{Performance Requirements}
\subsection{Speed and Latency Requirements}
\lips
\subsection{Safety-Critical Requirements}
\lips
\subsection{Precision or Accuracy Requirements}
\lips
\subsection{Robustness or Fault-Tolerance Requirements}
\lips
\subsection{Capacity Requirements}
\lips
\subsection{Scalability or Extensibility Requirements}
\lips
\subsection{Longevity Requirements}
\lips

\section{Operational and Environmental Requirements}
\subsection{Expected Physical Environment}
\lips
\subsection{Wider Environment Requirements}
\lips
\subsection{Requirements for Interfacing with Adjacent Systems}
\lips
\subsection{Productization Requirements}
\lips
\subsection{Release Requirements}
\lips

\section{Maintainability and Support Requirements}
\subsection{Maintenance Requirements}
\begin{enumerate} [{MS-MTN}1.]
  \item A deployment of the system should take no more than 30 minutes (not
  including testing, and building time).
  \item The build time of the system should be no longer than 10 minutes (not
  including testing time).
  \item All automated tests should be able to run in under 10 minutes
  \item The system should have rigourous unit testing, line coverage should be
  $\ge$ 95\%, branch coverage should be $\ge$ 90\%.
  \item All core functionalities of the system (i.e. Functional Requirements),
  should have both automated end-to-end and unit testing corresponding to them
  \item The project must be able to be maintained by its users, as original
  developers will not be maintaining it after April 2, 2025.
\end{enumerate}
\subsection{Supportability Requirements}
\begin{enumerate} [{MS-SUP}1.]
  \item The application should have user-facing documentation on how to use the
  core functionalities of the system (i.e. functionalities described in
  functional requirements).
  \item The application should have documentation for all API's for future
  maintainers.
  \item The application should have documentation of internal functions and 
  abstractions for future maintainers.
  \item The application should have documentation on deployment, so users can
  deploy this application for themselves.
\end{enumerate}
\subsection{Adaptability Requirements}
\begin{enumerate} [{MS-ADP}1.]
  \item The application must be able to run on at least Google Chrome and
  Microsoft Edge browsers.
  \item The application must be able to run on tablets, smartphones, and
  laptops.
  \item The application must be able to run on Android, IOS, and Windows 10
\end{enumerate}

\section{Security Requirements}
\subsection{Access Requirements}
\lips
\subsection{Integrity Requirements}
\lips
\subsection{Privacy Requirements}
\lips
\subsection{Audit Requirements}
\lips
\subsection{Immunity Requirements}
\lips

\section{Cultural Requirements}
\subsection{Cultural Requirements}
\lips

\section{Compliance Requirements}
\subsection{Legal Requirements}
\lips
\subsection{Standards Compliance Requirements}
\lips

\section{Open Issues}
\lips

\section{Off-the-Shelf Solutions}
\subsection{Ready-Made Products}
\lips
\subsection{Reusable Components}
\lips
\subsection{Products That Can Be Copied}
\lips

\section{New Problems}
\subsection{Effects on the Current Environment}
\lips
\subsection{Effects on the Installed Systems}
\lips
\subsection{Potential User Problems}
\lips
\subsection{Limitations in the Anticipated Implementation Environment That May
Inhibit the New Product}
\lips
\subsection{Follow-Up Problems}
\lips

\section{Tasks}
\subsection{Project Planning}
\lips
\subsection{Planning of the Development Phases}
\lips

\section{Migration to the New Product}
\subsection{Requirements for Migration to the New Product}
\lips
\subsection{Data That Has to be Modified or Translated for the New System}
\lips

\section{Costs}
\lips
\section{User Documentation and Training}
\subsection{User Documentation Requirements}
\lips
\subsection{Training Requirements}
\lips

\section{Waiting Room}
\lips

\section{Ideas for Solution}
\lips

\newpage{}
\section*{Appendix --- Reflection}

The information in this section will be used to evaluate the team members on the
graduate attribute of Lifelong Learning.  Please answer the following questions:

\begin{enumerate}
  \item What knowledge and skills will the team collectively need to acquire to
  successfully complete this capstone project?  Examples of possible knowledge
  to acquire include domain specific knowledge from the domain of your
  application, or software engineering knowledge, mechatronics knowledge or
  computer science knowledge.  Skills may be related to technology, or writing,
  or presentation, or team management, etc.  You should look to identify at
  least one item for each team member.
  \item For each of the knowledge areas and skills identified in the previous
  question, what are at least two approaches to acquiring the knowledge or
  mastering the skill?  Of the identified approaches, which will each team
  member pursue, and why did they make this choice?
\end{enumerate}

\end{document}
