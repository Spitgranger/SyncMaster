% THIS DOCUMENT IS FOLLOWS THE VOLERE TEMPLATE BY Suzanne Robertson and James Robertson
% ONLY THE SECTION HEADINGS ARE PROVIDED
%
% Initial draft from https://github.com/Dieblich/volere
%
% Risks are removed because they are covered by the Hazard Analysis
\documentclass[12pt]{article}

\usepackage{booktabs}
\usepackage{tabularx}
\usepackage{hyperref}
\usepackage{enumerate}
\usepackage{amsmath}
\hypersetup{
    bookmarks=true,         % show bookmarks bar?
      colorlinks=true,      % false: boxed links; true: colored links
    linkcolor=red,          % color of internal links (change box color with linkbordercolor)
    citecolor=green,        % color of links to bibliography
    filecolor=magenta,      % color of file links
    urlcolor=cyan           % color of external links
}

\newcommand{\lips}{\textit{Insert your content here.}}

%% Comments

\usepackage{color}

\newif\ifcomments\commentstrue %displays comments
%\newif\ifcomments\commentsfalse %so that comments do not display

\ifcomments
\newcommand{\authornote}[3]{\textcolor{#1}{[#3 ---#2]}}
\newcommand{\todo}[1]{\textcolor{red}{[TODO: #1]}}
\else
\newcommand{\authornote}[3]{}
\newcommand{\todo}[1]{}
\fi

\newcommand{\wss}[1]{\authornote{blue}{SS}{#1}} 
\newcommand{\plt}[1]{\authornote{magenta}{TPLT}{#1}} %For explanation of the template
\newcommand{\an}[1]{\authornote{cyan}{Author}{#1}}

%% Common Parts

\newcommand{\progname}{ProgName} % PUT YOUR PROGRAM NAME HERE
\newcommand{\authname}{Team \#, Team Name
\\ Student 1 name
\\ Student 2 name
\\ Student 3 name
\\ Student 4 name} % AUTHOR NAMES                  

\usepackage{hyperref}
    \hypersetup{colorlinks=true, linkcolor=blue, citecolor=blue, filecolor=blue,
                urlcolor=blue, unicode=false}
    \urlstyle{same}
                                


\begin{document}

\title{Software Requirements Specification for \progname: Document Management System} 
\author{\authname}
\date{\today}
	
\maketitle

~\newpage

\pagenumbering{roman}

\tableofcontents

~\newpage

\section*{Revision History}

\begin{tabularx}{\textwidth}{p{3cm}p{2cm}X}
\toprule {\textbf{Date}} & {\textbf{Version}} & {\textbf{Notes}}\\
\midrule
Date 1 & 1.0 & Notes\\
Date 2 & 1.1 & Notes\\
\bottomrule
\end{tabularx}

~\\

~\newpage
\section{Purpose of the Project}
\subsection{User Business}
\lips
\subsection{Goals of the Project}
\lips
\section{Stakeholders}
\subsection{Client}

\begin{itemize}
\item Matt Yakymyshyn (P.Eng)
    \begin{itemize}
        \item[-] Role: Senior Project Manager of Technical Services
        \item[-] Interest: Team leader for the facilities team within
        Technical Services who are the primary group responsible for 
        maintaining facility infrastructure at pumping stations.
        Primary contact at the City for the capstone team.
    \end{itemize}
\end{itemize}

\subsection{Customer}

\begin{itemize}
    \item Technical Services team
    \begin{itemize}
        \item[-] Role: Primary user within Hamilton Water
        \item[-] Interest: As the primary stakeholder, the requirements 
        of this project are aimed at their specific needs. Their main concern
        is that station documentation is managed as a single source of truth,
        and is easily distributed/retrieved with relevant parties.
    \end{itemize} 
\end{itemize}
\subsection{Other Stakeholders}

\begin{itemize}
    \item Supervisory Control and Data Acquisition (SCADA) team
    \begin{itemize}
        \item[-] Role: Responsible for the control system controlling
        the treatment process.
        \item[-] Interest: The SCADA team also manages contractors and is 
        interested in accessing and updating documentation maintained 
        in this system. 
        \item[-] They will have valuable input regarding existing City 
        platforms and technologies, as well as how technology can be 
        integrated into pumping stations.
    \end{itemize}
    \item Corporate Security

    \begin{itemize}
        \item[-] Role: Pumping Station Security
        \item[-] Interest: Provide feedback on any system involving station 
        access.
    \end{itemize}
\end{itemize}

\subsection{Hands-On Users of the Project}
\begin{itemize}
    \item Facilities project managers
    \begin{itemize}
        \item[-] The facilities project managers will be using the application
        daily to perform a variety of tasks. This includes verifying work was
        performed by contractors, accessing station documentation to share with
        contractors and internal staff, receiving signed forms and storing in a
        manner which is associated to the signee, and other tasks.
        They will have to be able to achieve all their needed requirements 
        through the applications user interface, and will be a critical
        stakeholder to receive feedback from through user testing.
    \end{itemize}
    \item Facilities co-op students
    \begin{itemize}
        \item[-] Facilities co-op students will be using the application to
        physically authenticate on site when performing inspections, 
        scoping work, and verifying completion of work.
    \end{itemize}
    \item Facilities contractors
    \begin{itemize}
        \item[-] Many contractors access pumping stations daily. They will
        interact with the application each time they perform facilities related
        work at the station. Interactions would include uploading and 
        receiving documentation, and authenticating their presence on site 
        to verify completion of work.
        Contractors come from a wide range of backgrounds depending on their
        trade, but are typically skilled workers in a particular field and are
        of working age (18 - 65 years old).
    \end{itemize}
    \item Facilities sub-contractors
    \begin{itemize}
        \item[-] Many contractors employ sub-contractors as part of their work.
        While these subcontractors work for the contractor who hired them, they
        will still require access to the application to authenticate on site
        and access site information. Their roles and interactions
        will be similar to contractors, except they will be much less familiar
        with the application and may only use it for a short duration as they 
        are not regular workers.
    \end{itemize}
\end{itemize}
\subsection{Personas}

\begin{itemize}
    \item Greg: A facilities manager
    \begin{itemize}
        \item[-] Greg has a few tasks to complete this morning at work.
        He sees that another department has requested compliance documentation 
        on crane inspections for the ministry, so he signs onto SyncMaster to 
        retrieve the current version of these documents for each of these sites.
        While on the application, he receives a notification that a contractor 
        he manages authenticated at site to complete a work order, but the 
        application identified they had not completed their health and safety 
        training. Greg prompts the system to issue this training to them, and 
        runs a report on this contractor to see if any other employees of 
        theirs require their training soon.
    \end{itemize}
    \item Nancy: A newly onboarded contractor
    \begin{itemize}
        \item[-] Nancy is a plumber who is servicing her first work order after 
        being hired by ``Worldly Plumbing''.
        Her company has notified her of the application she must use to 
        authenticate for the work. When she authenticates, she is notified that 
        the device she is servicing is located within a confined space
        and that she must sign and return a hazard assessment form and proof 
        of confined space training.
    \end{itemize}
\end{itemize}
\subsection{Priorities Assigned to Users}

\begin{itemize}
    \item Key users
    \begin{itemize}
        \item[-] Facilities managers. They are the primary stakeholder of the 
        application and it is intended to be a tool to greatly improve the 
        efficiency of their work.
    \end{itemize}
    \item Secondary users
    \begin{itemize}
        \item[-] Contractors. They will frequently use the system, but their 
        needs are secondary to the key users.
    \end{itemize}
\end{itemize}
\subsection{User Participation}

\begin{itemize}
    \item Facilities managers will be frequently consulted throughout the 
    duration of this project to demonstrate prototypes, receive feedback, 
    and conduct user testing.
    \item Co-op students can also be consulted for user testing.
    They tend to have more availability for longer durations and will be able
    to provide insight into how a less experienced user interacts with the 
    system.

\end{itemize}
\subsection{Maintenance Users and Service Technicians}
The facilities managers will be responsible for ensuring that day-to-day data 
is added to the application appropriately. The team does not have a dedicated
software position, so maintainability of the technical aspects of the 
application would need to be brought to the IT division of the City
when the project reaches that stage.

\section{Mandated Constraints}
\subsection{Solution Constraints}
\begin{enumerate} [{C-SOL}1.]
  \item System must be cloud-based to fit in with current existing systems at
  the City of Hamilton.
\end{enumerate}

\subsection{Implementation Environment of the Current System}
To understand the current practices at the City of Hamilton see the
\textit{Problem} section of the Problem Statement and Goals
\href{https://github.com/Spitgranger/capstone/blob/main/docs/ProblemStatementAndGoals/ProblemStatement.md#11-problem}
{here}.

\subsection{Partner or Collaborative Applications}
N/A
\subsection{Off-the-Shelf Software}
\begin{enumerate} [{C-OTS}1.]
  \item The system must integrate with Sharepoint to synchronize documents in
  Sharepoint with documents in the system.
  \item The system must integrate with Infor EAM, to show the status of work
  orders associated with a document.
  \item The system must integrate with MySDS to show relevent relevant SDS
  documents to users on a given site.
\end{enumerate}

\subsection{Anticipated Workplace Environment}
% TODO

\subsection{Schedule Constraints}
\begin{enumerate} [{C-SCH}1.]
  \item A requirement is to integrate with the Infor EAM system the city intends
  on using however, this system will not be available until February 2025, so no
  testing can be done on this system until then.

  \item The project deadline is April 2, 2025.
\end{enumerate}

\subsection{Budget Constraints}
\begin{enumerate} [{C-BDG}1.]
  \item Total expenses up until April 2, 2025 must not exceed \$750.
\end{enumerate}

\subsection{Enterprise Constraints}
N/A

\section{Naming Conventions and Terminology}
\subsection{Glossary of All Terms, Including Acronyms, Used by Stakeholders
involved in the Project}
\begin{itemize}
    \item BAS: Building Automation System.
    \item BCOS: Beyond Compliance Operating System.
    \item CMMS: Computerized Maintenance Management System.
    \item Confined Space: A partially or fully enclosed space, not designed for 
    continuous human occupancy, which has the potential for 
    atmospheric hazards.
    \item Controlled Space: A City defined space which has the potential to
    become a confined space.
    \item CSE: Confined Space Entry
    \item Dry Well: A room which houses industrial equipment such as pumps 
    and valves.
    \item Hazard Assessment Form: A form outlining potential
    hazards at a location.
    \item Hot Works: Work which produces ignition sources.
    Requires an accompanying hot works permit.
    \item HVAC: Heating, Ventilation, and Air Conditiong.
    \item Infor EAM: An Enterprise Asset Management system.
    \item PMATS: Plant Maintenance and Technical Services.
    \item PO: Purchase Order.
    \item PPE: Personal Protective Equipment.
    \item SCADA: Supervisory Control and Data Acquisition.
    \item SDS: Safety Data Sheets.
    \item Wet Well: A portion of a wastewater pumping station which receives
    and temporarily stores wastewater.
\end{itemize}

\section{Relevant Facts And Assumptions}
\subsection{Relevant Facts}
\begin{itemize}
    \item There are over 100 pumping stations throughout the City.
    \item There are dozens of service contracts, and many contractors
    and staff accessing stations each day.
    \item Some stations are in remote locations and may have poor cell signal.
    \item Some properties have multiple stations at the same site.
    \item Certain stations have special entry procedures.
\end{itemize}

\subsection{Business Rules}
\begin{itemize}
    \item External contractors do not have access to the internal network.
    \item Only authorized users can approve new procedures. 
    \item There is work at stations performed through a work order
    and routine work which doesn't have a work order.
    \item Employee collective agreements have restrictions on the release of
    GPS logs and video recording, which would require their approval.
\end{itemize}
\subsection{Assumptions}
\begin{itemize}
    \item Our project will focus on the application itself
    and our team will implement an authentication system adequate for our
    development. When the City integrates our application into their systems 
    they would replace our authentication system with one that meets their 
    detailed cybersecurity specifications.
    \item The existing entry and exit procedure for stations is not modified
    and our application will coexist with it.
\end{itemize}
\section{The Scope of the Work}
\subsection{The Current Situation}
\lips
\subsection{The Context of the Work}
\lips
\subsection{Work Partitioning}
\lips
\subsection{Specifying a Business Use Case (BUC)}
\lips

\section{Business Data Model and Data Dictionary}
\subsection{Business Data Model}
\lips
\subsection{Data Dictionary}
\lips

\section{The Scope of the Product}
\subsection{Product Boundary}
\lips
\subsection{Product Use Case Table}
\lips
\subsection{Individual Product Use Cases (PUC's)}
\lips

\section{Functional Requirements}
\subsection{Formal Definitions}
\begin{itemize}
  \item Let \(U = \{u_1, u_2, \cdots, u_i\}\) be the set of users in the system.
  \item Let \(R = \{r_1, r_2, \cdots, r_r\}\) be the set of roles (e.g. manager,
  contractor, subcontractor).
  \item Let \(A = \{a_1, a_2, \cdots, a_s\}\) be the set of actions (e.g.
    upload, authenticate, sign, view).
  \item Let \(C = \{c_1, c_2, \cdots, c_t\}\) be the set of compliance
    requirements (e.g. safety training, hazard assesments).
  \item Let \(I = \{l_1, l_2, \cdots, l_q\}\) be the set of allowed locations.
  \item Let \(loc(u_i)\) represent the current location of user \(u_i\).
  \item Let \(D = \{d_1, d_2, \cdots, d_n\}\) be the set of documents.
  \item Let \(hasRole(u_i, r_k)\) represent that user \(u_i \in U\) has a role
    \(r_k \in R\).
  \item Let \(notify(u_m, u_i\) represent a notification from user \(u_m \in U\)
    to user \(u_i \in U\).
  \item Let \(manages(u_m, u_i)\) represent that user \(u_m \in U\)
    manages user \(u_i \in U\).
  \item Let \(permitted(r_k, a_s, d_j)\) represent that a user with role \(r_k \in R\)
    is permitted to perform action \(a_s \in A\) on document \(d_j \in D\)
  \item Let \(requires(a_s, c_t)\) represent that action \(a_s \in A\) requires
    compliance requirement \(c_t \in C\)
  \item Let \(T(u_i) \subset C\) represent the set of completed training for a
    user \(u_i \in U\).
  \item Let \(assoc(d_j, u_i)\) denote that document \(d_j\) is associated with
    user \(u_i\).
  \item Let \(upload(u_i, d_j)\) represent the action of user \(u_i \in U\)
    uploading document \(d_j \in D\).
  \item Let \(performAction(u_i, a_s, d_j)\) represent that user \(u_i \in U\)
    performs action \(a_s \in A\) on document \(d_j \in D\).
\end{itemize}
\subsection{Formal Expressions}
\begin{enumerate}
  \item \(\forall u_i \in U, \forall d_j \in D, \forall r_k \in R, \forall a_s
    \in A: \)
    \begin{align*}
      hasRole(u_i, r_k) \land permitted(r_k, a_s, d_j) \implies performAction(u_i, a_s, d_j)
    \end{align*}
  \item \(\forall u_i \in U, \forall c_t \in C:\)
    \begin{align*}
      (requires(a_s, c_t) \land a_s = sign \implies (c_t \in T(u_i)) \implies
      performAction(u_i, a_s, d_j))
    \end{align*}
  \item \(\forall u_i \in U, \forall d_j \in D:\)
    \begin{align*}
      (loc(u_i) \in L) \implies upload(u_i, d_j)
    \end{align*}
  \item \(\forall u_i \in U, \forall d_j \in D:\)
    \begin{align*}
      (loc(u_i) \notin L) \implies \lnot upload(u_i, d_j)
    \end{align*}
  \item \(\forall u_i \in U, \forall d_j \in D:\)
    \begin{align*}
      sign(u_i, d_j) \implies assoc(d_j, u_i)
    \end{align*}
  \item \(\forall u_i \in U, \forall c_t \in C, \forall u_m \in U:\)
    \begin{align*}
      (manages(u_m, u_i) \land c_t \notin T(u_i)) \implies notify(u_m, u_i)
    \end{align*}
\end{enumerate}
\subsection{Functional Requirements}
\begin{enumerate} [{FR}1.]
  \item System should support at a minimum the docx, xlsx, and pdf file formats for
        upload and viewing.
  \item Documents in this system must be synchronized with Sharepoint, this means any
        changes in this system must be reflected in the corresponding Sharepoint
        document, and any changes to the document in Sharepoint must be reflected in
        this system.
  \item The system should have different access levels depending on the type of user.
        Only Admin users should be able to change access permissions of other users and
        have read and write accesses to all documents in the system.
        Contractors/general users should only have read access to documents and be able
        to sign documents.
  \item The current site, work order number, job details, and entry/exit time of 
        contractors should be visible to admin users in the system.
\end{enumerate}

\section{Look and Feel Requirements}
\subsection{Appearance Requirements}
\lips
\subsection{Style Requirements}
\lips

\section{Usability and Humanity Requirements}
\subsection{Ease of Use Requirements}
\begin{enumerate}[{UH-EU}1.]
    \item Users between the ages of {MIN\_AGE} and {MAX\_AGE},
      regardless of technical expertise, must be able to discover at least 70\% of
      the web application functionality within 10 minutes of being introduced to
      the system without any explaination or training.\\
      \textbf{Rationale}: Usability is of high importance to the users of the system.
      They will be mainly non-technical and must be able to learn and use the
      system quickly if it is to be of any value to them.
    \item The system should provide undo options or warnings for irreversible actions
      and 95\% of users should successfully use these features when prompted.\\
      \textbf{Rationale}: It is important to provide users of the systems with
      confirmations or warnings to prevent errors and if they do happen, make it
      easy to recover from them.
\end{enumerate}
\subsection{Personalization and Internationalization Requirements}
N/A
\subsection{Learning Requirements}
\begin{enumerate}[{UH-LR}1.]
    \item Users between the ages of {MIN\_AGE} and {MAX\_AGE} should not take
      more than 10 minutes to a learn feature of the web application after it is
      discovered.\\
      \textbf{Rationale}: A well designed system should give the user the most amount of
      functionality while being easy to understand. Users will be able to
      productive with the system in a short amount of time, reducing
      frustration.
    \item Users between the ages of {MIN\_AGE} and {MAX\_AGE} should be able to 
      complete the system onboarding tutorial or walkthrough within 5 minutes,
      and 80\% of users should report feeling confident in using the system afterward.\\
      \textbf{Rationale}: It is important that users are able to understand
      system documentation quickly to reduce mental overhead and increase
      productivity.
\end{enumerate}
\subsection{Understandability and Politeness Requirements}
\begin{enumerate}[{UH-UP}1.]
    \item The application must not contain symbols or allusions that
      may be offensive or politically charged.\\
      \textbf{Rationale}: As the system is to be used by people of diverse backgrounds
      and in a professional setting, it is important to keep a professional and
      neutural tone.
    \item System error messages should clearly explain the issue and suggest a resolution
      within 2 sentences. 90\% of users between the ages of {MIN\_AGE} and {MAX\_AGE}
      should report understanding the error and being able to resolve the issue without
      external support.\\
      \textbf{Rationale}: Useful system feedback is crucial since users need to
      know what the system is doing and how it interprets their input in order to
      determine next steps.
\end{enumerate}
\subsection{Accessibility Requirements}
TBD may be in constraints.
\section{Performance Requirements}
\subsection{Speed and Latency Requirements}
\lips
\subsection{Safety-Critical Requirements}
\lips
\subsection{Precision or Accuracy Requirements}
\lips
\subsection{Robustness or Fault-Tolerance Requirements}
\lips
\subsection{Capacity Requirements}
\lips
\subsection{Scalability or Extensibility Requirements}
\lips
\subsection{Longevity Requirements}
\lips

\section{Operational and Environmental Requirements}
\subsection{Expected Physical Environment}
\begin{enumerate} [{OE-PE}1.]
  \item Application should be functional in City of Hamilton, Water Division sites and offices.
\end{enumerate}
\subsection{Wider Environment Requirements}
\begin{enumerate} [{OE-WE}1.]
  \item Application should be functional on Mobile and Desktop web browser layouts.
  \item Application should be able to run on Chrome, Microsoft Edge, and Mobile Browsers.
\end{enumerate}
\subsection{Requirements for Interfacing with Adjacent Systems}
\begin{enumerate} [{OE-IAS}1.]
  \item Application should integrate with existing SharePoint repositories.
  \item Application should be able to provide up-to-date Safety Data Sheets from MySDS.
  \item Application should be open for integration with upcoming Work Order tracking system in the city's Enterprise Asset Management software.
\end{enumerate}
\subsection{Productization Requirements}
N/A
\subsection{Release Requirements}
\begin{enumerate} [{OE-REL}1.]
  \item A changelog should be generated with every release documenting changes in features, requirements and fixes made.
  \item A release is defined as a Revision. Every revision should be a major deployment of new features and/or fixes into production.
  \item Expected release of Revision 0: February 1st, 2024
  \item Expected release of Revision 1: March 30th, 2025
\end{enumerate}

\section{Maintainability and Support Requirements}
\subsection{Maintenance Requirements}
\begin{enumerate} [{MS-MTN}1.]
  \item A deployment of the system should take no more than 30 minutes (not
  including testing, and building time).
  \item The build time of the system should be no longer than 10 minutes (not
  including testing time).
  \item All automated tests should be able to run in under 10 minutes
  \item The system should have rigourous unit testing, line coverage should be
  $\ge$ 95\%, branch coverage should be $\ge$ 90\%.
  \item All core functionalities of the system (i.e. Functional Requirements),
  should have both automated end-to-end and unit testing corresponding to them
  \item The project must be able to be maintained by its users, as original
  developers will not be maintaining it after April 2, 2025.
\end{enumerate}
\subsection{Supportability Requirements}
\begin{enumerate} [{MS-SUP}1.]
  \item The application should have user-facing documentation on how to use the
  core functionalities of the system (i.e. functionalities described in
  functional requirements).
  \item The application should have documentation for all API's for future
  maintainers.
  \item The application should have documentation of internal functions and 
  abstractions for future maintainers.
  \item The application should have documentation on deployment, so users can
  deploy this application for themselves.
\end{enumerate}
\subsection{Adaptability Requirements}
\begin{enumerate} [{MS-ADP}1.]
  \item The application must be able to run on at least Google Chrome and
  Microsoft Edge browsers.
  \item The application must be able to run on tablets, smartphones, and
  laptops.
  \item The application must be able to run on Android, IOS, and Windows 10
\end{enumerate}

\section{Security Requirements}
\subsection{Access Requirements}
\lips
\subsection{Integrity Requirements}
\lips
\subsection{Privacy Requirements}
\lips
\subsection{Audit Requirements}
\lips
\subsection{Immunity Requirements}
\lips

\section{Cultural Requirements}
\subsection{Cultural Requirements}
\lips

\section{Compliance Requirements}
\subsection{Legal Requirements}
\lips
\subsection{Standards Compliance Requirements}
\lips

\section{Open Issues}
\lips

\section{Off-the-Shelf Solutions}
\subsection{Ready-Made Products}
Currently there exist many document management systems (i.e. Google Docs,
Sharepoint). However, They miss some of the clients major requirements. The
city wants to be able to integrate with their work order management system to
show the status of a work order that is associated with any given document,
but existing solutions do not provide this capability. They also want to be
able to verify that people were at a given site, when completing work, which
again there isn't a ready made product to do.
\subsection{Reusable Components}
We can use Sharepoint as file storage, since the city wants Sharepoint and this
system to be in sync, and storing the files in two seperate locations and then
syncing them will introduce a lot of overhead. Instead, all files can just be
stored on Sharepoint.
\subsection{Products That Can Be Copied}
N/A

\section{New Problems}
\subsection{Effects on the Current Environment}
\lips
\subsection{Effects on the Installed Systems}
\lips
\subsection{Potential User Problems}
\lips
\subsection{Limitations in the Anticipated Implementation Environment That May
Inhibit the New Product}
\lips
\subsection{Follow-Up Problems}
\lips

\section{Tasks}
\subsection{Project Planning}
\lips
\subsection{Planning of the Development Phases}
\lips

\section{Migration to the New Product}
\subsection{Requirements for Migration to the New Product}
\begin{enumerate} [{MI-NP}1.]
  \item The system must be compatible with existing user roles in Active
    Directory.\\
    \textbf{Rationale}: Compatibility with existing business rules is needed
    to ensure migration can be completed in a short period of time without
    having to defined new roles.
\end{enumerate}

\subsection{Data That Has to be Modified or Translated for the New System}
\begin{enumerate} [{MI-TR}1.]
  \item Information stored on paper must be digitized for consumption for the
    system.\\
    \textbf{Rationale}: Some content that the system is to consume has not yet
    been digitized. It will have to be digitized before the system is able to
    use it.
\end{enumerate}

\section{Costs}
\lips
\section{User Documentation and Training}
\subsection{User Documentation Requirements}
\lips
\subsection{Training Requirements}
\lips

\section{Waiting Room}
\lips

\section{Ideas for Solution}
\lips

\newpage{}
\section*{Appendix --- Reflection}
\subsection{Symbolic Parameters}
$\hypertarget{min_age}{MIN\_AGE}$ = 18\\
$\hypertarget{min_age}{MAX\_AGE}$ = 70\\

The information in this section will be used to evaluate the team members on the
graduate attribute of Lifelong Learning.  Please answer the following questions:

\begin{enumerate}
  \item What knowledge and skills will the team collectively need to acquire to
  successfully complete this capstone project?  Examples of possible knowledge
  to acquire include domain specific knowledge from the domain of your
  application, or software engineering knowledge, mechatronics knowledge or
  computer science knowledge.  Skills may be related to technology, or writing,
  or presentation, or team management, etc.  You should look to identify at
  least one item for each team member.
  \item For each of the knowledge areas and skills identified in the previous
  question, what are at least two approaches to acquiring the knowledge or
  mastering the skill?  Of the identified approaches, which will each team
  member pursue, and why did they make this choice?
\end{enumerate}

\end{document}
