\documentclass[12pt, titlepage]{article}

\usepackage{booktabs}
\usepackage{tabularx}
\usepackage{hyperref}
\hypersetup{
  colorlinks,
  citecolor=blue,
  filecolor=black,
  linkcolor=red,
  urlcolor=blue
}
\usepackage[round]{natbib}
\usepackage{longtable}

%% Comments

\usepackage{color}

\newif\ifcomments\commentstrue %displays comments
%\newif\ifcomments\commentsfalse %so that comments do not display

\ifcomments
\newcommand{\authornote}[3]{\textcolor{#1}{[#3 ---#2]}}
\newcommand{\todo}[1]{\textcolor{red}{[TODO: #1]}}
\else
\newcommand{\authornote}[3]{}
\newcommand{\todo}[1]{}
\fi

\newcommand{\wss}[1]{\authornote{blue}{SS}{#1}} 
\newcommand{\plt}[1]{\authornote{magenta}{TPLT}{#1}} %For explanation of the template
\newcommand{\an}[1]{\authornote{cyan}{Author}{#1}}

%% Common Parts

\newcommand{\progname}{ProgName} % PUT YOUR PROGRAM NAME HERE
\newcommand{\authname}{Team \#, Team Name
\\ Student 1 name
\\ Student 2 name
\\ Student 3 name
\\ Student 4 name} % AUTHOR NAMES                  

\usepackage{hyperref}
    \hypersetup{colorlinks=true, linkcolor=blue, citecolor=blue, filecolor=blue,
                urlcolor=blue, unicode=false}
    \urlstyle{same}
                                


\begin{document}

\title{System Verification and Validation Plan for \progname{}}
\author{\authname}
\date{\today}

\maketitle

\pagenumbering{roman}

\section*{Revision History}

\begin{tabularx}{\textwidth}{p{3cm}p{2cm}X}
  \toprule {\bf Date} & {\bf Version} & {\bf Notes}\\
  \midrule
  Date 1 & 1.0 & Notes\\
  Date 2 & 1.1 & Notes\\
  \bottomrule
\end{tabularx}

~\\
\wss{The intention of the VnV plan is to increase confidence in the software.
  However, this does not mean listing every verification and
  validation technique
  that has ever been devised.  The VnV plan should also be a \textbf{feasible}
  plan. Execution of the plan should be possible with the time and
  team available.
  If the full plan cannot be completed during the time available, it
  can either be
  modified to ``fake it'', or a better solution is to add a section describing
what work has been completed and what work is still planned for the future.}

\wss{The VnV plan is typically started after the requirements stage, but before
  the design stage.  This means that the sections related to unit testing cannot
  initially be completed.  The sections will be filled in after the design stage
  is complete.  the final version of the VnV plan should have all
  sections filled
in.}

\newpage

\tableofcontents

\listoftables
\wss{Remove this section if it isn't needed}

\listoffigures
\wss{Remove this section if it isn't needed}

\newpage

\section{Symbols, Abbreviations, and Acronyms}

\renewcommand{\arraystretch}{1.2}
\begin{tabular}{l l}
  \toprule
  \textbf{symbol} & \textbf{description}\\
  \midrule
  T & Test\\
  \bottomrule
\end{tabular}\\

\wss{symbols, abbreviations, or acronyms --- you can simply reference the SRS
\citep{SRS} tables, if appropriate}

\wss{Remove this section if it isn't needed}

\newpage

\pagenumbering{arabic}

This document ... \wss{provide an introductory blurb and roadmap of the
Verification and Validation plan}

\section{General Information}

\subsection{Summary}

\wss{Say what software is being tested.  Give its name and a brief overview of
its general functions.}

\subsection{Objectives}

\wss{State what is intended to be accomplished.  The objective will be around
  the qualities that are most important for your project.  You might have
  something like: ``build confidence in the software correctness,''
  ``demonstrate adequate usability.'' etc.  You won't list all of the qualities,
just those that are most important.}

\wss{You should also list the objectives that are out of scope.  You don't have
  the resources to do everything, so what will you be leaving out.
  For instance,
  if you are not going to verify the quality of usability, state
  this.  It is also
worthwhile to justify why the objectives are left out.}

\wss{The objectives are important because they highlight that you are aware of
  limitations in your resources for verification and validation.  You
  can't do everything,
  so what are you going to prioritize?  As an example, if your system
  depends on an
  external library, you can explicitly state that you will assume
  that external library
has already been verified by its implementation team.}

\subsection{Challenge Level and Extras}

The challenge level for this project is a general level, and the
extras are conducting user testing and developing a user manual. For
more information on the challenge level and extras see section
\textit{Challenge Level and Extras} of the Problem Statement
\href{https://github.com/Spitgranger/SyncMaster/blob/main/docs/ProblemStatementAndGoals/ProblemStatement.md#4-challenge-level-and-extras}
{here}.

\subsection{Relevant Documentation}

\wss{Reference relevant documentation.  This will definitely include your SRS
  and your other project documents (design documents, like MG, MIS, etc).  You
  can include these even before they are written, since by the time the project
  is done, they will be written.  You can create BibTeX entries for your
documents and within those entries include a hyperlink to the documents.}

\citet{SRS}

\wss{Don't just list the other documents.  You should explain why
  they are relevant and
how they relate to your VnV efforts.}

\section{Plan}

\wss{Introduce this section.  You can provide a roadmap of the sections to
come.}

\subsection{Verification and Validation Team}

\wss{Your teammates.  Maybe your supervisor.
  You should do more than list names.  You should say what each person's role is
  for the project's verification.  A table is a good way to summarize
this information.}

\subsection{SRS Verification Plan}

To verify the SRS document, we will use the following methods:
\begin{enumerate}
  \item Formal reviews with our TA, Yiding. A checklist will be used to track
    the status of the review and be used as an instrument to verify our SRS.
    The checklist to be used is at the end of this section of the document.
  \item Ad-hoc peer reviews tracked through GitHub issues. Throughout the
    course we receive peer feedback on the quality of our SRS and VnV Plan. The
    peer feedback provided is valuable input which we will action through issues
    and pull requests as identified in the development plan.
  \item Milestone feedback provided through Avenue will also be used to
    improve the checklist which we use as our instrument to measure
    SRS verification.
\end{enumerate}
The checklist below will be used during formal reviews for SRS verification
with our TA.
\begin{itemize}
  \item Have the requirements been listed in the appropriate sections of the
    SRS document?
  \item Are all definitions and acronyms defined in the glossary of terms?
  \item Are the system inputs properly specified?
  \item Are the system outputs properly specified?
  \item Do the functional requirements avoid conflicts with other requirements?
  \item Do the functional requirements avoid specifying the system design?
  \item Are the requirements clear enough that they could be implemented by an
    independent engineering team correctly?
  \item Is each requirement testable?
  \item Is independent testing of each requirement possible?
  \item Are the functional requirements traceable to a use case of the system?
  \item Are there open issues from reviewers or pull requests not merged
    into the main branch?
\end{itemize}

\subsection{Design Verification Plan}

\wss{Plans for design verification}

\wss{The review will include reviews by your classmates}

\wss{Create a checklists?}

\subsection{Verification and Validation Plan Verification Plan}

\wss{The verification and validation plan is an artifact that should also be
verified.  Techniques for this include review and mutation testing.}

\wss{The review will include reviews by your classmates}

\wss{Create a checklists?}

\subsection{Implementation Verification Plan}

\wss{You should at least point to the tests listed in this document and the unit
testing plan.}

\wss{In this section you would also give any details of any plans for static
  verification of the implementation.  Potential techniques include code
walkthroughs, code inspection, static analyzers, etc.}

\wss{The final class presentation in CAS 741 could be used as a code
  walkthrough.  There is also a possibility of using the final presentation (in
CAS741) for a partial usability survey.}

\subsection{Automated Testing and Verification Tools}

\wss{What tools are you using for automated testing.  Likely a unit testing
  framework and maybe a profiling tool, like ValGrind.  Other possible tools
  include a static analyzer, make, continuous integration tools, test coverage
  tools, etc.  Explain your plans for summarizing code coverage metrics.
  Linters are another important class of tools.  For the programming language
  you select, you should look at the available linters.  There may also be tools
  that verify that coding standards have been respected, like flake9 for
Python.}

\wss{If you have already done this in the development plan, you can point to
that document.}

\wss{The details of this section will likely evolve as you get closer to the
implementation.}

\subsection{Software Validation Plan}

\wss{If there is any external data that can be used for validation, you should
  point to it here.  If there are no plans for validation, you should state that
here.}

\wss{You might want to use review sessions with the stakeholder to check that
  the requirements document captures the right requirements.  Maybe task based
inspection?}

\wss{For those capstone teams with an external supervisor, the Rev 0 demo should
  be used as an opportunity to validate the requirements.  You should plan on
  demonstrating your project to your supervisor shortly after the
  scheduled Rev 0 demo.
  The feedback from your supervisor will be very useful for improving
your project.}

\wss{For teams without an external supervisor, user testing can serve
  the same purpose
as a Rev 0 demo for the supervisor.}

\wss{This section might reference back to the SRS verification section.}

\section{System Tests}

\wss{There should be text between all headings, even if it is just a roadmap of
the contents of the subsections.}

\subsection{Tests for Functional Requirements}

\wss{Subsets of the tests may be in related, so this section is divided into
  different areas.  If there are no identifiable subsets for the tests, this
level of document structure can be removed.}

\wss{Include a blurb here to explain why the subsections below
cover the requirements.  References to the SRS would be good here.}

\subsubsection{Area of Testing1}

\wss{It would be nice to have a blurb here to explain why the subsections below
  cover the requirements.  References to the SRS would be good here.
  If a section
  covers tests for input constraints, you should reference the data constraints
table in the SRS.}

\paragraph{Title for Test}

\begin{enumerate}

  \item{test-id1\\}

    Control: Manual versus Automatic

    Initial State:

    Input:

    Output: \wss{The expected result for the given inputs.  Output is
      not how you
      are going to return the results of the test.  The output is the expected
    result.}

    Test Case Derivation: \wss{Justify the expected value given in
    the Output field}

    How test will be performed:

  \item{test-id2\\}

    Control: Manual versus Automatic

    Initial State:

    Input:

    Output: \wss{The expected result for the given inputs}

    Test Case Derivation: \wss{Justify the expected value given in
    the Output field}

    How test will be performed:

\end{enumerate}

\subsubsection{Area of Testing2}

...

\subsection{Tests for Nonfunctional Requirements}

\wss{The nonfunctional requirements for accuracy will likely just reference the
  appropriate functional tests from above.  The test cases should mention
  reporting the relative error for these tests.  Not all projects will
necessarily have nonfunctional requirements related to accuracy.}

\wss{For some nonfunctional tests, you won't be setting a target threshold for
  passing the test, but rather describing the experiment you will do to measure
  the quality for different inputs.  For instance, you could measure
  speed versus
  the problem size.  The output of the test isn't pass/fail, but
  rather a summary
table or graph.}

\wss{Tests related to usability could include conducting a usability test and
survey.  The survey will be in the Appendix.}

\wss{Static tests, review, inspections, and walkthroughs, will not follow the
format for the tests given below.}

\wss{If you introduce static tests in your plan, you need to provide details.
  How will they be done?  In cases like code (or document)
  walkthroughs, who will
be involved? Be specific.}

\subsubsection{Area of Testing1}

\paragraph{Title for Test}

\begin{enumerate}

  \item{test-id1\\}

    Type: Functional, Dynamic, Manual, Static etc.

    Initial State:

    Input/Condition:

    Output/Result:

    How test will be performed:

  \item{test-id2\\}

    Type: Functional, Dynamic, Manual, Static etc.

    Initial State:

    Input:

    Output:

    How test will be performed:

\end{enumerate}

\subsubsection{Area of Testing2}

...

\subsection{Traceability Between Test Cases and Requirements}

\wss{Provide a table that shows which test cases are supporting which
requirements.}

\begin{longtable}{|l|l|}
  \hline
  \textbf{Req. ID} & \textbf{Test ID's} \\
  \hline
  FR1 & \\ \hline
  FR2 & \\ \hline
  FR3 & \\ \hline
  FR4 & \\ \hline
  FR5 & \\ \hline
  FR6 & \\ \hline
  FR7 & \\ \hline
  FR8 & \\ \hline
  FR9 & \\ \hline
  LF-AP1 & \\ \hline
  LF-ST1 & \\ \hline
  UH-EU1 & \\ \hline
  UH-EU2 & \\ \hline
  UH-LR1 & \\ \hline
  UH-LR2 & \\ \hline
  UH-UP1 & \\ \hline
  UH-UP2 & \\ \hline
  UH-AS1 & \\ \hline
  PR-SL1 & \\ \hline
  PR-SL2 & \\ \hline
  PR-SL3 & \\ \hline
  PR-SC1 & \\ \hline
  PR-SC2 & \\ \hline
  PR-PA1 & \\ \hline
  PR-PA2 & \\ \hline
  PR-RFT1 & \\ \hline
  PR-RFT2 & \\ \hline
  PR-CR1 & \\ \hline
  PR-CR2 & \\ \hline
  PR-SE1 & \\ \hline
  PR-SE2 & \\ \hline
  PR-LR1 & \\ \hline
  OE-PE1 & \\ \hline
  OE-WE1 & \\ \hline
  OE-WE2 & \\ \hline
  OE-IAS1 & \\ \hline
  OE-IAS2 & \\ \hline
  OE-IAS2 & \\ \hline
  OE-REL1 & \\ \hline
  OE-REL2 & \\ \hline
  OE-REL3 & \\ \hline
  OE-REL4 & \\ \hline
  MS-MTN1 & \\ \hline
  MS-MTN2 & \\ \hline
  MS-MTN3 & \\ \hline
  MS-MTN4 & \\ \hline
  MS-MTN5 & \\ \hline
  MS-MTN6 & \\ \hline
  MS-SUP1 & \\ \hline
  MS-SUP2 & \\ \hline
  MS-SUP3 & \\ \hline
  MS-SUP4 & \\ \hline
  MS-ADP1 & \\ \hline
  MS-ADP2 & \\ \hline
  MS-ADP3 & \\ \hline
  SR-AR1 & \\ \hline
  SR-AR2 & \\ \hline
  SR-AR3 & \\ \hline
  SR-AR4 & \\ \hline
  SR-IR1 & \\ \hline
  SR-IR2 & \\ \hline
  SR-IR3 & \\ \hline
  SR-PR1 & \\ \hline
  MS-PR2 & \\ \hline
  SR-AU1 & \\ \hline
  SR-IMR1 & \\ \hline
  CR-CR1 & \\ \hline
  CR-L1 & \\ \hline
  CR-L2 & \\ \hline
  CR-S1 & \\ \hline
  \caption{Requirements to Test Case Traceability Matrix}
\end{longtable}

\section{Unit Test Description}

\wss{This section should not be filled in until after the MIS (detailed design
document) has been completed.}

\wss{Reference your MIS (detailed design document) and explain your overall
philosophy for test case selection.}

\wss{To save space and time, it may be an option to provide less
  detail in this section.
  For the unit tests you can potentially layout your testing strategy
  here.  That is, you
  can explain how tests will be selected for each module.  For
  instance, your test building
  approach could be test cases for each access program, including one
  test for normal behaviour
  and as many tests as needed for edge cases.  Rather than create the
  details of the input
  and output here, you could point to the unit testing code.  For
  this to work, you code
needs to be well-documented, with meaningful names for all of the tests.}

\subsection{Unit Testing Scope}

\wss{What modules are outside of the scope.  If there are modules that are
  developed by someone else, then you would say here if you aren't planning on
  verifying them.  There may also be modules that are part of your software, but
  have a lower priority for verification than others.  If this is the case,
explain your rationale for the ranking of module importance.}

\subsection{Tests for Functional Requirements}

\wss{Most of the verification will be through automated unit testing.  If
  appropriate specific modules can be verified by a non-testing based
technique.  That can also be documented in this section.}

\subsubsection{Module 1}

\wss{Include a blurb here to explain why the subsections below cover the module.
  References to the MIS would be good.  You will want tests from a black box
  perspective and from a white box perspective.  Explain to the reader how the
tests were selected.}

\begin{enumerate}

  \item{test-id1\\}

    Type: \wss{Functional, Dynamic, Manual, Automatic, Static etc. Most will
    be automatic}

    Initial State:

    Input:

    Output: \wss{The expected result for the given inputs}

    Test Case Derivation: \wss{Justify the expected value given in
    the Output field}

    How test will be performed:

  \item{test-id2\\}

    Type: \wss{Functional, Dynamic, Manual, Automatic, Static etc. Most will
    be automatic}

    Initial State:

    Input:

    Output: \wss{The expected result for the given inputs}

    Test Case Derivation: \wss{Justify the expected value given in
    the Output field}

    How test will be performed:

  \item{...\\}

\end{enumerate}

\subsubsection{Module 2}

...

\subsection{Tests for Nonfunctional Requirements}

\wss{If there is a module that needs to be independently assessed for
  performance, those test cases can go here.  In some projects, planning for
nonfunctional tests of units will not be that relevant.}

\wss{These tests may involve collecting performance data from previously
mentioned functional tests.}

\subsubsection{Module ?}

\begin{enumerate}

  \item{test-id1\\}

    Type: \wss{Functional, Dynamic, Manual, Automatic, Static etc. Most will
    be automatic}

    Initial State:

    Input/Condition:

    Output/Result:

    How test will be performed:

  \item{test-id2\\}

    Type: Functional, Dynamic, Manual, Static etc.

    Initial State:

    Input:

    Output:

    How test will be performed:

\end{enumerate}

\subsubsection{Module ?}

...

\subsection{Traceability Between Test Cases and Modules}

\wss{Provide evidence that all of the modules have been considered.}

\bibliographystyle{plainnat}

\bibliography{../../refs/References}

\newpage

\section{Appendix}

This is where you can place additional information.

\subsection{Symbolic Parameters}

The definition of the test cases will call for SYMBOLIC\_CONSTANTS.
Their values are defined in this section for easy maintenance.

\subsection{Usability Survey Questions?}

\wss{This is a section that would be appropriate for some projects.}

\newpage{}
\section*{Appendix --- Reflection}

\wss{This section is not required for CAS 741}

The information in this section will be used to evaluate the team members on the
graduate attribute of Lifelong Learning.

The purpose of reflection questions is to give you a chance to assess your own
learning and that of your group as a whole, and to find ways to improve in the
future. Reflection is an important part of the learning process.  Reflection is
also an essential component of a successful software development process.  

Reflections are most interesting and useful when they're honest, even if the
stories they tell are imperfect. You will be marked based on your depth of
thought and analysis, and not based on the content of the reflections
themselves. Thus, for full marks we encourage you to answer openly and honestly
and to avoid simply writing ``what you think the evaluator wants to hear.''

Please answer the following questions.  Some questions can be answered on the
team level, but where appropriate, each team member should write their own
response:


\begin{enumerate}
  \item What went well while writing this deliverable?
  \item What pain points did you experience during this deliverable, and how
    did you resolve them?
  \item What knowledge and skills will the team collectively need to acquire to
    successfully complete the verification and validation of your project?
    Examples of possible knowledge and skills include dynamic testing knowledge,
    static testing knowledge, specific tool usage, Valgrind etc.  You
    should look to
    identify at least one item for each team member.
  \item For each of the knowledge areas and skills identified in the previous
    question, what are at least two approaches to acquiring the knowledge or
    mastering the skill?  Of the identified approaches, which will each team
    member pursue, and why did they make this choice?
\end{enumerate}

\end{document}
