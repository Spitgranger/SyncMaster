\documentclass[12pt, titlepage]{article}

\usepackage{booktabs}
\usepackage{tabularx}
\usepackage{amsmath}
\usepackage{hyperref}
\hypersetup{
  colorlinks,
  citecolor=blue,
  filecolor=black,
  linkcolor=red,
  urlcolor=blue
}
\usepackage[round]{natbib}
\usepackage{longtable}
\usepackage{enumitem,amssymb}
\usepackage[normalem]{ulem}
\newlist{todolist}{itemize}{2}
\setlist[todolist]{label=$\square$}

%% Comments

\usepackage{color}

\newif\ifcomments\commentstrue %displays comments
%\newif\ifcomments\commentsfalse %so that comments do not display

\ifcomments
\newcommand{\authornote}[3]{\textcolor{#1}{[#3 ---#2]}}
\newcommand{\todo}[1]{\textcolor{red}{[TODO: #1]}}
\else
\newcommand{\authornote}[3]{}
\newcommand{\todo}[1]{}
\fi

\newcommand{\wss}[1]{\authornote{blue}{SS}{#1}} 
\newcommand{\plt}[1]{\authornote{magenta}{TPLT}{#1}} %For explanation of the template
\newcommand{\an}[1]{\authornote{cyan}{Author}{#1}}

%% Common Parts

\newcommand{\progname}{ProgName} % PUT YOUR PROGRAM NAME HERE
\newcommand{\authname}{Team \#, Team Name
\\ Student 1 name
\\ Student 2 name
\\ Student 3 name
\\ Student 4 name} % AUTHOR NAMES                  

\usepackage{hyperref}
    \hypersetup{colorlinks=true, linkcolor=blue, citecolor=blue, filecolor=blue,
                urlcolor=blue, unicode=false}
    \urlstyle{same}
                                


\setcitestyle{numbers}

\begin{document}

\title{System Verification and Validation Plan for \progname{}}
\author{\authname}
\date{\today}

\maketitle

\pagenumbering{roman}

\section*{Revision History}

\begin{tabularx}{\textwidth}{p{3cm}p{2cm}X}
  \toprule {\bf Date} & {\bf Version} & {\bf Notes}\\
  \midrule
  11/4/24 & 1.0 & Initial Draft of VnV Plan for Rev0\\
  03/04/25 & 1.1 & Modifications to Plan for Rev1\\
  \bottomrule
\end{tabularx}

~\\

\newpage

\tableofcontents

\listoftables

\section*{Symbols, Abbreviations, and Acronyms}

Refer to \textit{Section 4 Naming Conventions and Terminology} in the
SRS document \href{https://github.com/Spitgranger/SyncMaster/blob/main/docs/SRS-Volere/SRS.pdf}{SRS.pdf} 
for all relevant symbols, abbreviations, and acronyms.

\newpage

\pagenumbering{arabic}

\section{Overview}

This document outlines the Verification and Validation plan which will be used
to ensure the SyncMaster application meets the requirements specification
and the City of Hamilton's acceptance. The verification plan is specified
in detail outlining what methods will be used to verify the
functional and non-functional
requirements. The system tests to support this process are specified in detail.
The validation plan to ensure stakeholder acceptance is further identified.

\section{General Information}

\subsection{Summary}

The software being tested is `SyncMaster', an application being
developed for the City of Hamilton, Water Division. The general
functions of the application are:

\begin{itemize}
  \item Document Management: The application will have a document
    hosting functionality, enabling effetive document management and a single source of truth.
    It will allow contractors and employees to view all relevant station documentation.
    It allows contractor users to acknowledge they have read required documents.
  \item Site Visits: The application will have a portal
    accessible to contractors attending site to complete work. The portal will allow users to acknowledge they have
    completed the necessary health and safety training and are aware
    of any hazards. The portal will also collect
    information from the user regarding the purpose of their site visit and allow the contractor to view
    the station documents.
  \item Displaying and Exporting Data: Data is collected through the
    portal including arrival time of contractors, the time they sign out,
    and any files uploaded during their visit. This data is
    available to be viewed and exported by users with the correct
    level of access.
  \item Permission Based Access: Users registered on the system will
    have different access to features depending the permissions given
    to their account type.
  \item Geolocation Verification: The application will be able to
    verify if a user is at the appropriate site using GPS.
  \item Notifications: The application will be able to indicate to admin users
    when documents or and contractor annual Health and Safety Training is expired.
\end{itemize}

\subsection{Objectives}

The objectives of this document are to be able to build confidence in
the correctness of the application being built, and to demonstrate
adequate usability of the application by performing the tests
outlined in this document.

It is out of scope for the verification and validation plans outlined
in this document to test external libraries. Instead it is assumed
that the implementation team for the external library has already
validated the library.

\subsection{Challenge Level and Extras}

The challenge level for this project is a general level, and the
extras are conducting user testing and developing a user manual. For
more information refer to the Problem Statement and Goals
\citep[\textit{Challenge Level and Extras}]{ProblemStatement}.

\subsection{Relevant Documentation}

\begin{enumerate}
  \item Problem Statement and Goals \citep{ProblemStatement}:
    Provides context on the goals and scope of the application.
  \item Software Requirements Specification \Citep{SRS}:
    Specification of all requirements being tested and validated in
    this document.
  \item Hazard Analysis \Citep{HazardAnalysis}: Additional
    specifications for security and safety requirements.
  \item Design Documents \Citep{MG, MIS}: Reference for the
    mechanisms and design behind functionality and systems being tested.
\end{enumerate}

\newpage
\section{Plan}

This section outlines our plans for verifying and validating the
software under development, along with its accompanying
documentation. It provides an overview of the validation team and
details the approach to verifying the software's documentation,
design, and implementation, as well as validating the final product.

\subsection{Verification and Validation Team}

The verification and validation team will consist of members from the
development team as well as members from the project stakeholders. A summary of
the members and their roles are given in the table below.
\setlength{\arrayrulewidth}{0.5mm}
\setlength{\tabcolsep}{18pt}
\renewcommand{\arraystretch}{1.5}\\
\begin{table}
  \begin{tabular}{ | m{5cm} | m{9cm} | }
    \hline
    \textbf{Names and Role} & \textbf{Responsibilites} \\
    \hline
    \textbf{Mitchell Hynes, Kyle D'Souza, Richard Fan, Akshit Gulia, Rafeed
    Iqbal} - Development Team &
    \begin{itemize}
      \item Responsible for verification of implementation details and
        adherance of the system to the SRS. The development team will own the
        most of the verification activities pertaining to the software. This
        includes activities such as developing verification plans, code reviews,
        unit testing, integration testing, and SRS verification.

    \end{itemize}\\
    \hline
    \textbf{Matthew Yakymyshyn} - Stakeholder from the City of Hamilton &
    \begin{itemize}
      \item Responsible for the verification of the system. They will be
        responsible for ensuring, and determining to what degree the
        system fulfils its intended purpose.
        Some examples of the activities that they would be responsible for
        would be SRS validation and acceptance testing.

    \end{itemize}\\
    \hline
    \textbf{Tarnveer Takhtar, Matthew Bradbury, Harman Bassi, Kyen So}
    - Peer Reviewers &
    \begin{itemize}
      \item Responsible for critiquing and providing suggestions for artifacts
        produced. This would include reviewing design documents to ensure that
        they are complete and unambiguous. They will provide opinions and
        insights as third parties who are not directly involved in the project.
    \end{itemize}\\
    \hline
    \textbf{Spencer Smith, Yiding Li} - Teaching Team for 4G06 &
    \begin{itemize}
      \item They will also be responsible for the final
        evaluation of the system and will determine if the system meets the
        initially specified functional and non-functional requirements.
        They will also play a role in providing feedback on produced artifacts
        throughout the project.
    \end{itemize}\\
    \hline

  \end{tabular}
  \caption{Testing Team}
\end{table}
\newpage
\subsection{SRS Verification Plan}

To verify the SRS document, we will use the following methods:
\begin{enumerate}
  \item Formal reviews with our TA, Yiding. A checklist will be used to track
    the status of the review and be used as an instrument to verify our SRS.
    The checklist to be used is at the end of this section of the document.
  \item Ad-hoc peer reviews tracked through GitHub issues. Throughout the
    course we receive peer feedback on the quality of our SRS and VnV Plan. The
    peer feedback provided is valuable input which we will action through issues
    and pull requests as identified in the development plan.
  \item Milestone feedback provided through Avenue will also be used to
    improve the checklist which we use as our instrument to measure
    SRS verification.
\end{enumerate}
The checklist below will be used during formal reviews for SRS verification
with our TA.
\begin{itemize}
  \item Have the requirements been listed in the appropriate sections of the
    SRS document?
  \item Are all definitions and acronyms defined in the glossary of terms?
  \item Are the system inputs properly specified?
  \item Are the system outputs properly specified?
  \item Do the functional requirements avoid conflicts with other requirements?
  \item Do the functional requirements avoid specifying the system design?
  \item Are the requirements clear enough that they could be implemented by an
    independent engineering team correctly?
  \item Is each requirement testable?
  \item Is independent testing of each requirement possible?
  \item Are the functional requirements traceable to a use case of the system?
  \item Are there any open issues from reviewers on a requirement?
\end{itemize}

\subsection{Design Verification Plan}

% \wss{Plans for design verification}

% \wss{The review will include reviews by your classmates}

% \wss{Create a checklists?}

To verify the design, we will use the following methods:
\begin{enumerate}
\item Ensure that each module specified in section 8 of \href{https://github.com/Spitgranger/SyncMaster/blob/main/docs/Design/SoftArchitecture/MG.pdf}{MG.pdf}
satisfies its corresponding requirement outlined in the table. This will be completed as a check in our team meeting on Mar 6, 2025.
\item Ensure all outstanding GitHub issues connected to the Design are addressed and closed.
\end{enumerate}

\subsection{Verification and Validation Plan Verification Plan}

To ensure the quality and completeness of the verification and
validation plan, the testing team will ensure the following:

\begin{enumerate}
  \item There will be peer reviews of the document coming from our
    assigned Peer Review team, to provide feedback on the plan.
  \item The TA assigned to the project will review and provide
    feedback on the plan when grading.
  \item The development team will review the document to ensure the
    quality of the plan.
  \item It should be checked by the testing team that each
    requirement has a test case associated with it. This can be done
    using traceability matrices and cross-referencing requirements
    with what exists in the SRS.
\end{enumerate}

The following is a checklist that can be used to verify the
verification and validation plan:
\begin{todolist}
\item Ensure all issues on the project GitHub related to the document
  from the Peer Review team is closed.
\item Ensure that all feedback provided by the TA in the rubric for
  the document on Avenue is addressed.
\item Ensure the Development team has reviewed the document and is in
  agreement that the document is up to quality and completeness standards.
\item Check the traceability matrices to ensure that all requirements
  have a test case associated with them.
\end{todolist}

\subsection{Implementation Verification Plan}

\begin{itemize}
  \item To ensure proper implementation of application functionality,
    unit tests will be conducted on all functions and modules as
    specified in this document.
  \item Continuous Integration (CI) tools will be utilized to
    automatically test all code before deployment, verifying code
    integrity and identifying issues early in the development process.
  \item All pull requests will undergo a detailed review process to
    ensure code quality, adherence to standards, and functional
    accuracy before merging into the main codebase.
  \item Following each revision, a team code walkthrough will be
    conducted to review changes collectively, fostering knowledge
    sharing and alignment on implementation details.
\end{itemize}

\subsection{Automated Testing and Verification Tools}
\begin{itemize}
  \item Python unit testing: PyTest
  \item JavaScript/TypeScript unit testing: Jest
\end{itemize}
Detailed information about automated testing and verification tools
that we plan to use are explained in the Development Plan
\citep[\textit{Expected Technology, Coding Standard}]{ProblemStatement}.
In terms of quality metrics, we plan on using GitHub actions to
generate a report
every time a pull request is made against the main branch. The report
will be generated from the unit testing framework selected, more
specifically jest-html-reporter and pytest-cov. Initially, these tests
will include unit tests with more types of automated testing to be determined as
the project progresses.

\subsection{Software Validation Plan}

\begin{enumerate}
  \item Rev0 prototype demonstration to Technical Services staff. The rev0
    prototype will be demonstrated to the City Staff on this team at
    an in-person
    meeting. Each requirement from the SRS will be demonstrated in
    the demo so the
    full scope of the application is displayed. The demonstration will also show
    how it satisfies the use cases identified in the SRS. Feedback
    received from the
    City at this meeting will be used to improve the prototype for rev1.

  \item Stakeholder progress check-ins. Short periodic meetings will be arranged
    with the Technical Services team as required to demonstrate the
    user interface
    design prototypes and receive feedback from the facilities
    managers. Email updates
    will also be sent for items which require stakeholder
    clarification. The user interface is the most important part of
    the application to validate with the end-users.
    Consistent communication on the direction of the interface design
    to ensure it
    is usable will catch problems early and greatly improve the quality and
    acceptance of the application.
\end{enumerate}

\section{System Tests}

This section outlines tests to determine if requirements from the SRS
are satisfied.

\subsection{Tests for Functional Requirements}

Below are tests for the functional requirements.

\subsubsection{File Handling}

\begin{enumerate}

  \item{TC-FR1\\}

    Control: Manual

    Initial State: The system is running and user is logged in

    Input: A file (can have any extension)

    Output: File should be uploaded successfully and visible in the folder/path
    it was uploded under with the following details:
    \begin{enumerate}
      \item File Name
      \item Date Modified
      \item Type
      \item Size
      \item Owner
    \end{enumerate}

    Test Case Derivation: The system is expected to support the upload of files
    regardless of their file types and also show the details mentioned above
    once they are successfully uploaded.

    How test will be performed: Files with common file types
    (.pdf, .docx, .xlsx, .txt, .png, .jpeg, .jpg, .zip and .rar
    will be used for testing) and uncommon file types (.rb, .rda, .bin and .vdi
    will be used for testing) will be manually uploaded. Afterwards, their
    visibility on the portal will be verified.

  \item{TC-FR2\\}

    Control: Manual

    Initial State: The system is running, and the user is logged in

    Input: A file present in the portal (i.e., an uploaded file)

    Output: The user should able to download the file in the portal if it is one
    of the following types:
    \begin{enumerate}
      \item .doc
      \item .docx
      \item .xls
      \item .xlsx
      \item .ppt
      \item .pptx
      \item .pdf
      \item .csv
      \item .txt
    \end{enumerate}

    Test Case Derivation: The above mentioned types are commonly used by the
    stakeholders in day-to-day operations and therefore, they should be able to
    download them. Less common file types (ones not mentioned above) are not used by
    our stakeholders. Moreover, file types like .xlsm (macro-enabled excel
    workbook) can be malcious and thus, can compromise the safety of the system
    (malcious macro present in a macro-enabled excel workbook executes when file
    is opened/viewed). Therefore, they are not viewable on the portal.

    How test will be performed: One file of each of the above file types  will be selected and downloaded.

\end{enumerate}

\subsubsection{Access Control}

\begin{enumerate}
  \item {TC-FR3\\}

    Control: Manual

    Initial State: The system is running and different user roles of Admin,
    Contractor, and Employee user exist.

    Input: User Role (Admin, Contractor, Employee user)

    Output: Admin users should be able to change permissions and access all
    documents. Contractor users should only have read access to the contractor portal and the
    ability to acknowledge documents.
    Test Case Derivation: The system should enforce different access levels as
    specified in the functional requirements in the SRS document.

    How test will be performed: Manually log in with different users roles (
    as mentioned above) and verify the access permissions and capabilities.

\end{enumerate}

\subsubsection{User Information Visibility}

\begin{enumerate}
  \item {TC-FR4\\}

    Control: Manual

    Initial State: The system is running and the number of users in the system
    is greater than or equal to 2 and there is atleast one admin user and
    atleast one contractor user.

    Input: A valid user id

    Output: Admin users should be able to view all the details of the other user
    with the corresponding user id provided.

    Test Case Derivation: Admin users need to access contractor information to
    manage site visit details. In some cases, admin users will also
    need to access details of other admin users.

    How test will be performed: Manually log in as an admin user and verify the
    visibility of the user details based on the user id provided.

\end{enumerate}

\subsubsection{Geolocation Verification}

\begin{enumerate}
  \item {TC-FR5\\}

    Control: Manual

    Initial State: The system is running and the location service is enabled
    on the contractor's device

    Input: Contractor user's location

    Output: Authentication and acknowledging of the document should be
    allowed if within the allowed geolocation.

    Test Case Derivation: To ensure presence of contractors on the site, the
    should restrict the actions mentioned above if they are not within premises
    of the plant.

    How test will be performed: Manually attempt to authenticate and sign/upload
    documents from different locations and verify the results.

\end{enumerate}

\subsubsection{Notifications}

\begin{enumerate}
  \item {TC-FR6\\}

    Control: Manual

    Initial State: The system is running, and a document with an expiry date is present
    in the system

    Input: Expired document

    Output: The system should indicate to admin users that a document has expired.

    Test Case Derivation: Notifications for expired documents ensure documents are
    compliant with any regulations and up-to-date.

    How test will be performed: Manually expire a document and verify
    that the system indicates this to admin users on the admin portal.

  \item {TC-FR7\\}

    Control: Manual

    Initial State: The system is running, and there are contractor users who
    have not completed their annual health and safety training.

    Input: Contractor user without completed training.

    Output: The system should indicate to admin users when a contractors annual health and safety training is expired.

    Test Case Derivation: Admin users need to be aware of contractor's training
    status to ensure compliance.

    How test will be performed: Manually create a contractor user with an expired training date and observe that the system
    indicates this to admin users on the admin portal.

\end{enumerate}

\subsubsection{Document Acknowledgement}

\begin{enumerate}
  \item {TC-FR8\\}

    Control: Manual

    Initial State: The system is running, and documents are available for
    acknowledgement.

    Input: Acknowledgement of documents during site visit.

    Output: The should store the date, time, and name of the
    user who acknowledged the documents during the site visit.

    Test Case Derivation: Accurate record-keeping is required for
    accountability and traceability.
    How test will be performed: Manually acknowledge documents and verify that
    the system stores the correct information.

\end{enumerate}

\subsubsection{Training Verification}

\begin{enumerate}
  \item {TC-FR9\\}

    Control: Manual

    Initial State: The system is running, and a contractor user has not completed their annual health and safety training

    Input: A contractor user that has not completed the training accesses portal for a site visit
    without an employee accompanying them. They acknowledge this during sign in.

    Output: The system notifies contractor they are not authorized to access the site and should contact their facilities manager.

    Test Case Derivation: Contractors are required to complete training before
    they can access site, or else must be accompanied by an employee.

    How test will be performed: Manually try to authenticate
    as a contractor user without completing required training or an emplyee during authentication process and verify that
    the system prevents it.

\end{enumerate}

\subsection{Tests for Nonfunctional Requirements}

Below are tests for the nonfunctional requirements.

\subsubsection{Usability and Humanity Requirements}

\paragraph{Ease of Use}

\begin{enumerate}

  \item{TC-EU-1\\}

    Type: Dynamic, Manual

    Initial State: System is functional with test accounts made for users. Users
    are between MIN\_AGE and MAX\_AGE age limits. No prior instructions or
    training are provided.

    Input/Condition: User who has no prior experience using the system attempts
    to use it.

    Output/Result: Users are able to discover at least 70\% of the main
    functionalities of the system within 10 minutes without assitance.

    How test will be performed: The test will be performed by letting users from
    our stakeholder who have no prior knowledge of the system use the system.
    User must successfully understand the difference between being a contractor and admin user 
    and the respective authentication and document management systems.\\

  \item{TC-EU-2\\}

    Type: Dynamic, Manual

    Initial State: System is functional with test accounts made for users. Users
    are between MIN\_AGE and MAX\_AGE age limits.

    Input: Users logs in, and performs an action classified as irreversible
    (e.g. deletion).

    Output: The user sees a notification informing them that their action is
    irreversible or an undo option.

    How test will be performed: The test will be performed by a user from our
    stakeholder. The user will be instructed to perform an irrversible action on
    the system. After performing the action and viewing the prompt, the user
    will be asked if they understood the implications of their actions. If the
    user understands the prompt and actions correctly, the test is considered
    successful.

\end{enumerate}

\paragraph{Learning}
\begin{enumerate}
  \item{TC-LR-1\\}

    Type: Dynamic, Manual

    Initial State: System is functional with test accounts made for users. Users
    are between MIN\_AGE and MAX\_AGE age limits.

    Input: User logs in, and uses a feature that they have not used
    before (e.g. viewing documents)

    Output: User has explored the new feature and is confident in using the
    feature.

    How test will be performed: The test will be performed by a user from our
    stakeholder. The user will be instructed to use a feature of the system that
    they have not seem before. After 10 minutes, the user will be asked to rate
    their understanding of the feature and how confident they are using it. The
    questions will be a simple 1-10 scale. If the user selects a 7 or above, the
    test is considered passed. See \textbf{Section 6.2} Usability
    Survey Questions.

  \item{TC-LR-2\\}

    Type: Dynamic, Manual

    Initial State: System onboarding documents are avaliable. Users
    are between MIN\_AGE and MAX\_AGE age limits.

    Input: User views the system onboarding documents.

    Output: User has finished viewing the system onboarding documents and
    reports their confidence on how they feel about using the system.

    How test will be performed: The test will be performed by a user from our
    stakeholder. They will be asked to view the onboarding documents. After 5
    minutes the user will be asked a few questions to gauge their
    confidence when using the
    system. See \textbf{Section 6.2} Usability Survey Questions.

\end{enumerate}

\paragraph{Understandability and Politeness}
\begin{enumerate}
  \item{TC-UP-1\\}

    Type: Dynamic, Manual

    Initial State: System is functional with test accounts made for users. Users
    are between MIN\_AGE and MAX\_AGE age limits.

    Input: User logs in, and views all available screens of the user interface.

    Output: The user has seen the images and text on all available screens of
    the user interface and notes the content seen.

    How test will be performed: The test will be performed by a user from our
    stakeholder. They will be guided through all screens for every type of role
    (General/Contractor, Manager, Admin). They will then be asked if any of the
    images or text seen contains anything that they might have found offensive
    or politically charged. If the user reports they haven't seen anything that
    may have been offensive or poitically charged, the test is successful.

  \item{TC-UP-2\\}

    Type: Dynamic, Manual

    Initial State: System is functional with test accounts made for users. Users
    are between MIN\_AGE and MAX\_AGE age limits.

    Input: User navigates through the system as either a contractor or admin user.

    Output: User has performed the actions and the system has informed the user
    of the error. User actions the feedback given and understands the potential
    resolution paths.

    How test will be performed: The test will be performed by a user from our
    stakeholder. The user will be instructed to freely explore the menus in the portal and see if an error is encountered.
    If an error is encountered, The user will then be asked if they understood the error
    and if they were able to understand possible solutions.
    If no errors are encountered, the test is successful.

\end{enumerate}

\paragraph{Accessibility}
\begin{enumerate}
  \item{TC-AS-1\\}

    Type: Dynamic, Manual

    Initial State: System is functional with test accounts made for users. Users
    are between MIN\_AGE and MAX\_AGE age limits.

    Input: User logs in, and views all available screens of the user interface.

    Output: The user seen all available screens of
    the user interface and notes the content seen.

    How test will be performed: The test will be performed by a user from our
    stakeholder. They will be guided through all screens for every type of role
    (contractor, ddmin, employee). They will then be asked if the design
    of the user interface aligns with accessibility features that they have
    seen in other applications used by the city. If the user reports that the
    accessibility features seen are similar to other applications that
    they already use, the test is successful.

\end{enumerate}

\subsubsection{Look and Feel Requirements}

\paragraph{Appearance}

\begin{enumerate}

  \item{TC-LF-1\\}

    Type: Manual

    Initial State: The system is in an operational state, with all
    components running

    Input/Condition: User logs in as any possible user and views all available screens of
    the user interface.

    Output/Result: The user has seen all available screens of the
    user interface and feels that most of the colours used by the
    application are aesthetically pleasing and pleasant.

    How test will be performed: The test will be performed by
    a user acting as each of the user roles (contractor, admin, employee). A member of the
    development team will guide them to all available screens for their role and
    ask them to evaluate how aesthetically pleasing the colour palette is.
    If they report the colour palette is pleasing then the test is successful, otherwise it is failed.

  \item{TC-LF-2\\}

    Type: Dynamic, Manual.

    Initial State: The system is in an operational state, with all
    components running

    Input/Condition: Load and navigate through the application on
    various supported operating systems, with varying supported
    browsers and screen sizes.

    Output/Result: User is able to interact with the system on all screen
    sizes and understand the interface regardless of screen size

    How test will be performed: One user for each type of role
    (contractor, admin, employee) will ensure that 
    functionalities work as expected across screen sizes, browsers,
    and operating systems. The tests should be performed using
    Windows 10, IOS, and Android, which are the supported operating
    systems for this application. The browsers used should be
    Microsoft Edge and Google Chrome. Testing should include a mobile
    device and a desktop to ensure accommodation for
    different screen sizes and device types.

    The core functionalities for a contractor user will be
    signing into the system, viewing a document, acknowledging a document,
    and logging out. The core functionalities of an admin will be
    uploading a document, viewing document expiry status, viewing site visit logs, creating users, deleting users,
     and assigning roles to users. The core
    functionalities of an employee user will be viewing documents.

\end{enumerate}

\subsubsection{Performance Requirements}

\paragraph{Speed and Latency Requirements}

\begin{enumerate}
  \item {TC-PR-1\\}

    Type: Manual

    Initial State: The system is running, and documents are available in the
    database.

    Input/Condition: Request to retrieve a document.

    Output/Result: The document should be retrieved in less than 2 seconds

    How test will be performed: Manually request document retrieval and measure
    time taken. Verify that it does not exceed 2 seconds

  \item {TC-PR-2\\}

    Type: Manual

    Initial State: The system is running

    Input/Condition: Navigation to a different page

    Output/Result: The system should be able to navigate to a different page
    in under 1.5 seconds.

    How test will be performed: Manually navigate to a different page and
    measure time taken. Verify that it does not exceed 1.5 seconds

\end{enumerate}

\paragraph{Safety-Critical Requirements}
\begin{enumerate}

  \item {TC-PR-3\\}

    Type: Manual

    Initial State: The system is running and ready to upload and retrieve
    documents.

    Input/Condition: User uploads and downloads a document

    Output/Result: The downloaded document should look identical to the document which was uploaded

    How test will be performed: User manually uploads and then downloads a document. If the downloaded document looks
    identical to the uploaded document, then the test is successful.

  \item {TC-PR-4\\}

    Type: Manual

    Initial State: The system has different user roles configured (contractor, admin, employee)

    Input/Condition: Attempt to access documents with different user roles.

    Output/Result: Document access should be restricted based on user roles.
    Admin users should have the ability to read and write documents while
    contractor and employee users should only have the abilty to read documents.

    How test will be performed: Manually log in with each user role and
    attempt to access documents. Verify that access is correctly restricted.

\end{enumerate}

\paragraph{Precision or Accuracy Requirements}
\begin{enumerate}

  \item \sout{{TC-PR-5\\}

    Type: Manual

    Initial State: The system is running with a populated database with number
    of users and documents greater than equal to 1.

    Input/Condition: Perform search queries related to documents or user data

    Output/Result: At least 95\% of the search results should be relevant to
    search query

    How test will be performed: Manually perform search queries and evaluate the
    relevance of the results} \\
    \textcolor{red}{Search queries will no longer be a feature of the application}.

  \item \sout{{TC-PR-6\\}

    Type: Manual

    Initial State: The system is running and ready for document upload and
    retrieval

    Input/Condition: Upload and retrieve documents

    Output/Result: No corruption or data loss in 99.99\% of the cases

    How test will be performed: Manually upload and retrieve docuements. Then,
    verify their integrity.}\\
    \textcolor{red}{Removing due to assumption cloud services are fully functional and secure}.

\end{enumerate}

\paragraph{Robustness or Fault-Tolerance Requirements}
\begin{enumerate}

  \item {TC-PR-7\\}

    Type: Manual

    Initial State: The system is running

    Input/Condition: Provide unexpected inputs and events

    Output/Result: The system should handle them without crashing

    How test will be performed: Manually input unexpected data (such as files named with special characters) and observe the
    system's response. The test passes if the system does not crash.

  \item \sout{{TC-PR-8\\}

    Type: Manual

    Initial State: The system is running

    Input/Condition: Introduce anomalies or potential failures

    Output/Result: The system should detect and alert on these issues

    How test will be performed: Manually introduce anomalies and verify the
    system's detection and alert mechanism.}\\
    \textcolor{red}{Removing as corresponding requirement no longer exists.}

\end{enumerate}

\paragraph{Capacity Requirements}
\begin{enumerate}

  \item \sout{{TC-PR-9\\}

    Type: Manual

    Initial State: The system is running

    Input/Condition: Upload and download documents

    Output/Result: The system should support a peak data transfer rate of 1
    Gbps.

    How test will be performed: Manually upload and download documents and
    measure the data transfer rate.}\\
    \textcolor{red}{Removing as this is dependent on the internet speed of the user which we can not control}.

  \item {TC-PR-10\\}

    Type: Manual

    Initial State: The system is running

    Input/Condition: Attempt to upload individual files up to 1 GB.

    Output/Result: The files should be successfully uploaded

    How test will be performed: Manually upload files of varying sizes up to
    1 GB and verify successfull uploads.

\end{enumerate}

\paragraph{Scalability or Extensibility Requirements}
\begin{enumerate}
  \item {TC-PR-11\\}

    Type: Manual

    Initial State: The system is running

    Input/Condition: Create nested document categories with multiple
    sub-categories.

    Output/Result: The system should allow nesting of documents and categories.

    How test will be performed: Manually create nested document categories and
    verify their functionality.

  \item \sout{{TC-PR-12\\}

    Type: Automatic

    Initial State: The system is running under normal conditions

    Input/Condition: Simulate high traffic to the system.

    Output/Result: The system should distribute traffic evenly and avoid single
    points of failure.

    How test will be performed: Simulate high-traffic using a script to generate
    users and perform the task of uploading a document. Verify the system's load
    balancing capabilities.}\\
    \textcolor{red}{Removing due to elimination of corresponding requirement}.

\end{enumerate}

\paragraph{Scalability or Extensibility Requirements}
\begin{enumerate}

  \item \sout{{TC-PR-13}
    Type: Automatic

    Initial State: The system is running

    Input/Condition: Add new feature into the system

    Output/Result: The system is functional with the new functionalites and the
    behaviour of the old functionalities is not changed (the ones which
    were not intended to be impacted by the introduction of the new feature).
    The ease of introduction should atleast be 8.

    How test will be performed: Add code for the new feature and
    measure the ease
    of introduction of the feature on a scale of 1-10. Verify that the ease of
    introduction is greater or equal to 8.}\\
    \textcolor{red}{Removing due to elimination of corresponding requirement}.
\end{enumerate}

\subsubsection{Maintainability and Support Requirements}

\paragraph{Maintenance}

\begin{enumerate}

  \item \sout{{TC-MS-1\\}

    Type: Manual

    Initial State: Cloud platform used for deployments is functional,
    GitHub actions is functional.

    Input/Condition: A workflow run in GitHub actions is triggered to
    deploy the application into a specified environment, and
    building/automated unit testing of the application is completed

    Output/Result: The application is deployed to the specified
    environment in under 30 minutes

    How test will be performed: The test will be performed by
    triggering a GitHub actions workflow run to deploy the application
    into the development environment, and waiting for it to complete.
    Once completed, the unit testing and building time will be subtracted
    from the total runtime of the workflow run, to determine the
    deployment time, this should be under 30 minutes to be considered a success.}\\
    \textcolor{red}{Removing due to elimination of corresponding requirement}.

  \item \sout{{TC-MS-2\\}

    Type: Manual

    Initial State: GitHub actions is functional.

    Input/Condition: A workflow run in GitHub actions is triggered to
    deploy the application into a specified environment, and
    automated unit testing of the application is completed.

    Output/Result: The application is built in under 10 minutes.

    How test will be performed: The test will be performed by
    triggering a GitHub actions workflow run to deploy the application
    into the development environment, and waiting for it to complete.
    Once completed, the unit testing and deployment time will be subtracted
    from the total runtime of the workflow run, to determine the
    build time, this should be under 10 minutes to be considered a success.}\\
    \textcolor{red}{Removing due to elimination of corresponding requirement}.

  \item \sout{{TC-MS-3\\}

    Type: Manual

    Initial State: GitHub actions is functional.

    Input/Condition: A workflow run in GitHub actions is triggered to
    perform all automated testing (end-to-end or unit testing) of the
    application.

    Output/Result: The automated tests for the application run in
    under 10 minutes.

    How test will be performed: The test will be performed by
    triggering a GitHub actions workflow run to perform all automated
    tests, and checking that once this run is completed, the total
    runtime is under 10 minutes.}
    \textcolor{red}{Removing due to elimination of corresponding requirement}.

  \item{TC-MS-4\\}

    Type: Manual

    Initial State: GitHub actions is functional.

    Input/Condition: A push is made to a branch with a pull request
    open off of it.

    Output/Result: The branch is found to have line coverage $\ge$ 95\% and
    the branch coverage is found to be $\ge$ 90\%

    How test will be performed: The test will be performed by
    triggering a GitHub actions workfow run to run unit tests on the
    application upon every push to a branch with a PR open off of it.
    The workflow will generate a coverage report, and when the
    coverage report is generated, it should be checked that the line
    coverage and branch coverage meet the expectations.

  \item{TC-MS-5\\}

    Type: Manual

    Initial State: GitHub is functional

    Input/Condition: The GitHub repository for the project is checked
    to ensure that all functional requirements listed in the SRS have
    a unit test corresponding to them.

    Output/Result: All functional requirements found in the SRS have a 
    unit test corresponding to them

    How test will be performed: The test will be performed by
    checking the Traceability Matrices in this document, to make sure
    that there exists a unit test for all functional requirements.

  \item{TC-MS-6\\}

    Type: Manual

    Initial State: GitHub is functional

    Input/Condition: The GitHub repository for the project is checked
    to ensure that all appropriate documentation existed for users to
    be able to maintain the system.

    Output/Result: Instructions are provided in the GitHub repository
    for the project on how users can continue to maintain the system.
    This includes contribution guidelines, descriptions of all
    GitHub actions workflows, and documentation on the design of the system.

    How test will be performed: The test will be performed by
    checking that the documentation listed in the output/result
    exists in the repository, and by verifying with the Matthew
    Yakymyshyn that this documentation is able to be understood by
    the city of Hamilton.

  \item{TC-MS-7\\}

    Type: Manual

    Initial State: GitHub is functional

    Input/Condition: The GitHub repository for the project is checked
    to see the user manual.

    Output/Result: There exists a user manual in the github
    repository which describes, at a minimum, how to leverage the
    functionalities described in the functional requirements of the
    SRS from the user interface.

    How test will be performed: The test will be performed by
    checking that the user manual exists, and that there is
    user-facing documentation in the manual on how to achieve all
    functionalities described in the functional requirements.

  \item{TC-MS-8\\}

    Type: Manual

    Initial State: GitHub is functional

    Input/Condition: The GitHub repository for the project is checked
    to see the API documentation.

    Output/Result: There exists API documentation in the github
    repository which follows the OpenAPI Specification (OAS) standard.

    How test will be performed: The test will be performed by
    checking that an OpenAPI Specification for the API's provided by
    the system exists on the project repository.

  \item{TC-MS-9\\}

    Type: Manual

    Initial State: GitHub is functional

    Input/Condition: The GitHub repository for the project is checked
    to see the the internal documentation.

    Output/Result: There exists documentation on all internal
    functions and classes defined in the project repository.

    How test will be performed: The test will be performed by
    checking that documentation exists for all functions and classes
    defined in the project repository.

  \item{TC-MS-10\\}

    Type: Manual

    Initial State: GitHub is functional, Cloud deployment platform is functional

    Input/Condition: The GitHub repository for the project is checked
    to see the the deployment documentation

    Output/Result: There exists documentation on how to deploy the
    project, which is up-to-date and works when attempted.

    How test will be performed: The test will be performed by
    checking that the deployment documentation exists and that a
    member of our development team can follow the provided steps to
    successfully deploy the application.

\end{enumerate}

\subsubsection{Compliance Requirements}

\paragraph{Compliance}

\begin{enumerate}

  \item \sout{{TC-L-1\\}

    Type: Manual

    Initial State: Github is functional, cloud deployment platform is functional

    Input/Condition: A contractor uses the authentication feature of
    the application
    to verify their presence on site for the facilities managers.

    Output/Result: The contractor successfully authenticates in the application
    and successfully complies with each step of the entry/exit
    procedure document
    for the station provided through the application.

    How the test will be performed: The contractor user will authenticate at the
    site for a work order to fix an exhaust fan. Through the
    authentication process,
    the entry/exit procedure will be used as a checklist to ensure it
    is followed
    and the legislation specified in CR-L1 is not violated.}
    \textcolor{red}{Removing as corresponding requirement has been eliminated}.

  \item \sout{{TC-L-2\\}

    Type: Manual

    Initial State: Github is functional, cloud deployment platform is functional

    Input/Condition: The admin account registers a contractor account with the
    contractors personal email address.

    Output/Result: The contractors account is successfully
    registered, and the email
    address is only possible to view either from the facility managers
    account, the admin
    account, and the contractors account.

    How the test will be performed: The facility manager will use the
    interface to
    register an account in the application. The application will then
    be opened in
    different views based on profile. These profiles will be the
    registered contractor,
    the admin user, the facilities manager, and a second contractor profile.
    The test is successful when the email address of the contractor is only
    visible in the profile of that contractor, the facilities manager,
    and the admin user.}\\
    \textcolor{red}{Removing as corresponding requirement has been eliminated}.

  \item \sout{{TC-L-3\\}

    Type: Manual

    Initial State: Github is functional, cloud deployment platform is functional

    Input/Condition: A facilities manager signs into the application and
    downloads a compliance document.

    Output/Result: This document is sent to the administrative team of the BCOS
    system who are able to record the document in their system.

    How the test will be performed: The facilities manager signs into the system
    and downloads a compliance document. The test is successful if this
    document is confirmed by the City it is in a form that is able to be
    uploaded to the BCOS system.}\\
    \textcolor{red}{Removing as corresponding requirement has been eliminated}.

\end{enumerate}

\subsubsection{Cultural Requirements}

\paragraph{Cultural}

\begin{enumerate}
  \item{TC-CR-1\\}

    Type: Manual

    Initial State: Github is functional, cloud deployment platform is functional

    Input/Condition: The Github repository code is checked and the code for the
    display of the user interface is opened.

    Output/Result: The user interface and user experience is aligned with the 5
    City of Hamilton cultural pillars. These are Collective Ownership,
    Steadfast Integrity, Courageous Change, Sensational Service, and
    Engaged and Empowered Employees.

    How the test will be performed: The capstone team will compare the qualities
    of the user interface application code with the 5 pillars and assess whether
    each pillar is met by the application. The test passes if the
    team determines
    that the pillars are satisfied.
\end{enumerate}

\subsubsection{Operational and Environmental Requirements}

\paragraph{Ability to Connect and Use}

\begin{enumerate}

  \item{TC-OE-1\\}

    Type: Manual

    Initial State: Device connected to internet.

    Input/Condition: Attempt to connect to the application.

    Output/Result: The device is successfully able to connect to the application.

    How test will be performed: The test is successful if a user is able to connect to the application through the internet.

  \item{TC-OE-2\\}

    Type: Manual

    Initial State: Devices connected to the internet and have the
    supported browsers available.

    Input: Desktop Browsers being tested are Chrome and Edge, and the
    mobile devices being tested are Android and iPhone.

    Output: The devices successfully connect to and are able to use
    the application.

    How test will be performed: Exploratory testing using the
    supported devices and browsers. The devices should be able to
    connect and use all functionality in the application, and provide
    a similar experience.

\end{enumerate}

\paragraph{Future Integration}

\begin{enumerate}

  \item \sout{{TC-OE-3\\}

    Type: Manual

    Initial State: Branch with functionality to fill in data fields
    when provided a valid work order number

    Input/Condition: Valid work order number

    Output/Result: Relevant data fields should be filled in by
    the application

    How test will be performed: Dummy API used which returns work
    order information given a work order number. Will test if the
    functionality in the branch is able to fill in the fields with
    data from the API, indicating it is viable to add this
    integration in the future.}\\
    \textcolor{red}{Removing due to elimination of corresponding requirement}.

\end{enumerate}

\paragraph{Release and Change Log}

\begin{enumerate}

  \item{TC-OE-4\\}

    Type: Manual

    Initial State: New revision

    Input/Condition: Revision released, change log produced

    Output/Result: Revision is released by planned date, change log
    details all changes.

    How test will be performed: Whenever a revision is released, will
    inspect the change log to check if all changes have been noted.

\end{enumerate}

\subsubsection{Safety and Security Requirements}

\paragraph{Account Permissions}

\begin{enumerate}

  \item{TC-SS-1\\}

    Type: Manual

    Initial State: User IDs of each possible access level created.

    Input/Condition: Attempt to use every feature available using each ID

    Output/Result: Users only have access to features allowed by
    their permissions.

    How test will be performed: Accounts will be created on the
    application for each available level of access. Using each
    account, the use of each feature of the application will be
    attempted. Wether an action was possible or not will be noted and
    compared to the permissions of the account.

  \item{TC-SS-2\\}

    Type: Manual

    Initial State: User IDs of each possible access level created.

    Input/Condition: Attempt to upload, delete, modify files in
    different directories.

    Output/Result: Users only have access to files and directories
    allowed by their permissions.

    How test will be performed: Accounts will be created on the
    application for each available level of access. Using each
    account, the creation, deletion and modification of of files with
    varying permission requirements will be attempted. The testing
    will be done to cover directories of varying permission
    requirements. Wether an action was possible or not will be noted
    and compared to the permissions of the account.

\end{enumerate}

\paragraph{Data Validation}

\begin{enumerate}

  \item \sout{{TC-SS-3\\}

    Type: Manual

    Initial State: Initial set of files uploaded to application

    Input/Condition: View files on the application

    Output/Result: Files on the application are the appropriate
    version available on SharePoint and/or MySDS

    How test will be performed: Files will be checked against their
    current versions. Users with appropriate permissions should have
    the ability to update the files if outdated.}
    \textcolor{red}{Removing due to elimination of corresponding requirement}.

  \item{TC-SS-4\\}

    Type: Manual

    Initial State: Data fields empty.

    Input/Condition: Submit incomplete, impossible, or malicious data
    through data fields

    Output/Result: Fail-safes trigger, noting the bad data and
    informing the administrator.

    How test will be performed: Attempt to submit data fields filled
    in with a set of entries which vary between incomplete,
    impossible, and malicious and check how the application handles the inputs.

\end{enumerate}

\paragraph{Encryption}

\begin{enumerate}

  \item{TC-SS-5\\}

    Type: Manual

    Initial State: The application is accessed via a web browser with
    tools for inspecting network requests.

    Input/Condition: Access and use the application using https:// in a browser

    Output/Result: Network requests displayed as HTTPS in the
    browser's developer tools, and the security certificates are
    valid and properly configured.

    How test will be performed: Connect to and use the application
    using https:// and ensure the connection is secured with SSL/TLS.
    Use the browser's developer tools to inspect the requests made by
    the site. Check that all requests are made over HTTPS and the
    lock icon/secure status is visible in the browser's address bar.
    Check if SSL certificate is valid and correctly configured.

\end{enumerate}

\paragraph{Adherance to Policies, Regulations, and Laws}

\begin{enumerate}

  \item \sout{{TC-SS-6\\}

    Type: Manual

    Initial State: The application is complete, checklist(s)
    outlining all Federal, Provincial, Municipal and City of
    Hamilton, Water Division rules, regulations and policies created.

    Input/Condition: Review the application's functionality, data
    handling, and security measures against checklist(s)

    Output/Result: The application should align with all defined
    requirements, policies, and regulations, showing full compliance
    with legal and departmental standards without any deviations.

    How test will be performed: Develop a comprehensive checklist
    that includes all applicable government regulations, standards,
    and internal policies. Cross-reference the application's design
    and operational documentation with the compliance checklist.
    Perform a manual inspection of the application's source code,
    configuration settings, and security protocols to ensure they
    meet the required standards. Test relevant features to confirm
    that they adhere to the compliance checklist.}
    \textcolor{red}{Removing due to elimination of corresponding requirement}.

\end{enumerate}

\paragraph{Adherance to Policies, Regulations, and Laws}

\begin{enumerate}

  \item{TC-SS-7\\}

    Type: Manual

    Initial State: The application is being actively maintained.

    Input/Condition: A security vulnerability or weakness is discovered.

    Output/Result: Pull requests created by Dependabot are merged within a week.

    How test will be performed: Inspection of whether pull requests created by Dependabot are merged within a week.

\end{enumerate}

\paragraph{Hazard Handling}

\begin{enumerate}

  \item{TC-SS-8\\}

    Type: Manual

    Initial State: Unacknowledged documents during site visit and a contractor user account
    exists in application

    Input/Condition: Contractor user declines acknowledgement, attempts to
    acknowledge but does not complete action, or ignores acknowledgement.

    Output/Result: Visit logs identify the failure to complete acknowledgement.

    How test will be performed: Attempt to complete a visit without acknowledgements
    and check how the application handles the action.

\end{enumerate}

\subsection{Traceability Between Test Cases and Requirements}

\begin{longtable}{|l|l|}
  \hline
  \textbf{Req. ID} & \textbf{Test ID's} \\
  \hline
  FR1 & TC-FR1, TC-FR2\\ \hline
  FR3 & TC-FR3\\ \hline
  FR4 & TC-FR4\\ \hline
  FR5 & TC-FR5\\ \hline
  FR6 & TC-FR6\\ \hline
  FR7 & TC-FR8\\ \hline
  FR8 & TC-FR7\\ \hline
  FR9 & TC-FR9\\ \hline
  LF-AP1 & TC-LF-1 \\ \hline
  LF-ST1 & TC-LF-2 \\ \hline
  UH-EU1 & TC-EU1\\ \hline
  UH-EU2 & TC-EU2\\ \hline
  UH-LR1 & TC-LR1\\ \hline
  UH-LR2 & TC-LR2\\ \hline
  UH-UP1 & TC-UP1\\ \hline
  UH-UP2 & TC-UP2\\ \hline
  UH-AS1 & TC-AS1\\ \hline
  PR-SL1 & TC-PR-1\\ \hline
  PR-SL3 & TC-PR-2\\ \hline
  PR-SC1 & TC-PR-3\\ \hline
  PR-SC2 & TC-PR-4\\ \hline
  PR-PA1 & TC-PR-5\\ \hline
  PR-RFT1 & TC-PR-7\\ \hline
  PR-CR2 & TC-PR-10\\ \hline
  PR-SE1 & TC-PR-11\\ \hline
  OE-PE1 & TC-OE-1 \\ \hline
  OE-WE1 & TC-OE-2 \\ \hline
  OE-WE2 & TC-OE-2 \\ \hline
  OE-REL1 & TC-OE-4 \\ \hline
  OE-REL2 & TC-OE-4 \\ \hline
  OE-REL3 & TC-OE-4 \\ \hline
  OE-REL4 & TC-OE-4 \\ \hline
  MS-MTN4 & TC-MS-4 \\ \hline
  MS-MTN5 & TC-MS-5 \\ \hline
  MS-MTN6 & TC-MS-6 \\ \hline
  MS-SUP1 & TC-MS-7 \\ \hline
  MS-SUP2 & TC-MS-8 \\ \hline
  MS-SUP3 & TC-MS-9 \\ \hline
  MS-SUP4 & TC-MS-10 \\ \hline
  MS-ADP1 & TC-LF-2 \\ \hline
  MS-ADP2 & TC-LF-2 \\ \hline
  MS-ADP3 & TC-LF-2 \\ \hline
  SR-AR1 & TC-SS-1 \\ \hline
  SR-AR2 & TC-SS-1 \\ \hline
  SR-AR3 & TC-SS-1 \\ \hline
  SR-AR4 & TC-SS-2 \\ \hline
  SR-IR1 & TC-SS-2 \\ \hline
  SR-IR3 & TC-SS-4 \\ \hline
  SR-PR1 & TC-SS-5\\ \hline
  SR-AU1 & TC-SS-6 \\ \hline
  SR-IMR1 & TC-SS-7 \\ \hline
  SR-S1 & TC-SS-8 \\ \hline
  CR-CR1 & TC-CR-1 \\ \hline
  \caption{Requirements to Test Case Traceability Matrix}
\end{longtable}

\section{Unit Test Description}

This section will be filled out once the design document is completed.

% \wss{This section should not be filled in until after the MIS
%   (detailed design
% document) has been completed.}

% \wss{Reference your MIS (detailed design document) and explain your overall
% philosophy for test case selection.}

% \wss{To save space and time, it may be an option to provide less
%   detail in this section.
%   For the unit tests you can potentially layout your testing strategy
%   here.  That is, you
%   can explain how tests will be selected for each module.  For
%   instance, your test building
%   approach could be test cases for each access program, including one
%   test for normal behaviour
%   and as many tests as needed for edge cases.  Rather than create the
%   details of the input
%   and output here, you could point to the unit testing code.  For
%   this to work, you code
% needs to be well-documented, with meaningful names for all of the tests.}

% \subsection{Unit Testing Scope}

% \wss{What modules are outside of the scope.  If there are modules that are
%   developed by someone else, then you would say here if you
%   aren't planning on
%   verifying them.  There may also be modules that are part of
%   your software, but
%   have a lower priority for verification than others.  If this is the case,
% explain your rationale for the ranking of module importance.}

% \subsection{Tests for Functional Requirements}

% \wss{Most of the verification will be through automated unit testing.  If
%   appropriate specific modules can be verified by a non-testing based
% technique.  That can also be documented in this section.}

% \subsubsection{Module 1}

% \wss{Include a blurb here to explain why the subsections below
%   cover the module.
%   References to the MIS would be good.  You will want tests from a black box
%   perspective and from a white box perspective.  Explain to the
%   reader how the
% tests were selected.}

% \begin{enumerate}

%   \item{test-id1\\}

%     Type: \wss{Functional, Dynamic, Manual, Automatic, Static etc. Most will
%     be automatic}

%     Initial State:

%     Input:

%     Output: \wss{The expected result for the given inputs}

%     Test Case Derivation: \wss{Justify the expected value given in
%     the Output field}

%     How test will be performed:

%   \item{test-id2\\}

%     Type: \wss{Functional, Dynamic, Manual, Automatic, Static etc. Most will
%     be automatic}

%     Initial State:

%     Input:

%     Output: \wss{The expected result for the given inputs}

%     Test Case Derivation: \wss{Justify the expected value given in
%     the Output field}

%     How test will be performed:

%   \item{...\\}

% \end{enumerate}

% \subsubsection{Module 2}

% ...

% \subsection{Tests for Nonfunctional Requirements}

% \wss{If there is a module that needs to be independently assessed for
%   performance, those test cases can go here.  In some projects, planning for
% nonfunctional tests of units will not be that relevant.}

% \wss{These tests may involve collecting performance data from previously
% mentioned functional tests.}

% \subsubsection{Module ?}

% \begin{enumerate}

%   \item{test-id1\\}

%     Type: \wss{Functional, Dynamic, Manual, Automatic, Static etc. Most will
%     be automatic}

%     Initial State:

%     Input/Condition:

%     Output/Result:

%     How test will be performed:

%   \item{test-id2\\}

%     Type: Functional, Dynamic, Manual, Static etc.

%     Initial State:

%     Input:

%     Output:

%     How test will be performed:

% \end{enumerate}

% \subsubsection{Module ?}

% ...

% \subsection{Traceability Between Test Cases and Modules}

% \wss{Provide evidence that all of the modules have been considered.}

\newpage

\bibliographystyle{plainnat}

\bibliography{../../refs/References}

\newpage

\section{Appendix}

\subsection{Symbolic Parameters}
The following symbolic parameters are used:

\begin{align*}
 \text{MIN\_AGE} &= 18\\
 \text{MAX\_AGE} &= 70\\
\end{align*}

\subsection{Usability Survey Questions}
\begin{enumerate}
  \item Do you feel the initial screen you saw made the purpose of the system clear?
  \item Was the colour palette of the system aesthetically pleasing and pleasant to look at?
  \item Do you feel comfortable if you used the system again at a later date, you would be able to navigate the screens successfully?
  \item Were you able to successfully locate the site wide and site specific documents in the admin portal? 
  \item On a scale of 1 to 10, how confident are you using the system without assistance?
  \item On a scale of 1 to 10, how easy was it to discover features of the system?
  \item On a scale of 1 to 10, was the experience of adding documents to the admin portal a smooth experience?
  \item Did you feel frustrated at all in the time it took for you to view a document?
  \item How confident were you in interpreting the system feedback? Was it able
    provide confirmation or possible solutions to your input?
  \item Did you find the content of the system to be politically neutral and
    unoffensive?
\end{enumerate}

\newpage{}
\section*{Appendix --- Reflection}

The information in this section will be used to evaluate the team members on the
graduate attribute of Lifelong Learning.

The purpose of reflection questions is to give you a chance to assess your own
learning and that of your group as a whole, and to find ways to improve in the
future. Reflection is an important part of the learning process.  Reflection is
also an essential component of a successful software development process.  

Reflections are most interesting and useful when they're honest, even if the
stories they tell are imperfect. You will be marked based on your depth of
thought and analysis, and not based on the content of the reflections
themselves. Thus, for full marks we encourage you to answer openly and honestly
and to avoid simply writing ``what you think the evaluator wants to hear.''

Please answer the following questions.  Some questions can be answered on the
team level, but where appropriate, each team member should write their own
response:


\begin{enumerate}
  \item What went well while writing this deliverable?\\
    \\
    When writing this deliverable, our team was able to quickly meet
    and delegate
    sections of the document between members. We had a very productive TA
    meeting for the VnV plan which answered many questions we had. One example
    was we were able to clarify that section 5 of the document would be
    completed later when the design of the system had been done.

  \item What pain points did you experience during this deliverable, and how
    did you resolve them?\\
    \\
    One pain point we experienced was deciding what the best verification and
    validation methods would be. We wanted to ensure that we chose the right
    approaches for this application which would result in the highest quality
    of work. We resolved this challenge by discussing and investigating the
    merits of different options, and developing a test suite which covered
    every requirement we had at least once.

  \item What knowledge and skills will the team collectively need to acquire to
    successfully complete the verification and validation of your project?
    Examples of possible knowledge and skills include dynamic
    testing knowledge,
    static testing knowledge, specific tool usage, Valgrind etc.  You
    should look to
    identify at least one item for each team member.\\
    \\

    Akshit Gulia - Needs to get familiar with the PyTest testing
    framework for python. \\

    Mitchell Hynes - Needs to get familiar with the PyTest testing
    framework for python and Figma for prototyping designs. \\

    Richard Fan - Needs to get familiar with Jest testing framework
    for javascript and PyTest testing framework in python. \\

    Kyle D'Souza - Needs to get familiar with Jest test framework for
    javascript and Figma for prototyping designs. \\

    Rafeed Iqbal - Needs to get familiar with PyTest framework for
    python and GitHub actions for running tests in GitHub. \\

  \item For each of the knowledge areas and skills identified in the previous
    question, what are at least two approaches to acquiring the knowledge or
    mastering the skill?  Of the identified approaches, which will each team
    member pursue, and why did they make this choice?\\
    \\

    For all of the aforementioned skills some potential ways of
    acquiring knowledge or mastering the skill would be to get hands
    on experience with the skill by applying it in a project, reading
    documentation, and watching online tutorials.

    Akshit Gulia - Plans on watching tutorials on youtube related to
    PyTest, and also plans on building a small personal learning
    project on the side in python which uses PyTest for testing. \\

    Mitchell Hynes - Plans to follow the introductory tutorials
    provided by the creators of PyTest and Figma and will also read
    documentation on them. \\

    Richard Fan - Plans to read documentation on Jest and PyTest, and
    also use them in a personal project. \\

    Kyle D'Souza - Plans to watch online tutorials on Jest and Figma.
    Also plans to read documentation on them. \\

    Rafeed Iqbal - Plans to read documentation on GitHub actions and
    Pytest. Also plans to watch online tutorials on them. \\
\end{enumerate}

\end{document}
