\documentclass[12pt, titlepage]{article}

\usepackage{booktabs}
\usepackage{tabularx}
\usepackage{hyperref}
\hypersetup{
  colorlinks,
  citecolor=blue,
  filecolor=black,
  linkcolor=red,
  urlcolor=blue
}
\usepackage[round]{natbib}
\usepackage{longtable}

%% Comments

\usepackage{color}

\newif\ifcomments\commentstrue %displays comments
%\newif\ifcomments\commentsfalse %so that comments do not display

\ifcomments
\newcommand{\authornote}[3]{\textcolor{#1}{[#3 ---#2]}}
\newcommand{\todo}[1]{\textcolor{red}{[TODO: #1]}}
\else
\newcommand{\authornote}[3]{}
\newcommand{\todo}[1]{}
\fi

\newcommand{\wss}[1]{\authornote{blue}{SS}{#1}} 
\newcommand{\plt}[1]{\authornote{magenta}{TPLT}{#1}} %For explanation of the template
\newcommand{\an}[1]{\authornote{cyan}{Author}{#1}}

%% Common Parts

\newcommand{\progname}{ProgName} % PUT YOUR PROGRAM NAME HERE
\newcommand{\authname}{Team \#, Team Name
\\ Student 1 name
\\ Student 2 name
\\ Student 3 name
\\ Student 4 name} % AUTHOR NAMES                  

\usepackage{hyperref}
    \hypersetup{colorlinks=true, linkcolor=blue, citecolor=blue, filecolor=blue,
                urlcolor=blue, unicode=false}
    \urlstyle{same}
                                


\begin{document}

\title{System Verification and Validation Plan for \progname{}}
\author{\authname}
\date{\today}

\maketitle

\pagenumbering{roman}

\section*{Revision History}

\begin{tabularx}{\textwidth}{p{3cm}p{2cm}X}
  \toprule {\bf Date} & {\bf Version} & {\bf Notes}\\
  \midrule
  Date 1 & 1.0 & Notes\\
  Date 2 & 1.1 & Notes\\
  \bottomrule
\end{tabularx}

~\\
\wss{The intention of the VnV plan is to increase confidence in the software.
  However, this does not mean listing every verification and
  validation technique
  that has ever been devised.  The VnV plan should also be a \textbf{feasible}
  plan. Execution of the plan should be possible with the time and
  team available.
  If the full plan cannot be completed during the time available, it
  can either be
  modified to ``fake it'', or a better solution is to add a section describing
what work has been completed and what work is still planned for the future.}

\wss{The VnV plan is typically started after the requirements stage, but before
  the design stage.  This means that the sections related to unit testing cannot
  initially be completed.  The sections will be filled in after the design stage
  is complete.  the final version of the VnV plan should have all
  sections filled
in.}

\newpage

\tableofcontents

\listoftables
\wss{Remove this section if it isn't needed}

\listoffigures
\wss{Remove this section if it isn't needed}

\newpage

\section{Symbols, Abbreviations, and Acronyms}

\renewcommand{\arraystretch}{1.2}
\begin{tabular}{l l}
  \toprule
  \textbf{symbol} & \textbf{description}\\
  \midrule
  T & Test\\
  \bottomrule
\end{tabular}\\

\wss{symbols, abbreviations, or acronyms --- you can simply reference the SRS
\citep{SRS} tables, if appropriate}

\wss{Remove this section if it isn't needed}

\newpage

\pagenumbering{arabic}

This document ... \wss{provide an introductory blurb and roadmap of the
Verification and Validation plan}

\section{General Information}

\subsection{Summary}

\wss{Say what software is being tested.  Give its name and a brief overview of
its general functions.}

\subsection{Objectives}

\wss{State what is intended to be accomplished.  The objective will be around
  the qualities that are most important for your project.  You might have
  something like: ``build confidence in the software correctness,''
  ``demonstrate adequate usability.'' etc.  You won't list all of the qualities,
just those that are most important.}

\wss{You should also list the objectives that are out of scope.  You don't have
  the resources to do everything, so what will you be leaving out.
  For instance,
  if you are not going to verify the quality of usability, state
  this.  It is also
worthwhile to justify why the objectives are left out.}

\wss{The objectives are important because they highlight that you are aware of
  limitations in your resources for verification and validation.  You
  can't do everything,
  so what are you going to prioritize?  As an example, if your system
  depends on an
  external library, you can explicitly state that you will assume
  that external library
has already been verified by its implementation team.}

\subsection{Challenge Level and Extras}

The challenge level for this project is a general level, and the
extras are conducting user testing and developing a user manual. For
more information on the challenge level and extras see section
\textit{Challenge Level and Extras} of the Problem Statement
\href{https://github.com/Spitgranger/SyncMaster/blob/main/docs/ProblemStatementAndGoals/ProblemStatement.md#4-challenge-level-and-extras}
{here}.

\subsection{Relevant Documentation}

\wss{Reference relevant documentation.  This will definitely include your SRS
  and your other project documents (design documents, like MG, MIS, etc).  You
  can include these even before they are written, since by the time the project
  is done, they will be written.  You can create BibTeX entries for your
documents and within those entries include a hyperlink to the documents.}

\citet{SRS}

\wss{Don't just list the other documents.  You should explain why
  they are relevant and
how they relate to your VnV efforts.}

\section{Plan}

\wss{Introduce this section.  You can provide a roadmap of the sections to
come.}

\subsection{Verification and Validation Team}

The verification and validation team will consist of members from the
development team as well as members from the project stakeholders. A summary of
the members and their roles are given in the following table:
\setlength{\arrayrulewidth}{0.5mm}
\setlength{\tabcolsep}{18pt}
\renewcommand{\arraystretch}{1.5}s
\begin{tabular}{ | m{5cm} | m{9cm} | }
  \hline
  \textbf{Names and Role} & \textbf{Responsibilites} \\
  \hline
  \textbf{Mitchell Hynes, Kyle D'Souza, Richard Fan, Akshit Gulia, Rafeed
  Iqbal} - Development Team &
  \begin{itemize}
    \item Responsible for verification of implementation details and
      adherance of the system to the SRS. The development team will own the
      most of the verification activities pertaining to the software. This
      includes activities such as developing verification plans, code reviews,
      unit testing, integration testing, and SRS verification.

  \end{itemize}\\
  \hline
  \textbf{Matthew Yakymyshyn} - Stakeholder from the City of Hamilton &
  \begin{itemize}
    \item Responsible for the verification of the system. They will be
      responsible for ensuring, and determining to what degree the
      system fulfils its intended purpose.
      Some examples of the activities that they would be responsible for
      would be SRS validation and acceptance testing.

  \end{itemize}\\
  \hline
  \textbf{Tarnveer Takhtar, Matthew Bradbury, Harman Bassi, Kyen So}
  - Peer Reviewers &
  \begin{itemize}
    \item Responsible for critiquing and providing suggestions for artifacts
      produced. This would include reviewing design documents to ensure that
      they are complete and unambiguous. They will provide opinions and
      insights as third parties who are not directly involved in the project.
  \end{itemize}\\
  \hline
  \textbf{Spencer Smith, Yiding Li} - Teaching Team for 4G06 &
  \begin{itemize}
    \item They will also be responsible for the final
      evaluation of the system and will determine if the system meets the
      initially specified functional and non-functional requirements.
      They will also play a role in providing feedback on produced artifacts
      throughout the project.
  \end{itemize}\\
  \hline

\end{tabular}

\subsection{SRS Verification Plan}

\wss{List any approaches you intend to use for SRS verification.  This may
  include ad hoc feedback from reviewers, like your classmates (like your
primary reviewer), or you may plan for something more rigorous/systematic.}

\wss{If you have a supervisor for the project, you shouldn't just say they will
  read over the SRS.  You should explain your structured approach to the review.
  Will you have a meeting?  What will you present?  What questions will you ask?
  Will you give them instructions for a task-based inspection?  Will
  you use your
issue tracker?}

\wss{Maybe create an SRS checklist?}

\subsection{Design Verification Plan}

\wss{Plans for design verification}

\wss{The review will include reviews by your classmates}

\wss{Create a checklists?}

\subsection{Verification and Validation Plan Verification Plan}

\wss{The verification and validation plan is an artifact that should also be
verified.  Techniques for this include review and mutation testing.}

\wss{The review will include reviews by your classmates}

\wss{Create a checklists?}

\subsection{Implementation Verification Plan}

\wss{You should at least point to the tests listed in this document and the unit
testing plan.}

\wss{In this section you would also give any details of any plans for static
  verification of the implementation.  Potential techniques include code
walkthroughs, code inspection, static analyzers, etc.}

\wss{The final class presentation in CAS 741 could be used as a code
  walkthrough.  There is also a possibility of using the final presentation (in
CAS741) for a partial usability survey.}

\subsection{Automated Testing and Verification Tools}

\wss{What tools are you using for automated testing.  Likely a unit testing
  framework and maybe a profiling tool, like ValGrind.  Other possible tools
  include a static analyzer, make, continuous integration tools, test coverage
  tools, etc.  Explain your plans for summarizing code coverage metrics.
  Linters are another important class of tools.  For the programming language
  you select, you should look at the available linters.  There may also be tools
  that verify that coding standards have been respected, like flake9 for
Python.}

\wss{If you have already done this in the development plan, you can point to
that document.}

\wss{The details of this section will likely evolve as you get closer to the
implementation.}

\subsection{Software Validation Plan}

\begin{enumerate}
  \item Rev0 prototype demonstration to Technical Services staff. The rev0
    prototype will be demonstrated to the City Staff on this team at
    an in-person
    meeting. Each requirement from the SRS will be demonstrated in
    the demo so the
    full scope of the application is displayed. The demonstration will also show
    how it satisfies the use cases identified in the SRS. Feedback
    received from the
    City at this meeting will be used to improve the prototype for rev1.

  \item Stakeholder progress check-ins. Short periodic meetings will be arranged
    with the Technical Services team as required to demonstrate the
    user interface
    design prototypes and receive feedback from the facilities
    managers. Email updates
    will also be sent for items which require stakeholder
    clarification. The user interface is the most important part of
    the application to validate with the end-users.
    Consistent communication on the direction of the interface design
    to ensure it
    is usable will catch problems early and greatly improve the quality and
    acceptance of the application.
\end{enumerate}

\section{System Tests}

\wss{There should be text between all headings, even if it is just a roadmap of
the contents of the subsections.}

\subsection{Tests for Functional Requirements}

\wss{Subsets of the tests may be in related, so this section is divided into
  different areas.  If there are no identifiable subsets for the tests, this
level of document structure can be removed.}

\wss{Include a blurb here to explain why the subsections below
cover the requirements.  References to the SRS would be good here.}

\subsubsection{Area of Testing1}

\wss{It would be nice to have a blurb here to explain why the subsections below
  cover the requirements.  References to the SRS would be good here.
  If a section
  covers tests for input constraints, you should reference the data constraints
table in the SRS.}

\paragraph{Title for Test}

\begin{enumerate}

  \item{test-id1\\}

    Control: Manual versus Automatic

    Initial State:

    Input:

    Output: \wss{The expected result for the given inputs.  Output is
      not how you
      are going to return the results of the test.  The output is the expected
    result.}

    Test Case Derivation: \wss{Justify the expected value given in
    the Output field}

    How test will be performed:

  \item{test-id2\\}

    Control: Manual versus Automatic

    Initial State:

    Input:

    Output: \wss{The expected result for the given inputs}

    Test Case Derivation: \wss{Justify the expected value given in
    the Output field}

    How test will be performed:

\end{enumerate}

\subsubsection{Area of Testing2}

...

\subsection{Tests for Nonfunctional Requirements}

\wss{The nonfunctional requirements for accuracy will likely just reference the
  appropriate functional tests from above.  The test cases should mention
  reporting the relative error for these tests.  Not all projects will
necessarily have nonfunctional requirements related to accuracy.}

\wss{For some nonfunctional tests, you won't be setting a target threshold for
  passing the test, but rather describing the experiment you will do to measure
  the quality for different inputs.  For instance, you could measure
  speed versus
  the problem size.  The output of the test isn't pass/fail, but
  rather a summary
table or graph.}

\wss{Tests related to usability could include conducting a usability test and
survey.  The survey will be in the Appendix.}

\wss{Static tests, review, inspections, and walkthroughs, will not follow the
format for the tests given below.}

\wss{If you introduce static tests in your plan, you need to provide details.
  How will they be done?  In cases like code (or document)
  walkthroughs, who will
be involved? Be specific.}

\subsubsection{Area of Testing1}

\paragraph{Title for Test}

\begin{enumerate}

  \item{test-id1\\}

    Type: Functional, Dynamic, Manual, Static etc.

    Initial State:

    Input/Condition:

    Output/Result:

    How test will be performed:

  \item{test-id2\\}

    Type: Functional, Dynamic, Manual, Static etc.

    Initial State:

    Input:

    Output:

    How test will be performed:

\end{enumerate}

\subsubsection{Look and Feel Requirements}

\paragraph{Appearance}

\begin{enumerate}

  \item{TC-LF-1\\}

    Type: Manual

    Initial State: The system is in an operational state, with all
    components running

    Input/Condition: User logs in, and views all available screens of
    the user interface

    Output/Result: The user has seen all available screens of the
    user interface and finds that most of the colours used by the
    application match a similar colour palette to existing
    applications used by the city of Hamilton.

    How test will be performed: The test will be performed by getting
    one user who works at the city of Hamilton for each type of role
    (General/Contractor, Manager, Admin). Then a member the
    development team will guide them to all available screens for their role and
    ask them to evaluate how many colours that were seen are were
    similar to ones used in other applications used by the city of
    Hamilton and how many were not similar. If the number of colours
    that are similar is larger than the number that are not, then the
    test is successful, otherwise it is failed.

  \item{TC-LF-2\\}

    Type: Dynamic, Manual.

    Initial State: The system is in an operational state, with all
    components running

    Input/Condition: Load and navigate through the application on
    various supported operating systems, with varying supported
    browsers and screen sizes.

    Output/Result: User is able to interact with the system on all screen
    sizes and understand the interface regardless of screen size

    How test will be performed: One user for each type of role
    (General/Contractor, Manager, Admin), will be go through the core
    functionalities important for their roles and ensure that all
    functionalities work as expected across screen sizes, browsers,
    and operating systems. The tests should be performed using
    Windows 10, IOS, and Android, which are the supported operating
    systems for this application. The browsers used should be
    Microsoft Edge and Google Chrome. Testing should include a mobile
    device, a desktop, and a tablet to ensure accommodation of
    different screen sizes and of different types of devices.

    The core functionalites for a contractor/general user will be
    signing into the system, viewing a document, signing a document,
    and logging out. The core functionalities of a manager will be
    uploading a document, viewing document update history, and seeing
    the status of a contractor/general user under them. The core
    functionalities of an admin will be creating users, deleting
    users, and assigning roles to users.

\end{enumerate}

\subsubsection{Maintainability and Support Requirements}

\paragraph{Maintenance}

\begin{enumerate}

  \item{TC-MS-1\\}

    Type: Manual

    Initial State: Cloud platform used for deployments is functional,
    GitHub actions is functional.

    Input/Condition: A workflow run in GitHub actions is triggered to
    deploy the application into a specified environment, and
    building/automated unit testing of the application is completed

    Output/Result: The application is deployed to the specified
    environment in under 30 minutes

    How test will be performed: The test will be performed by
    triggering a GitHub actions workflow run to deploy the application
    into the development environment, and waiting for it to complete.
    Once completed, the unit testing and building time will be subtracted
    from the total runtime of the workflow run, to determine the
    deployment time, this should be under 30 minutes to be considered a success.

  \item{TC-MS-2\\}

    Type: Manual

    Initial State: GitHub actions is functional.

    Input/Condition: A workflow run in GitHub actions is triggered to
    deploy the application into a specified environment, and
    automated unit testing of the application is completed.

    Output/Result: The application is built in under 10 minutes.

    How test will be performed: The test will be performed by
    triggering a GitHub actions workflow run to deploy the application
    into the development environment, and waiting for it to complete.
    Once completed, the unit testing and deployment time will be subtracted
    from the total runtime of the workflow run, to determine the
    build time, this should be under 10 minutes to be considered a success.

  \item{TC-MS-3\\}

    Type: Manual

    Initial State: GitHub actions is functional.

    Input/Condition: A workflow run in GitHub actions is triggered to
    perform all automated testing (end-to-end or unit testing) of the
    application.

    Output/Result: The automated tests for the application run in
    under 10 minutes.

    How test will be performed: The test will be performed by
    triggering a GitHub actions workflow run to perform all automated
    tests, and checking that once this run is completed, the total
    runtime is under 10 minutes.

  \item{TC-MS-4\\}

    Type: Manual

    Initial State: GitHub actions is functional.

    Input/Condition: A push is made to a branch with a pull request
    open off of it.

    Output/Result: The branch is found to have line coverage $\ge$ 95\% and
    the branch coverage is found to be $\ge$ 90\%

    How test will be performed: The test will be performed by
    triggering a GitHub actions workfow run to run unit tests on the
    application upon every push to a branch with a PR open off of it.
    The workflow will generate a coverage report, and when the
    coverage report is generated, it should be checked that the line
    coverage and branch coverage meet the expectations.

  \item{TC-MS-5\\}

    Type: Manual

    Initial State: GitHub is functional

    Input/Condition: The GitHub repository for the project is checked
    to ensure that all functional requirements listed in the SRS have
    a unit and end-to-end test cases corresponding to them.

    Output/Result: All functional requirements found in the SRS have a unit
    and end-to-end test cases corresponding to them

    How test will be performed: The test will be performed by
    checking the Traceability Matrices in this document, to make sure
    that there exists a unit and end-to-end test case for all
    functional requirements.

  \item{TC-MS-6\\}

    Type: Manual

    Initial State: GitHub is functional

    Input/Condition: The GitHub repository for the project is checked
    to ensure that all appropriate documentation existed for users to
    be able to maintain the system.

    Output/Result: Instructions are provided in the GitHub repository
    for the project on how users can continue to maintain the system.
    This includes contribution guidelines, descriptions of all
    GitHub actions workflows, and documentation on the design of the system.

    How test will be performed: The test will be performed by
    checking that the documentation listed in the output/result
    exists in the repository, and by verifying with the Matthew
    Yakymyshyn that this documentation is able to be understood by
    the city of Hamilton.

  \item{TC-MS-7\\}

    Type: Manual

    Initial State: GitHub is functional

    Input/Condition: The GitHub repository for the project is checked
    to see the user manual.

    Output/Result: There exists a user manual in the github
    repository which describes, at a minimum, how to leverage the
    functionalities described in the functional requirements of the
    SRS from the user interface.

    How test will be performed: The test will be performed by
    checking that the user manual exists, and that there is
    user-facing documentation in the manual on how to achieve all
    functionalities described in the functional requirements.

  \item{TC-MS-8\\}

    Type: Manual

    Initial State: GitHub is functional

    Input/Condition: The GitHub repository for the project is checked
    to see the API documentation.

    Output/Result: There exists API documentation in the github
    repository which follows the OpenAPI Specification (OAS) standard.

    How test will be performed: The test will be performed by
    checking that an OpenAPI Specification for the API's provided by
    the system exists on the project repository.

  \item{TC-MS-9\\}

    Type: Manual

    Initial State: GitHub is functional

    Input/Condition: The GitHub repository for the project is checked
    to see the the internal documentation.

    Output/Result: There exists documentation on all internal
    functions and classes defined in the project repository.

    How test will be performed: The test will be performed by
    checking that documentation exists for all functions and classes
    defined in the project repository.

  \item{TC-MS-10\\}

    Type: Manual

    Initial State: GitHub is functional, Cloud deployment platform is functional

    Input/Condition: The GitHub repository for the project is checked
    to see the the deployment documentation

    Output/Result: There exists documentation on how to deploy the
    project, which is up-to-date and works when attempted.

    How test will be performed: The test will be performed by
    checking that the deployment documentation exists and that a
    member of our development team can follow the provided steps to
    successfully deploy the appication.

\end{enumerate}

\subsection{Compliance Requirements}

    \item{TC-CR-1\\}
        
    Type: Manual

    Initial State: Github is functional, cloud deployment platform is functional

    Input/Condition: A contractor uses the authentication feature of the application
    to verify their presence on site for the facilities managers.

    Output/Result: The contractor successfully authenticates in the application
    and successfully complies with each step of the entry/exit procedure document
    for the station provided through the application.

    How the test will be performed: The contractor user will authenticate at the
    site for a work order to fix an exhaust fan. Through the authentication process,
    the entry/exit procedure will be used as a checklist to ensure it is followed
    and the legislaiton specified in CR-L1 is not violated.
...

\subsection{Traceability Between Test Cases and Requirements}

\wss{Provide a table that shows which test cases are supporting which
requirements.}

\begin{longtable}{|l|l|}
  \hline
  \textbf{Req. ID} & \textbf{Test ID's} \\
  \hline
  FR1 & \\ \hline
  FR2 & \\ \hline
  FR3 & \\ \hline
  FR4 & \\ \hline
  FR5 & \\ \hline
  FR6 & \\ \hline
  FR7 & \\ \hline
  FR8 & \\ \hline
  FR9 & \\ \hline
  LF-AP1 & TC-LF-1 \\ \hline
  LF-ST1 & TC-LF-2 \\ \hline
  UH-EU1 & \\ \hline
  UH-EU2 & \\ \hline
  UH-LR1 & \\ \hline
  UH-LR2 & \\ \hline
  UH-UP1 & \\ \hline
  UH-UP2 & \\ \hline
  UH-AS1 & \\ \hline
  PR-SL1 & \\ \hline
  PR-SL2 & \\ \hline
  PR-SL3 & \\ \hline
  PR-SC1 & \\ \hline
  PR-SC2 & \\ \hline
  PR-PA1 & \\ \hline
  PR-PA2 & \\ \hline
  PR-RFT1 & \\ \hline
  PR-RFT2 & \\ \hline
  PR-CR1 & \\ \hline
  PR-CR2 & \\ \hline
  PR-SE1 & \\ \hline
  PR-SE2 & \\ \hline
  PR-LR1 & \\ \hline
  OE-PE1 & \\ \hline
  OE-WE1 & \\ \hline
  OE-WE2 & \\ \hline
  OE-IAS1 & \\ \hline
  OE-IAS2 & \\ \hline
  OE-IAS2 & \\ \hline
  OE-REL1 & \\ \hline
  OE-REL2 & \\ \hline
  OE-REL3 & \\ \hline
  OE-REL4 & \\ \hline
  MS-MTN1 & TC-MS-1 \\ \hline
  MS-MTN2 & TC-MS-2 \\ \hline
  MS-MTN3 & TC-MS-3 \\ \hline
  MS-MTN4 & TC-MS-4 \\ \hline
  MS-MTN5 & TC-MS-5 \\ \hline
  MS-MTN6 & TC-MS-6 \\ \hline
  MS-SUP1 & TC-MS-7 \\ \hline
  MS-SUP2 & TC-MS-8 \\ \hline
  MS-SUP3 & TC-MS-9 \\ \hline
  MS-SUP4 & TC-MS-10 \\ \hline
  MS-ADP1 & TC-LF-2 \\ \hline
  MS-ADP2 & TC-LF-2 \\ \hline
  MS-ADP3 & TC-LF-2 \\ \hline
  SR-AR1 & \\ \hline
  SR-AR2 & \\ \hline
  SR-AR3 & \\ \hline
  SR-AR4 & \\ \hline
  SR-IR1 & \\ \hline
  SR-IR2 & \\ \hline
  SR-IR3 & \\ \hline
  SR-PR1 & \\ \hline
  MS-PR2 & \\ \hline
  SR-AU1 & \\ \hline
  SR-IMR1 & \\ \hline
  CR-CR1 & \\ \hline
  CR-L1 & \\ \hline
  CR-L2 & \\ \hline
  CR-S1 & \\ \hline
  \caption{Requirements to Test Case Traceability Matrix}
\end{longtable}

\section{Unit Test Description}

\wss{This section should not be filled in until after the MIS (detailed design
document) has been completed.}

\wss{Reference your MIS (detailed design document) and explain your overall
philosophy for test case selection.}

\wss{To save space and time, it may be an option to provide less
  detail in this section.
  For the unit tests you can potentially layout your testing strategy
  here.  That is, you
  can explain how tests will be selected for each module.  For
  instance, your test building
  approach could be test cases for each access program, including one
  test for normal behaviour
  and as many tests as needed for edge cases.  Rather than create the
  details of the input
  and output here, you could point to the unit testing code.  For
  this to work, you code
needs to be well-documented, with meaningful names for all of the tests.}

\subsection{Unit Testing Scope}

\wss{What modules are outside of the scope.  If there are modules that are
  developed by someone else, then you would say here if you aren't planning on
  verifying them.  There may also be modules that are part of your software, but
  have a lower priority for verification than others.  If this is the case,
explain your rationale for the ranking of module importance.}

\subsection{Tests for Functional Requirements}

\wss{Most of the verification will be through automated unit testing.  If
  appropriate specific modules can be verified by a non-testing based
technique.  That can also be documented in this section.}

\subsubsection{Module 1}

\wss{Include a blurb here to explain why the subsections below cover the module.
  References to the MIS would be good.  You will want tests from a black box
  perspective and from a white box perspective.  Explain to the reader how the
tests were selected.}

\begin{enumerate}

  \item{test-id1\\}

    Type: \wss{Functional, Dynamic, Manual, Automatic, Static etc. Most will
    be automatic}

    Initial State:

    Input:

    Output: \wss{The expected result for the given inputs}

    Test Case Derivation: \wss{Justify the expected value given in
    the Output field}

    How test will be performed:

  \item{test-id2\\}

    Type: \wss{Functional, Dynamic, Manual, Automatic, Static etc. Most will
    be automatic}

    Initial State:

    Input:

    Output: \wss{The expected result for the given inputs}

    Test Case Derivation: \wss{Justify the expected value given in
    the Output field}

    How test will be performed:

  \item{...\\}

\end{enumerate}

\subsubsection{Module 2}

...

\subsection{Tests for Nonfunctional Requirements}

\wss{If there is a module that needs to be independently assessed for
  performance, those test cases can go here.  In some projects, planning for
nonfunctional tests of units will not be that relevant.}

\wss{These tests may involve collecting performance data from previously
mentioned functional tests.}

\subsubsection{Module ?}

\begin{enumerate}

  \item{test-id1\\}

    Type: \wss{Functional, Dynamic, Manual, Automatic, Static etc. Most will
    be automatic}

    Initial State:

    Input/Condition:

    Output/Result:

    How test will be performed:

  \item{test-id2\\}

    Type: Functional, Dynamic, Manual, Static etc.

    Initial State:

    Input:

    Output:

    How test will be performed:

\end{enumerate}

\subsubsection{Module ?}

...

\subsection{Traceability Between Test Cases and Modules}

\wss{Provide evidence that all of the modules have been considered.}

\bibliographystyle{plainnat}

\bibliography{../../refs/References}

\newpage

\section{Appendix}

This is where you can place additional information.

\subsection{Symbolic Parameters}

The definition of the test cases will call for SYMBOLIC\_CONSTANTS.
Their values are defined in this section for easy maintenance.

\subsection{Usability Survey Questions?}

\wss{This is a section that would be appropriate for some projects.}

\newpage{}
\section*{Appendix --- Reflection}

\wss{This section is not required for CAS 741}

The information in this section will be used to evaluate the team members on the
graduate attribute of Lifelong Learning.

The purpose of reflection questions is to give you a chance to assess your own
learning and that of your group as a whole, and to find ways to improve in the
future. Reflection is an important part of the learning process.  Reflection is
also an essential component of a successful software development process.  

Reflections are most interesting and useful when they're honest, even if the
stories they tell are imperfect. You will be marked based on your depth of
thought and analysis, and not based on the content of the reflections
themselves. Thus, for full marks we encourage you to answer openly and honestly
and to avoid simply writing ``what you think the evaluator wants to hear.''

Please answer the following questions.  Some questions can be answered on the
team level, but where appropriate, each team member should write their own
response:


\begin{enumerate}
  \item What went well while writing this deliverable?
  \item What pain points did you experience during this deliverable, and how
    did you resolve them?
  \item What knowledge and skills will the team collectively need to acquire to
    successfully complete the verification and validation of your project?
    Examples of possible knowledge and skills include dynamic testing knowledge,
    static testing knowledge, specific tool usage, Valgrind etc.  You
    should look to
    identify at least one item for each team member.
  \item For each of the knowledge areas and skills identified in the previous
    question, what are at least two approaches to acquiring the knowledge or
    mastering the skill?  Of the identified approaches, which will each team
    member pursue, and why did they make this choice?
\end{enumerate}

\end{document}
