\documentclass{article}

\usepackage{booktabs}
\usepackage{tabularx}
\usepackage{hyperref}
\usepackage{graphicx}
\usepackage{enumerate}
\usepackage{float}
\usepackage[normalem]{ulem}
\hypersetup{
  colorlinks=true,       % false: boxed links; true: colored links
  linkcolor=red,          % color of internal links (change box color
  % with linkbordercolor)
  citecolor=green,        % color of links to bibliography
  filecolor=magenta,      % color of file links
  urlcolor=cyan           % color of external links
}

\title{Hazard Analysis\\\progname}

\author{\authname}

\date{}

%% Comments

\usepackage{color}

\newif\ifcomments\commentstrue %displays comments
%\newif\ifcomments\commentsfalse %so that comments do not display

\ifcomments
\newcommand{\authornote}[3]{\textcolor{#1}{[#3 ---#2]}}
\newcommand{\todo}[1]{\textcolor{red}{[TODO: #1]}}
\else
\newcommand{\authornote}[3]{}
\newcommand{\todo}[1]{}
\fi

\newcommand{\wss}[1]{\authornote{blue}{SS}{#1}} 
\newcommand{\plt}[1]{\authornote{magenta}{TPLT}{#1}} %For explanation of the template
\newcommand{\an}[1]{\authornote{cyan}{Author}{#1}}

%% Common Parts

\newcommand{\progname}{ProgName} % PUT YOUR PROGRAM NAME HERE
\newcommand{\authname}{Team \#, Team Name
\\ Student 1 name
\\ Student 2 name
\\ Student 3 name
\\ Student 4 name} % AUTHOR NAMES                  

\usepackage{hyperref}
    \hypersetup{colorlinks=true, linkcolor=blue, citecolor=blue, filecolor=blue,
                urlcolor=blue, unicode=false}
    \urlstyle{same}
                                


\begin{document}

\maketitle
\thispagestyle{empty}

~\newpage

\pagenumbering{roman}

\begin{table}[hp]
  \caption{Revision History} \label{TblRevisionHistory}
  \begin{tabularx}{\textwidth}{llX}
    \toprule
    \textbf{Date} & \textbf{Developer(s)} & \textbf{Change}\\
    \midrule
    10/25/2024 & Whole Team & Rev0 Hazard Analysis\\
    11/04/2024 & Rafeed Iqbal & Changed ``SFR1'' to ``SR-S1''\\
    02/20/2025 & Mitchell Hynes & added peer feedback \#195\\
    03/31/2025 & Mitchell Hynes & removed H2-3\\
    \bottomrule
  \end{tabularx}
\end{table}

~\newpage

\tableofcontents

\listoftables

~\newpage

\pagenumbering{arabic}

\section{Introduction}

Hazards can be defined as either a state of the system or an action
taken on the system, which causes system failures. System failures
include, but are not limited to, an inability of the system to
perform its required functions, loss of data, unauthorized access,
or harm to users.

\section{Scope and Purpose of Hazard Analysis}

This document outlines a hazard analysis for SyncMaster, which is a system that
manages documentation and enables administrators to verify presence of users on
site. The purpose of this analysis is to identify and evaluate potential
hazards, outline how they will be detected/introduced to the system, and to
establish mitigation strategies. This analysis will also determine
the parties responsible
for each of the aforementioned points. Some of the losses that
may be incurred as a result of hazards are, but are not limited to:
financial loss, data loss, regulatory violations, and
operational interruption. These losses stemming from hazards must be mitigated
to ensure that the system operates securely, and in compliance with relevant
regulations.\\


Financial loss would be possible if the User Presence Verification Module stops functioning. This provides a record
of contractors actually being on site, allowing for a history that can be audited to ensure invoices for work are accurate.
Data loss is possible if the database goes down. Information that the contractor uploads about their visit may be lost.
There is an increased risk of regulatory violations outlined in the Software Requirements Specification 17 - Compliance Requirements.
This system is a tool to effectively distribute documentation concerning station access to contractors. If they lose this access, there
is an increased risk they do not follow the correct entry/exit procedure for stations.
Operational interruption could occur if contractors are unable to view documentation automatically through the system, and
would instead use a slower manual process over email with a facilities manager to view documentation.

\section{System Components and Boundaries}
The scope of the Hazard Analysis covers the components and boundaries identified in this section. These system components
are responsible for implementing the functionality which makes the hazards identified in section 2 possible.
\subsection{System Components}
\begin{itemize}
  \item Document Management Module: This module is responsible for handling the
    storage, retrieval, categorization, and versioning of the documents.
  \item User Authentication and Authorization Module: This module is responsible
    for ensuring that only authorized users can access or modify the
    documents.
  \item User Presence Verification Module: This module is responsible for
    tracking and verifying the presence of users on site.
  \item Logging and Monitoring Module: This module is responsible for generating
    and monitoring the logs related to system usage, document changes, and
    user presence verification activities.
  \item Notification Module: This module is responsible for flagging to
    admin users about important events such as expiring documents, user training, or user presence
    verification issues.
  \item Database: The database stores all documents, user information, presence
    verification data, and logs.
\end{itemize}

\subsection{System Boundaries}
\begin{itemize}
  \item User Authentication: The authentication of users will be handled through
    the third party provider Amazon Cognito. The application will only interface with this
    provider. This trusted provider has established user authentication processes which mitigates the security risks
    and complexity of attempting to implement our own authentication system.
  \item System Infrastructure: This application will be cloud-based and the
    underlying infrastructure including computation, storage, and network
    resources will be maintained by a cloud service provider.

\end{itemize}
\section{Critical Assumptions}

\begin{enumerate}
  \item There are trained, trusted employees with appropriate access
    in the system to resolve failures should they occur.
  \item There are workflows and processes in place to complete
    necessary tasks should the application fail.
  \item External libraries used are trustworthy, secure, and will not
    be a cause of failure.
\end{enumerate}

\section{Failure Mode and Effect Analysis}

\begin{table}[H]
  \scalebox{0.6}{
    \begin{tabular}{|>{\raggedright}p{2.0cm}|>{\raggedright}p{2.0cm}|>{\raggedright}p{2.6cm}|>{\raggedright}p{4cm}|>{\raggedright}p{3cm}|>{\raggedright}p{4cm}|>{\raggedright}p{1.5cm}|p{1.5cm}|}
      \hline
      \textbf{Design Function} & \textbf{Failure Modes} & \textbf{Effects
      of Failure} & \textbf{Causes of Failure} & \textbf{Detection} &
      \textbf{Recommended Actions} & \textbf{SR} & \textbf{Ref.} \\
      \hline
      Geo-blocking & Geolocation is inaccurate & Contractor may be on
      site, but unable to follow entry/exit protocols & a. GPS
      Signal Blocked \newline b. Use of inaccurate Geolocation API
      \newline c. Out of date third-party databases \newline d.
      Device does not have a GPS \newline e. VPN usage & System
      provides error message to user when they get geo-blocked and
      sends an email to manager indicating that the user has been
      geo-blocked & a. User reports inaccurate location detection to
      manager & SR-AR1 & H1-1 \\
      \cline{2-8}
      & GPS signal spoofing & Contractors are able to access the system and
      upload documents even when they are not on site & a. Use of spoofing
      applications \newline b. Physical spoofing of GPS signals &
      System compares GPS location
      along with the ip address of the originating request & a. System will
      detect and report suspicious location activity to administrators & SR-AR1
      & H1-2 \\
      \hline
      User Authentication & Unauthorized Access & Comprimises system
      security & a. Weak protocols for user authentication &
      Geo-blocking & a. System should require multi-factor
      authentication \newline
      b. System should limit login attempts & SR-AR1 & H2-1 \\
      \cline{2-8}
      & User has higher access level than they should & User may be
      able to view or edit documents they should not be able to
      causing security concerns & a. Permission issue from database
      failure & System maintains detailed logbook of all views and
      edits to documents from within the system. Managers can
      subscribe to a document to get a notification when it is
      edited. & a. Allow administrator to adjust permissions &
      SR-AR2 \newline SR-AR3 \newline SR-IR1 & H2-2 \\
      \cline{2-8}
      & \sout{User unable to complete Two-Factor Authentication} \textcolor{red}{Moving to future development} & \sout{User
      unable to access application} & \sout{a. User name and email not in
      database \newline b. User did not receive authentication email
      \newline c. User unable to access email account \newline d.
      Email address in database outdated or incorrect
      }& \sout{Self-report by user} & \sout{a. User to report authentication issue
      to manager}& \sout{SR-AR1} & \sout{H2-3}\\
      \cline{2-8}
      & Failure to revoke permissions & Contractor retains access to system
      after they have completed project/no longer a contractor & a.
      Admin fails to
      remove access & Access logs & a. System automates permission
      revocation after
      contractor ceases to work on project & SR-IR1 \newline SR-AR1 & H2-4\\
      \hline
      View Station Entry Protocols & Contractor Views Old Version of
      Entry Protocols & Contractor may view and abide by out-of-date
      and incorrect station entry protocols & a. Database Error &
      Send all documents that a contractor viewed to their manager
      for verification upon completion & a. Manager can reupload
      newer version of entry protocol documents & SR-IR2 & H3-1 \\
      \cline{2-8}
      & Contractor Unable to Sign Documents & Contractor cannot start
      doing their work as they are unable to complete appropriate
      documentation & a. Permission issue from database failure &
      System provides detailed error message to user explaining that
      they cannot sign and guiding them on next steps & a. Allow
      administrator to adjust permissions & SR-AR1 \newline SR-AR4 & H3-2 \\
      \hline
    \end{tabular}
  }
  \caption{Failure Mode and Effect Analysis (FMEA) Table Part 1}
\end{table}

\begin{table}[H]
  \scalebox{0.6}{
    \begin{tabular}{|>{\raggedright}p{2.0cm}|>{\raggedright}p{2.0cm}|>{\raggedright}p{2.6cm}|>{\raggedright}p{4cm}|>{\raggedright}p{3cm}|>{\raggedright}p{4cm}|>{\raggedright}p{1.5cm}|p{1.5cm}|}
      \hline
      \textbf{Design Function} & \textbf{Failure Modes} & \textbf{Effects
      of Failure} & \textbf{Causes of Failure} & \textbf{Detection} &
      \textbf{Recommended Actions} & \textbf{SR} & \textbf{Ref.} \\
      \hline
      Access to web-application & User is unable to access the
      application through their device & User unable to use
      application functionality & a. Poor or no internet on device
      \newline b. Device or browser unsupported or unable to run application
      & Self-report by user & a. User to report to Manager \newline
      b. User to follow alternative workflow and processes to
      complete tasks on site
      to manager & SR-AR1 & H4-1 \\
      \cline{4-8}
      & & & c. Unable to access URL (QR code damaged or missing, no
      camera on device, User unable to find URL) & Self-report by
      user & a. User to Ask Manager for URL \newline b. Employees to
      provide new scannable QR code at location if applicable & SR-AR1 & H4-2 \\
      \hline
      Accept Contractor Document Upload & Document is valid but is uploaded in
      the wrong location & Poor discoverability and organization &
      a. Human-computer interface failure &User detects document is
      in the wrong category &
      a. Alert administrator to adjust & FR1 & H5-1 \\
      \cline{2-8}
      & Document is malicious and should not be in the system & Malware
      infection &a. Malicious user gained access to system & Malware
      scan identifies malicious documents
      & a. Application removes file from system and notifies
      administrator & SR-AR1
      & H5-2 \\
      \cline{2-8}
      & File chosen for upload is corrupted and not able to be opened &
      User fails to open file &a. Failure in document chosen or upload process&
      Error messages to user& a. Application removes file from system and
      notifies administrator & SR-IR3 & H5-3 \\
      \cline{2-8}
      & File type uploaded is not a supported type & System can't display
      file correctly to the user & a. Failure in file type detection
      \newline b. System permitted user to attempt upload of wrong file &
      Error message to user& a. System notifies user of incompatible file type &
      FR1 & H5-4 \\
      \cline{2-8}
      & The upload process is interrupted and the file upload is incomplete. &
      File upload fails & a. System connection interruption &
      Error messages to system& a. System notifies user & PR-SC1 & H5-5\\
      \hline
      Contractor acknowledges document & Contractor does not acknowledge
      correctly and it is not saved to the system & Acknowledgement not received
      & a. User does not understand feedback from the application interface
      \newline b. User does not agree to sign the document & Application detects
      unsuccessful acknowledgement state & a. System notifies user &
      FR7 SR-AR1 & H6-1 \\
      \cline{2-8}
      & Device loses power and interrupts the acknowledgement & Acknowledgement
      is not received by the system &
      a. Loss of power disconnects users from the application & Server side
      detects user disconnection & a. Save the application state and return
      user to previous state when they reconnect & FR7 SR-AR1 & H6-2 \\
      \hline
    \end{tabular}
  }
  \caption{Failure Mode and Effect Analysis (FMEA) Table Part 2}
\end{table}

\section{Safety and Security Requirements}

\begin{enumerate}[{SR-S}1.]
  \item The system shall notify the facilities manager if a contractor declines
    to sign a document.\\
    \newline Rationale: The facilities manager should be aware if
    their contractor
    declines to sign a document. This creates a health and safety concern.\\
    \newline Fit criterion: The facilities manager receives a
    notification from the
    application identifying the contractor and the document.
\end{enumerate}

\section{Roadmap}

All existing and new safety requirements will be implemented within
the timeline of this project. All documented safety requirements are
essential to the stakeholders, and must be implemented before the
application can be deployed for use. The date to fully implement the requirements outlined in the FMEA table in section 5
and identified in section 6 is March 24, 2025.

\newpage{}

\section*{Appendix --- Reflection}

The purpose of reflection questions is to give you a chance to assess your own
learning and that of your group as a whole, and to find ways to improve in the
future. Reflection is an important part of the learning process.  Reflection is
also an essential component of a successful software development process.  

Reflections are most interesting and useful when they're honest, even if the
stories they tell are imperfect. You will be marked based on your depth of
thought and analysis, and not based on the content of the reflections
themselves. Thus, for full marks we encourage you to answer openly and honestly
and to avoid simply writing ``what you think the evaluator wants to hear.''

Please answer the following questions.  Some questions can be answered on the
team level, but where appropriate, each team member should write their own
response:


\begin{enumerate}
  \item What went well while writing this deliverable?\\
    \\
    When writing this deliverable, we were able to efficiently discuss and
    delegate sections of the document to team members. We made good use of the
    issues and PR's on GitHub to track the status.
  \item What pain points did you experience during this deliverable, and how
    did you resolve them?\\
    \\
    One pain point was identifying new safety requirements and determining
    the best way to incorporate them into the HazardAnalysis.pdf
    document while maintaining
    consistency with the SRS.pdf. The team utilized a pull request to discuss
    the best approach. The team decided to give each safety requirements it's
    own unique label to resolve this problem.
  \item Which of your listed risks had your team thought of before this
    deliverable, and which did you think of while doing this deliverable? For
    the latter ones (ones you thought of while doing the Hazard Analysis), how
    did they come about?\\
    \\
    Our team had thought about poor internet signal at stations and also the
    risk of GPS technology not working properly. This deliverable allowed us to
    consider in more depths what the causes of failure could be and what actions
    the system should take to respond. Our team discovered new risks, including
    what would happen if a contractor decided not to sign a document. This led
    to the creation of a new safety requirement to address this issue.
  \item Other than the risk of physical harm (some projects may not have any
    appreciable risks of this form), list at least 2 other types of risk in
    software products. Why are they important to consider?\\
    \\
    One important risk to consider is the risk of poor user
    experience. Our application
    will be used by dozens of contractors, so the application
    must not be frustrating for the user or they will not utilize it
    and the benefits it has to offer.\\
    \\
    Another important risk is the risk of malicious documents entering the
    system. Applications which accept the input of data are naturally
    at a higher risk from malicious actors. The hazard analysis allowed our
    team to consider the importance of scanning uploaded documents for threats
    instead of trusting the documents which the system receives.
\end{enumerate}

\end{document}
