\documentclass{article}

\usepackage{booktabs}
\usepackage{tabularx}
\usepackage{hyperref}
\usepackage{graphicx}

\hypersetup{
  colorlinks=true,       % false: boxed links; true: colored links
  linkcolor=red,          % color of internal links (change box color
  % with linkbordercolor)
  citecolor=green,        % color of links to bibliography
  filecolor=magenta,      % color of file links
  urlcolor=cyan           % color of external links
}

\title{Hazard Analysis\\\progname}

\author{\authname}

\date{}

\input{../Comments}
\input{../Common}

\begin{document}

\maketitle
\thispagestyle{empty}

~\newpage

\pagenumbering{roman}

\begin{table}[hp]
  \caption{Revision History} \label{TblRevisionHistory}
  \begin{tabularx}{\textwidth}{llX}
    \toprule
    \textbf{Date} & \textbf{Developer(s)} & \textbf{Change}\\
    \midrule
    Date1 & Name(s) & Description of changes\\
    Date2 & Name(s) & Description of changes\\
    ... & ... & ...\\
    \bottomrule
  \end{tabularx}
\end{table}

~\newpage

\tableofcontents

~\newpage

\pagenumbering{arabic}

\wss{You are free to modify this template.}

\section{Introduction}

Hazards can be defined as either a state of the system or an action
taken on the system, which causes system failures. System failures
include, but are not limited to, an inability of the system to
perform its required functions, loss of data, unauthorized access,
or harm to users.

\section{Scope and Purpose of Hazard Analysis}

This document outlines a hazard analysis for SyncMaster, which is a system that
manages documentation and enables administrators to verify presence of users on
site. The purpose of this analysis is to identify and evaluate potential
hazards, outline how they will be detected/introduced to the system, and to
establish mitigation strategies. This analysis will also determine
the parties responsible
for each of the aforementioned points. Some of the losses that
may be incurred as a result of hazards are, but are not limited to:
financial loss, data loss, loss of trust, regulatory violations, and
operational interruption. These losses stemming from hazards must be mitigated
to ensure that the system operates securely, and in compliance with relevant
regulations.

\section{System Boundaries and Components}

\wss{Dividing the system into components will help you brainstorm the hazards.
  You shouldn't do a full design of the components, just get a feel
  for the major
  ones.  For projects that involve hardware, the components will
  typically include
  each individual piece of hardware.  If your software will have a
  database, or an
important library, these are also potential components.}

\section{Critical Assumptions}

\begin{enumerate}
  \item There are trained, trusted employees with appropriate access
    in the system to resolve failures should they occur.
  \item There are workflows and processes in place to complete
    necessary tasks should the application fail.
  \item External libraries used are trustworthy, secure, and will not
    be a cause of failure.
\end{enumerate}

\section{Failure Mode and Effect Analysis}

\wss{Include your FMEA table here. This is the most important part of
this document.}
\wss{The safety requirements in the table do not have to have the prefix SR.
  The most important thing is to show traceability to your SRS. You
  might trace to
  requirements you have already written, or you might need to add new
requirements.}
\wss{If no safety requirement can be devised, other mitigation strategies can be
  entered in the table, including strategies involving providing additional
documentation, and/or test cases.}

\begin{table}
  \scalebox{0.6}{
    \begin{tabular}{|>{\raggedright}p{2.0cm}|>{\raggedright}p{2.0cm}|>{\raggedright}p{2.5cm}|>{\raggedright}p{4cm}|>{\raggedright}p{3cm}|>{\raggedright}p{4cm}|>{\raggedright}p{1.5cm}|p{1.5cm}|}
      \hline
      \textbf{Design Function} & \textbf{Failure Modes} & \textbf{Effects
      of Failure} & \textbf{Causes of Failure} & \textbf{Detection} &
      \textbf{Recommended Actions} & \textbf{SR} & \textbf{Ref.} \\
      \hline
      Geo-blocking & Geolocation is inaccurate & Contractor may be on
      site, but unable to follow entry/exit protocols & a. GPS
      Signal Blocked \newline b. Use of inaccurate Geolocation API
      \newline c. Out of date third-party databases \newline d.
      Device does not have a GPS \newline e. VPN usage & System
      provides error message to user when they get geo-blocked and
      sends an email to manager indicating that the user has been
      geo-blocked & a. User reports inaccurate location detection to
      manager & SR-AR1 & H1-1 \\
      \hline
      User Authentication & Unauthorized Access & Comprimises system
      security & a. Weak protocols for user authentication &
      Geo-blocking & a. System should require multi-factor
      authentication \newline
      b. System should limit login attempts & SR-AR1 & H2-1 \\
      \cline{2-8}
      & User has higher access level than they should & User may be
      able to view or edit documents they should not be able to
      causing security concerns & a. Permission issue from database
      failure & System maintains detailed logbook of all views and
      edits to documents from within the system. Managers can
      subscribe to a document to get a notification when it is
      edited. & a. Allow administrator to adjust permissions &
      SR-AR2 \newline SR-AR3 \newline SR-IR1 & H2-2 \\
      \cline{2-8}
      & User unable to complete Two-Factor Authentication & User
      unable to access application & a. User name and email not in
      database \newline b. User did not receive authentication email
      \newline c. User unable to access email account \newline d.
      Email address in database outdated or incorrect
      & Self-report by user & a. User to report authentication issue
      to manager & SR-AR1 & H2-3 \\
      \hline
      View Station Entry Protocols & Contractor Views Old Version of
      Entry Protocols & Contractor may view and abide by out-of-date
      and incorrect station entry protocols & a. Database Error &
      Send all documents that a contractor viewed to their manager
      for verification upon completion & a. Manager can reupload
      newer version of entry protocol documents & SR-IR2 & H3-1 \\
      \cline{2-8}
      & Contractor Unable to Sign Documents & Contractor cannot start
      doing their work as they are unable to complete appropriate
      documentation & a. Permission issue from database failure &
      System provides detailed error message to user explaining that
      they cannot sign and guiding them on next steps & a. Allow
      administrator to adjust permissions & SR-AR1 \newline SR-AR4 & H3-2 \\
      \hline
      Access to web-application & User is unable to access the
      application through their device & User unable to use
      application functionality & a. Poor or no internet on device
      \newline b. Device or browser unsupported or unable to run application
      & Self-report by user & a. User to report to Manager \newline
      b. User to follow alternative workflow and processes to
      complete tasks on site
      to manager & SR-AR1 & H4-1 \\
      \cline{4-8}
      & & & c. Unable to access URL (QR code damaged or missing, no
      camera on device, User unable to find URL) & Self-report by
      user & a. User to Ask Manager for URL \newline b. Employees to
      provide new scannable QR code at location if applicable & SR-AR1 & H4-2 \\
      \hline
    \end{tabular}
  }
  \caption{Failure Mode and Effect Analysis (FMEA) Table}
\end{table}

\section{Safety and Security Requirements}

\wss{Newly discovered requirements.  These should also be added to the SRS.  (A
rationale design process how and why to fake it.)}

\section{Roadmap}

\wss{Which safety requirements will be implemented as part of the
  capstone timeline?
Which requirements will be implemented in the future?}

\newpage{}

\section*{Appendix --- Reflection}

\input{../Reflection.tex}

\begin{enumerate}
  \item What went well while writing this deliverable?\\
  \\
  When writing this deliverable, we were able to efficiently discuss and 
  delegate sections of the document to team members. We made good use of the
  issues and PR's on GitHub to track the status.
  \item What pain points did you experience during this deliverable, and how
    did you resolve them?\\
    \\
    One pain point was identifying new safety requirements and determining
    the best way to incorporate them into the HazardAnalysis.pdf document while maintaining
    consistency with the SRS.pdf. The team utilized a pull request to discuss
    the best approach. The team decided to give each safety requirements it's
    own unique label to resolve this problem.
  \item Which of your listed risks had your team thought of before this
    deliverable, and which did you think of while doing this deliverable? For
    the latter ones (ones you thought of while doing the Hazard Analysis), how
    did they come about?\\
    \\
    Our team had thought about poor internet signal at stations and also the
    risk of GPS technology not working properly. This deliverable allowed us to
    consider in more depths what the causes of failure could be and what actions
    the system should take to respond. Our team discovered new risks, including
    what would happen if a contractor decided not to sign a document. This led
    to the creation of a new safety requirement to address this issue.
  \item Other than the risk of physical harm (some projects may not have any
    appreciable risks of this form), list at least 2 other types of risk in
    software products. Why are they important to consider?\\
    \\
    One important risk to consider is the risk of poor user experience. Our application
    will be used by dozens of contractors, so the application
    must not be frustrating for the user or they will not utilize it
    and the benefits it has to offer.\\
    \\
    Another important risk is the risk of malicious documents entering the
    system. Applications which accept the input of data are naturally
    at a higher risk from malicious actors. The hazard analysis allowed our
    team to consider the importance of scanning uploaded documents for threats
    instead of trusting the documents which the system receives.
\end{enumerate}

\end{document}
