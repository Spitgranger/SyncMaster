\documentclass{article}

\usepackage{booktabs}
\usepackage{tabularx}
\usepackage{hyperref}
\usepackage{graphicx}
\usepackage{enumerate}

\hypersetup{
  colorlinks=true,       % false: boxed links; true: colored links
  linkcolor=red,          % color of internal links (change box color
  % with linkbordercolor)
  citecolor=green,        % color of links to bibliography
  filecolor=magenta,      % color of file links
  urlcolor=cyan           % color of external links
}

\title{Hazard Analysis\\\progname}

\author{\authname}

\date{}

%% Comments

\usepackage{color}

\newif\ifcomments\commentstrue %displays comments
%\newif\ifcomments\commentsfalse %so that comments do not display

\ifcomments
\newcommand{\authornote}[3]{\textcolor{#1}{[#3 ---#2]}}
\newcommand{\todo}[1]{\textcolor{red}{[TODO: #1]}}
\else
\newcommand{\authornote}[3]{}
\newcommand{\todo}[1]{}
\fi

\newcommand{\wss}[1]{\authornote{blue}{SS}{#1}} 
\newcommand{\plt}[1]{\authornote{magenta}{TPLT}{#1}} %For explanation of the template
\newcommand{\an}[1]{\authornote{cyan}{Author}{#1}}

%% Common Parts

\newcommand{\progname}{ProgName} % PUT YOUR PROGRAM NAME HERE
\newcommand{\authname}{Team \#, Team Name
\\ Student 1 name
\\ Student 2 name
\\ Student 3 name
\\ Student 4 name} % AUTHOR NAMES                  

\usepackage{hyperref}
    \hypersetup{colorlinks=true, linkcolor=blue, citecolor=blue, filecolor=blue,
                urlcolor=blue, unicode=false}
    \urlstyle{same}
                                


\begin{document}

\maketitle
\thispagestyle{empty}

~\newpage

\pagenumbering{roman}

\begin{table}[hp]
  \caption{Revision History} \label{TblRevisionHistory}
  \begin{tabularx}{\textwidth}{llX}
    \toprule
    \textbf{Date} & \textbf{Developer(s)} & \textbf{Change}\\
    \midrule
    Date1 & Name(s) & Description of changes\\
    Date2 & Name(s) & Description of changes\\
    ... & ... & ...\\
    \bottomrule
  \end{tabularx}
\end{table}

~\newpage

\tableofcontents

~\newpage

\pagenumbering{arabic}

\wss{You are free to modify this template.}

\section{Introduction}

Hazards can be defined as either a state of the system or an action
taken on the system, which causes system failures. System failures
include, but are not limited to, an inability of the system to
perform its required functions, loss of data, unauthorized access,
or harm to users.

\section{Scope and Purpose of Hazard Analysis}

\wss{You should say what \textbf{loss} could be incurred because of the
hazards.}

\section{System Boundaries and Components}

\wss{Dividing the system into components will help you brainstorm the hazards.
  You shouldn't do a full design of the components, just get a feel
  for the major
  ones.  For projects that involve hardware, the components will
  typically include
  each individual piece of hardware.  If your software will have a
  database, or an
important library, these are also potential components.}

\section{Critical Assumptions}

\wss{These assumptions that are made about the software or system.  You should
  minimize the number of assumptions that remove potential hazards.
  For instance,
  you could assume a part will never fail, but it is generally better to include
this potential failure mode.}

\section{Failure Mode and Effect Analysis}

\wss{Include your FMEA table here. This is the most important part of
this document.}
\wss{The safety requirements in the table do not have to have the prefix SR.
  The most important thing is to show traceability to your SRS. You
  might trace to
  requirements you have already written, or you might need to add new
requirements.}
\wss{If no safety requirement can be devised, other mitigation strategies can be
  entered in the table, including strategies involving providing additional
documentation, and/or test cases.}

\begin{table}
  \scalebox{0.6}{
    \begin{tabular}{|>{\raggedright}p{2.0cm}|>{\raggedright}p{2.0cm}|>{\raggedright}p{2.5cm}|>{\raggedright}p{4cm}|>{\raggedright}p{3cm}|>{\raggedright}p{4cm}|>{\raggedright}p{1.5cm}|p{1.5cm}|}
      \hline
      \textbf{Design Function} & \textbf{Failure Modes} & \textbf{Effects
      of Failure} & \textbf{Causes of Failure} & \textbf{Detection} &
      \textbf{Recommended Actions} & \textbf{SR} & \textbf{Ref.} \\
      \hline
      Geo-blocking & Geolocation is inaccurate & Contractor may be on
      site, but unable to follow entry/exit protocols & a. GPS
      Signal Blocked \newline b. Use of inaccurate Geolocation API
      \newline c. Out of date third-party databases \newline d.
      Device does not have a GPS \newline e. VPN usage & Error
      messages and logging & a. Report inaccurate location detection
      to manager & SR-AR1 & H1-1 \\
      \hline
      User Authentication & Unauthorized Access & Comprimises system
      security & a. Weak protocols for user authentication &
      Geo-blocking & a. Require multi-factor authentication \newline
      b. limit login attempts & SR-AR1 & H2-1 \\
      \cline{2-8}
      & User has higher access level than they should & User may be
      able to view or edit documents they should not be able to
      causing security concerns & a. Permission issue from database
      failure & detailed logs of all views and edits to documents
      from within the system & a. Allow administrator to adjust
      permissions & SR-AR2 \newline SR-AR3 \newline SR-IR1 & H2-2 \\
      \hline
      View Station Entry Protocols & Contractor Views Old Version of
      Entry Protocols & Contractor may view and abide by out-of-date
      and incorrect station entry protocols & a. Database Error &
      Send all documents that a contractor viewed to their manager
      for verification upon completion & a. Reupload newer version of
      entry protocol documents & SR-IR2 & H3-1 \\
      \cline{2-8}
      & Contractor Unable to Sign Documents & Contractor cannot start
      doing their work as they are unable to complete appropriate
      documentation & a. Permission issue from database failure &
      Error messages and logging & a. Allow administrator to adjust
      permissions & SR-AR1 \newline SR-AR4 & H3-2 \\
      \hline
      Accept Contractor Document Upload & Document is valid but is uploaded in
      the wrong location & Poor discoverability and organization &
      a. Human-computer interface failure &User detects &
      a. Alert administrator to adjust & FR1 & H4-1 \\
      \cline{2-8}
      & Document is malicious and should not be in the system & Malware
      infection &a. Malicious user gained access to system & Error messages and
      logging & a. Remove file from system and notify administrator & SR-AR1
      & H4-2 \\
      \cline{2-8}
      & File chosen for upload is corrupted and not able to be opened &
      User fails to open file &a. Failure in document chosen or upload process&
      Error messages and logging & a. Remove file from system and
      notify administrator & SR-IR3 & H4-3 \\
      \cline{2-8}
      & File type uploaded is not a supported type & System can't display
      file correctly to the user & a. Failure in file type detection
      \newline b. System permitted user to attempt upload of wrong file &
      Error message and logging & a. Notify user of incompatible file type &
      FR1 & H4-4 \\
      \cline{2-8}
      & The upload process is interrupted and the file upload is incomplete. &
      File upload fails & a. System connection interruption &
      Error messages and logging of file name & a. Notify user & PR-SC1 & H4-5\\
      \hline
      Contractor acknowledges document & Contractor does not acknowledge
      correctly and it is not saved to the system & Acknowledgement not received
      & a. User does not understand feedback from the application interface
      \newline b. User does not agree to sign the document & Application detects
      unsuccessful acknowledgement state & a. Notify user and log failure &
      FR7 & H5-1 \\
      \cline{2-8}
      & Device loses power and interrupts the acknowledgement & Acknowledgement
      is not received by the system &
      a. Loss of power disconnects users from the application & Server side
      detects user disconnection & a. Save the application state and return
      user to previous state when they reconnect & FR7 & H5-2 \\
      \hline
    \end{tabular}
  }
  \caption{Failure Mode and Effect Analysis (FMEA) Table}
\end{table}

\section{Safety and Security Requirements}

\wss{Newly discovered requirements.  These should also be added to the SRS.  (A
rationale design process how and why to fake it.)}
\begin{enumerate}[{SFR}1.]
  \item The system shall notify the facilities manager if a contractor declines
    to sign a document.\\
    \newline Rationale: The facilities manager should be aware if
    their contractor
    declines to sign a document. This creates a health and safety concern.\\
    \newline Fit criterion: The facilities manager receives a
    notification from the
    application identifying the contractor and the document.
\end{enumerate}

\section{Roadmap}

\wss{Which safety requirements will be implemented as part of the
  capstone timeline?
Which requirements will be implemented in the future?}

\newpage{}

\section*{Appendix --- Reflection}

\wss{Not required for CAS 741}

The purpose of reflection questions is to give you a chance to assess your own
learning and that of your group as a whole, and to find ways to improve in the
future. Reflection is an important part of the learning process.  Reflection is
also an essential component of a successful software development process.  

Reflections are most interesting and useful when they're honest, even if the
stories they tell are imperfect. You will be marked based on your depth of
thought and analysis, and not based on the content of the reflections
themselves. Thus, for full marks we encourage you to answer openly and honestly
and to avoid simply writing ``what you think the evaluator wants to hear.''

Please answer the following questions.  Some questions can be answered on the
team level, but where appropriate, each team member should write their own
response:


\begin{enumerate}
  \item What went well while writing this deliverable?
  \item What pain points did you experience during this deliverable, and how
    did you resolve them?
  \item Which of your listed risks had your team thought of before this
    deliverable, and which did you think of while doing this deliverable? For
    the latter ones (ones you thought of while doing the Hazard Analysis), how
    did they come about?
  \item Other than the risk of physical harm (some projects may not have any
    appreciable risks of this form), list at least 2 other types of risk in
    software products. Why are they important to consider?
\end{enumerate}

\end{document}
