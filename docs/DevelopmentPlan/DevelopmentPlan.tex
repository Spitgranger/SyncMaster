\documentclass{article}

\usepackage{booktabs}
\usepackage{tabularx}
\usepackage{longtable}

\title{Development Plan\\\progname}

\author{\authname}

\date{}

%% Comments

\usepackage{color}

\newif\ifcomments\commentstrue %displays comments
%\newif\ifcomments\commentsfalse %so that comments do not display

\ifcomments
\newcommand{\authornote}[3]{\textcolor{#1}{[#3 ---#2]}}
\newcommand{\todo}[1]{\textcolor{red}{[TODO: #1]}}
\else
\newcommand{\authornote}[3]{}
\newcommand{\todo}[1]{}
\fi

\newcommand{\wss}[1]{\authornote{blue}{SS}{#1}} 
\newcommand{\plt}[1]{\authornote{magenta}{TPLT}{#1}} %For explanation of the template
\newcommand{\an}[1]{\authornote{cyan}{Author}{#1}}

%% Common Parts

\newcommand{\progname}{ProgName} % PUT YOUR PROGRAM NAME HERE
\newcommand{\authname}{Team \#, Team Name
\\ Student 1 name
\\ Student 2 name
\\ Student 3 name
\\ Student 4 name} % AUTHOR NAMES                  

\usepackage{hyperref}
    \hypersetup{colorlinks=true, linkcolor=blue, citecolor=blue, filecolor=blue,
                urlcolor=blue, unicode=false}
    \urlstyle{same}
                                


\begin{document}

\maketitle

\begin{table}[hp]
  \caption{Revision History} \label{TblRevisionHistory}
  \begin{tabularx}{\textwidth}{llX}
    \toprule
    \textbf{Date} & \textbf{Developer(s)} & \textbf{Change}\\
    \midrule
    2024/09/19 & Richard Fan & Conversion of tex to markdown\\
    2024/09/24 & Whole Team & Initial development plan document completed\\
    2025/03/15 & Mitchell Hynes & Convert from markdown into \LaTeX \\
    \bottomrule
  \end{tabularx}
\end{table}

\newpage{}

This document outlines the development plan for the project. Further information regarding the problem is
located in
\href{https://github.com/Spitgranger/SyncMaster/blob/main/docs/ProblemStatementAndGoals/ProblemStatement.pdf}{ProblemStatement.pdf}.

\section{Confidential Information?}

This project will not contain extensive confidential information warranting a confidentiality agreement. After a
team meeting with MILO, it was determined our team will inform the City to advise us of any confidential
information. We will ensure it is not included in the source code or at the capstone expo.

\section{IP to Protect}

The project does not have any IP to protect.

\section{Copyright License}

Our team will be licencing our project under the MIT license\\
(https://github.com/Spitgranger/capstone/blob/main/LICENSE). 
As our project is done in collaboration with
industry this license makes sense due to its flexibility. It will allow our industry partners to freely use and
modify the code as they see fit in both open source and proprietary projects. This may be the case after the
conclusion of this capstone project, hence the choice of this license.


\section{Team Meeting Plan}

Weekly team meetings will occur on Monday evenings. Meetings to be held online or in-person depending on
the groups preference that week.\\
\\
We will meet with our industry advisor bi-weekly, but this could be more or less depending on the needs of
the capstone team as we progress through the deliverables. Meetings will be on Microsoft Teams, with the
option to meet in-person if it is convenient for the particular meeting. Meetings will be structured with a
prepared agenda by the capstone team and the discussion with the stakeholder chaired by the team liaison.
Meeting minutes and action items will be recorded by the capstone team.


\section{Team Communication Plan}

\begin{longtable}{|m{5cm}|m{8cm}|}
  \hline
  \textbf{Type} & \textbf{Method}\\
  \hline
  Regular communications & Discord\\
  \hline
  Assignment of tasks & GitHub issues\\
  \hline
  Communications with professors, supervisor, the City, and other stakeholders. & Handled by Team Liaison Mitchell Hynes,
  in-person or over email.\\
  \hline
  Meetings & Discord, scheduled through outlook calendar.\\
  \hline
\end{longtable}

\section{Team Member Roles}

Meeting Chair: This position will coordinate meetings, ensure an agenda is set, and that all parties are
prepared for the meeting.\\
\\
Developer: Lead developer for project, subject matter expert on technologies being used by the team.\\
\\
Project Manager: Responsible for ensuring that all team members are up to date on work being performed
and that action items for needed tasks are completed.\\
\\
Team Liaison: Mitchell is the team liaison. They are responsible for maintaining communications between the
McMaster instructional team and the City for the group.\\
\\
Reviewer: This role is responsible for reviewing other work and ensuring the quality is at a required standard.
This includes reviewing commits and pull requests.

\section{Workflow Plan}
\textbf{Git Workflow}\\
\\
For this project, Git Feature Branch workflow will be utilized for feature development. This strategy will enable
the development of new features in isolated branches, enhancing collaboration and reducing the risk of
introducing bugs in the main branch. In general branches will follow the naming convention of /. Some
examples are given below:
\begin{itemize}
  \item feature/$<$name of feature$>$
  \item fix/$<$name of fix$>$
  \item docs/$<$name of doc changed$>$
\end{itemize}
\textbf{Pull Requests}\\
\\
Pull Requests (PRs) will be created by the collaborators of the project to merge their respective feature
branches into the main branch. Upon opening a PR, a short description of what the changes are should be
provided. PRs should be linked to at least one issue in the repository to ensure traceability of work. This
workflow will ensure that minimal conflicts are introduced and also will reduce the probability of introducing
bugs into the main branch. To merge a PR, it will need to be approved by at least one of the collaborators.\\
\\
\textbf{Github Project, Github Issues, Github Milestones, and Labels/Tags}\\
\\
Github Projects along with Github Issues will be utilized for the management and tracking of tasks in this
project. Github Projects will be used to track issues. Our project will classify the status of the issue as: "No
Status" (the issue has no status assigned to it), "Backlog" (we need to refine the issue by for example, adding
more details to it issue before it is ready to be picked up), "Ready" (issue is ready to be picked up for
development), "In Progress" (issue is currently in development), "In Review" (issue is currently under code
review), "Done" (issue has been completed). Additionally, issues will be assigned one of the three priority
levels: low, medium, and high. The issues will be created with different custom predefined templates to
standardize the process and save time containing the following details at a minimum:

\begin{itemize}
\item An outline of what needs to be done
\item A soft deadline on when the work should be done
\item A short description of what it means for this issue to be done/closable
\end{itemize}
Github Milestones will be used to keep track of the changes which need to be done for the current/next
deliverable (e.g., "Dev Plan and Problem Statement" deliverable).\\
\\
Issues will be classified using the following labels (the following is subject to change):
\begin{itemize}
\item Feature - For new development features
\item Documentation - For adding or modifying existing documentation
\item Bug - For fixing and tracking bugs
\item Chore - Simple tasks such as removing comments
\item Improvement - An improvement to the existing codebase (e.g. efficiency, complexity)
\end{itemize}
A minimum of one person will be assigned to each issue.\\
\\
Additionally, custom issue templates will be used to streamline the process and to ensure all the necessary
information is provided while creating the issue.\\
\\
Tags will be utilized to mark any significant releases (for example, “poc-release”).\\
\\
\textbf{CI/CD}\\
\\
For continuous integration (CI), we will be using Github Actions Workflows which will involve running linters,
tests (like unit tests) on the project. Additionally, actions will be created to automate builds and to ensure that
no breaking changes are introduced via pull requests. We currently plan to start out by introducing an action
to perform linting and formatting, as well as an action to perform unit testing. These actions will run anytime a
pull request is made against the main branch. In the future, we plan to set up actions to automatically update
dependencies to mitigate security risks.\\
\\
For continuous deployment (CD), we will create Github Actions for deploying the code into staging (testing)
and production environments.


\section{Project Decomposition and Scheduling}

\begin{itemize}
\item We will be using Github Projects to decompose our milestones. Each Milestone will be divided into
  individual tasks which will then be assigned to different team members.
\item Each document will be split into sections, and assigned individually or to groups as needed.
\item The proof of concept, revision 0 and revision 1 will be split feature-wise and assigned to individuals or
groups as necessary.
\item Time will be alloted for verification, validation and review after the completion of each milestone and
before the due date.
\item \href{https://github.com/users/Spitgranger/projects/2/views/1}{Our GitHub Project}
\end{itemize}
\newpage
\begin{longtable}{|m{4cm}|m{6cm}|}
\hline
\textbf{Milestone} & \textbf{Schedule}\\
\hline
Requirements Document Revision 0 & Start: September 24th\\
Due: October 9th & End: October 5th\\
\hline
Hazard Analysis 0 & Start: October 9th\\
Due: October 23rd & End: October 19th\\
\hline
V\&V Plan Revision 0 & Start: October 16th\\
Due: November 1st & End: October 28th\\
\hline
Proof of Concept Demonstration & Start: after completion of the Requirements document\\
Due: November 11-22 & End: by November 4th, with improvements being made and testing
being done until the day of our demonstration\\
\hline
Design Document Revision 0 & Start: after completion of the Proof of Concept demo\\
Due: January 15th & End: by January 11th\\
\hline
Revision 0 Demonstration & Start: alongside proof of concept (after finalizing requirements in the meeting with the City)\\
Due: February 3-14 & End: by February 1st\\
\hline
V\&V Report Revision 0 & Start: alongside the development of Revision 0 of the project\\
Due: March 7th & End: by March 4th\\
\hline
Final Demonstration (Revision 1) & Start: after the completion of Revision 0\\
Due: March 29th & End: by March 20th\\
\hline
EXPO Demonstration & Start: 2 weeks before the day of the expo\\
Due: April 8th & End: April 1st\\
\hline
Final Documentation (Revision 1) & Start: March 10th\\
Due: April 2nd & End: by March 30th\\
\hline
\end{longtable}

\section{Proof of Concept Demonstration Plan}

One of the main goals for this project is contractor transparency and accountability when working on
unsupervised city property. This leads us to one of the main risks of this project: ensuring that geolocation
data provided by contractor devices is accurate enough for location verification. Our system plans on using
contractor laptops to ensure that contractors are actually on site when they go through the process of
reviewing and uploading documents. To do this, we have to ensure that the location reported by laptops is
accurate up to 10m to accommodate for the smallest sites. However, since most laptops do not have GPS, the
location may only be accurate to a few hundred metres, which could pose a problem when it comes to
verifying contractor presence on site. To mitigate this risk, we plan on tying a mobile device with GPS into the
system. We will use the user’s phone to provide us with more accurate location data. Our proof of concept will
rely on demonstrating that a user’s location can be accurately and reliably given to the system through a
simple scan of a QR code. The POC will then be as follows: a simple web application running on a laptop that
allows users to view content only after their location is verified by scanning a QR code on their phone.

\section{Expected Technology}

We will be using Git for version control, the repository for the project will be hosted on GitHub and we will be
using GitHub projects to track issues and schedule tasks for the project. The programming languages that we
intend to use for this project are Javascript for our frontend and Python for our backend. On the frontend, we
will use the Next.js framework. We intend on using AWS to host our application, for this, we will use AWS
Cloudformation templates to define our infrastructure.\\
\\
Some of the linter tools we will be using are Ruff, Bandit, and Pylint for Python. We will use cfn-lint for linting
our Cloudformation templates. For Javascript, we will be using Prettier and ESLint. In terms of testing
frameworks, we will be using Jest for Javascript and Pytest for Python. For Python, a code coverage tool we
intend on using is Pytest-cov.\\
\\
For the CI we will be creating Github actions workflows for running linters, formatters and unit tests on the
project. We will also create actions to automate builds to ensure PRs do not contain breaking changes, and
create actions for deploying the application into production and testing environments.\\
\\
Of course, the technology mentioned here is not final and is subject to change as the project progresses.

\section{Coding Standard}

To enhance uniformity across developers and adherence to best practices, the following coding standard will
be adopted:
\begin{itemize}
  \item \textbf{Linting:} As mentioned above, we will be using ESLint with @typescript/es-lint plugin for TypeScript
  code and Pylint for Python code. The default ruleset for both tools will be used to enforce best
  practices pertaining to both syntax and semantics as identified by the community.\\
  \item \textbf{Code Style and Formatting:} Prettier will be the main code formatter used for this project. Prettier was
  chosen because of its opinionated nature by default with regard to formatting, as well as compatibility
  and popularity with the TypeScript language. Code written in Python will use Ruff. Naming conventions
  of variable and function names will follow the style guides provided by the creators/maintainers of the
  aforementioned tools.\\
  \item \textbf{Code Documentation:} To enhance readability and maintainability, all code which does not perform a
  trivial task shall be documented. When documentation is needed, it will follow JSDoc or Pydoc
  conventions so that auto-generated documentation can be produced.\\
  \item \textbf{Testability:} Although subjective, the team should try their best to write code in a way that is testable.
  This includes following the principles of high cohesion, low coupling and single responsibility.\\
\end{itemize}

\newpage{}

\section*{Appendix --- Reflection}

\begin{enumerate}
  \item Why is it important to create a development plan prior to starting the
    project?\\
    \\
    It is well known that figuring out what to do is harder than actually doing it. We have experienced this
issue firsthand, and are aware of the value that a well-written development plan provides. We found
that through the creation of the development plan, ambiguities surrounding the project became
clearer. For example, we were able to identify strengths and weaknesses in the team’s skill set which
allowed us to get started filling ourselves in on any knowledge gaps before starting the project.
Additionally, creating this development plan allowed us to research potential technologies relevant to
our project before starting, giving us time to think about alternatives and choose the best tools for the
job. Finally, it encouraged us to start communication with our stakeholders and allowed us to develop
some common ground about the project, especially when it came to expectations and goals. So in
summary, creating a well structured and organized development plan allows everyone to stay on track
and provides milestones to see if the project is progressing smoothly. It also gives an idea of what to
expect for the project and helps align the team and stakeholders prior to starting a project.

  \item In your opinion, what are the advantages and disadvantages of using
    CI/CD?\\
    \\
    CI/CD has several advantages but also has a couple of major disadvantages that are often not talked
about. CI/CD makes development at every stage cyclical and reduces time between development cycles
and release. The features such as the ability to set up pipelines to do automated continuous testing,
builds, and deployments with every incremental change allows for flexibility in terms of catching errors
and fixing them. This leads to improved quality as bugs can be caught early and fixed, reducing
technical debt and putting the focus on current goal. CI/CD also perfectly compliments modern
development ideologies such as scrum/agile which prioritize regular review and multiple iterations.
However CI/CD also has several disadvantages, most notably giving developers a false sense of security
when it comes to the correctness and quality of their code. This leads to developer “laziness”, an effect
where developers are less likely to scrutinize and manually check what they have written. This is
because the automated checks and builds are not perfect, but the illusion of them being perfect is
there. Developers will often see that all checks have passed and assume that their code is correct, but
this is false; checks passing do not necessitate correctness.

  \item What disagreements did your group have in this deliverable, if any,
    and how did you resolve them?\\
    \\
    We did not have many disagreements about this deliverable. One small disagreement was over the tech
stack. We were debating whether to use Firebase or using AWS. The arguments in favour of Firebase
were that from research it looks like it is easier to set up and easier for a complete beginner to use. The
arguments in favour of AWS are that there are a couple of team members who are more familiar with it
and have used it before. We voted and decided to go with AWS because some team members had
experience with it.
\end{enumerate}

\newpage{}

\section*{Appendix --- Team Charter}

Borrows from
  \href{https://engineering.up.edu/industry_partnerships/files/team-charter.pdf}
{University of Portland Team Charter}.

\subsection*{External Goals}

The goal of the project is to produce a well developed piece of software per the requirements of the course
 which the City is satisfied with and could use in their workflow.

\subsection*{Attendance}

\subsubsection*{Expectations}

Team members are expected to attend all scheduled meetings as reasonably possible. If a team member
needs to miss a meeting, they should inform the team beforehand. During the meeting, team members are
expected to be on time and stay for the entire duration of the meeting unless prior arrangements have been
made, something urgent comes up during the meeting (see Acceptable Excuse), or the meeting goes over the
scheduled amount of time.

\subsubsection*{Acceptable Excuse}

Acceptable excuses would include personal/family emergencies as well as conflicts and events beyond one’s
control such as a sudden power outage or a commitment that cannot be rescheduled. These excuses,
however, should be kept to a minimum and each team member shall do their best to meet deadlines and
attend meetings. Unacceptable excuses would include, but are not limited to:\\
\begin{itemize}
\item Poor time management: Claiming that there wasn’t enough time or forgetting
\item Personal issues that are not urgent: Being tired, or sleeping in
\item Last Minute Non-Urgent Events: Missing a meeting or deadline to attend a party at the last minute
\end{itemize}
The examples above provide a guideline as to what is acceptable or unacceptable. Ultimately excuses will be
considered by the team on an ad-hoc basis so that the best solution can be agreed upon given the
circumstances.

\subsubsection*{In Case of Emergency}

If a team member has an emergency, team members will inform the rest of the team at their earliest
convenience to ensure the other team members are aware that their responsibilities in the project will be
impacted.\\
\\
In case of an emergency, a team member will:
\begin{enumerate}
  \item Write a message in the project discord at their earliest convenience regarding what work will be missed,
  and during what time frame they may be unavailable
  \item Email team liaison Mitchell regarding the emergency to leave a record which can be shared with the
  course coordinators if necessary
\end{enumerate}

\subsection*{Accountability and Teamwork}

\subsubsection*{Quality}

Each team member is expected to come prepared with the agreed upon meeting prerequisites. This includes
any action items, questions, or code reviews. Each team member to is to put forth their best effort in the
course and ask for help from their teammates or the instructional team if needed to meet the course
deliverables at a high quality of work.

\subsubsection*{Attitude}

Our team will be using the following as our code of conduct:
\begin{itemize}
\item Respect: All team members are to be treated with respect and courtesy. Diverse contributions and
perspectives of each team member are to be valued.
\item Communication: All team members are expected to communicate openly and honestly with use of
clear and concise language to ensure no misunderstandings arise.
\item Commitment: All team members are expected to make an honest effort to attend all the meetings and
complete all assigned tasks on time and to the best of their abilities.
\item Integrity: All team members are to uphold academic integrity and avoid plagiarism. Team members are
also expected to be candid about their progress and any difficulties currently faced by them.
\end{itemize}
If any conflicts arise, team members are to first identify the issue by describing the problem clearly along with
any relevant information. After the problem has been identified, a meeting will be held with the parties
involved for an open discussion concerning the problem at hand. Each person in the meeting is encouraged
to provide their insight. If necessary, a neutral third party will be involved. Once a mutually agreed solution is
found, a document containing the record of the resolution and any action items will be created to ensure that
all parties involved understand the agreement. Finally, the situation will be monitored until the issue is fully
resolved.

\subsubsection*{Stay on Track}

To keep the team on track we will use GitHub projects to keep track of tasks we need to complete and their
deadlines. We will ensure that team members contribute to the team by assigning tasks for deliverables
equally amongst the team so everyone has an equal workload and knows what they need to do. When
someone’s performance is below expectations we will first try to figure out internally in the team if there is
anything that can be done to improve their performance. If nothing can be found within the team, then we
will contact our TA for advice. A punishment for performing below expectations might include assigning more
work to the individual in the following deliverables. For team members who do well, we will reward them by
giving them Timbits at the end of the course.\\
\\
The performance of team members will be judged based on the following:
\begin{itemize}
  \item Completion of all assigned issues before their deadlines
  \item Capstone lecture attendance for software engineering lectures* $>=$ 50\%
  \item Internal meeting attendance* $>=$ 80\%
  \item Stakeholder meeting attendance* $>=$ 90\%
  \item For commits, we will have to see how often things are getting committed into the repository. Still, as a
  general rule, we can say that if a group member has 35\% fewer commits than the average number of
  commits for all other team members and also 35\% fewer additions than the average for all other team
  members, then they are likely not meeting expectations.
\end{itemize}
*For attendance-based goals if someone has an exception for some meetings, which is covered by the
acceptable excuses, then the meetings that they were excused for are not counted towards their attendance.\\
\\
If team members do not meet their goals we will first internally discuss with the team member why they were
unable to achieve the goals and what can be done to prevent it from happening going forward. If no
resolution can be reached we will discuss the situation with our TA or the instructor. We will not implement
punishments like “the team member needs to bring coffee to the next meeting” because this is not productive
and generally does nothing in the way of solving the root issues. There won’t be any particular incentives for
reaching targets early other than not having to worry about reaching the targets.
\subsubsection*{Team Building}

We have been studying in the same program for nearly 5 years now, and so there are many shared
experiences which bring us together as a team. While it may not be necessary to schedule additional activities
for team building, it may prove beneficial for stress release and increasing team cohesion.
\begin{itemize}
\item Team Lunches: Between classes, days when everyone is on campus and has time free around noon.
\item Ad-hoc team activities: Other fun activities can be scheduled according to the free time of the team
members
\end{itemize}

\subsubsection*{Decision Making}

To make decisions, the group member who's role is to direct that task will organize the rest of the group
around their respective action items. All team members will be able to contribute their opinion, and in general
decisions will require a consensus amongst the group. If an issue is not resolved through discussion, it will be
brought to the attention of the instructional team for mediation and feedback.

\end{document}
